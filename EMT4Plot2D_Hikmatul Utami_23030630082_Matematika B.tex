% Options for packages loaded elsewhere
\PassOptionsToPackage{unicode}{hyperref}
\PassOptionsToPackage{hyphens}{url}
\documentclass[
]{book}
\usepackage{xcolor}
\usepackage{amsmath,amssymb}
\setcounter{secnumdepth}{-\maxdimen} % remove section numbering
\usepackage{iftex}
\ifPDFTeX
  \usepackage[T1]{fontenc}
  \usepackage[utf8]{inputenc}
  \usepackage{textcomp} % provide euro and other symbols
\else % if luatex or xetex
  \usepackage{unicode-math} % this also loads fontspec
  \defaultfontfeatures{Scale=MatchLowercase}
  \defaultfontfeatures[\rmfamily]{Ligatures=TeX,Scale=1}
\fi
\usepackage{lmodern}
\ifPDFTeX\else
  % xetex/luatex font selection
\fi
% Use upquote if available, for straight quotes in verbatim environments
\IfFileExists{upquote.sty}{\usepackage{upquote}}{}
\IfFileExists{microtype.sty}{% use microtype if available
  \usepackage[]{microtype}
  \UseMicrotypeSet[protrusion]{basicmath} % disable protrusion for tt fonts
}{}
\makeatletter
\@ifundefined{KOMAClassName}{% if non-KOMA class
  \IfFileExists{parskip.sty}{%
    \usepackage{parskip}
  }{% else
    \setlength{\parindent}{0pt}
    \setlength{\parskip}{6pt plus 2pt minus 1pt}}
}{% if KOMA class
  \KOMAoptions{parskip=half}}
\makeatother
\usepackage{graphicx}
\makeatletter
\newsavebox\pandoc@box
\newcommand*\pandocbounded[1]{% scales image to fit in text height/width
  \sbox\pandoc@box{#1}%
  \Gscale@div\@tempa{\textheight}{\dimexpr\ht\pandoc@box+\dp\pandoc@box\relax}%
  \Gscale@div\@tempb{\linewidth}{\wd\pandoc@box}%
  \ifdim\@tempb\p@<\@tempa\p@\let\@tempa\@tempb\fi% select the smaller of both
  \ifdim\@tempa\p@<\p@\scalebox{\@tempa}{\usebox\pandoc@box}%
  \else\usebox{\pandoc@box}%
  \fi%
}
% Set default figure placement to htbp
\def\fps@figure{htbp}
\makeatother
\setlength{\emergencystretch}{3em} % prevent overfull lines
\providecommand{\tightlist}{%
  \setlength{\itemsep}{0pt}\setlength{\parskip}{0pt}}
\usepackage{bookmark}
\IfFileExists{xurl.sty}{\usepackage{xurl}}{} % add URL line breaks if available
\urlstyle{same}
\hypersetup{
  hidelinks,
  pdfcreator={LaTeX via pandoc}}

\author{}
\date{}

\begin{document}
\frontmatter

\mainmatter
\chapter{Menggambar Grafik 2D dengan EMT}\label{menggambar-grafik-2d-dengan-emt}

Notebook ini menjelaskan tentang cara menggambar berbagaikurva dan grafik 2D dengan software EMT. EMT menyediakan fungsi plot2d() untuk menggambar berbagai kurva dan grafik dua dimensi (2D).

\section{Plot Dasar}\label{plot-dasar}

Ada fungsi plot yang sangat mendasar. Terdapat koordinat layar yang selalu berkisar antara 0 hingga 1024 di setiap sumbu, tidak peduli apakah layarnya berbentuk persegi atau tidak. Semut terdapat koordinat plot, yang dapat diatur dengan setplot(). Pemetaan antar koordinat bergantung pada jendela plot saat ini. Misalnya, shrinkwindow() default menyisakan ruang untuk label sumbu dan judul plot.

Dalam contoh ini, kita hanya menggambar beberapa garis acak dengan berbagai warna. Untuk rincian tentang fungsi-fungsi ini, pelajari fungsi inti EMT.

\textgreater clg; // clear screen

\textgreater window(0,0,1024,1024); // use all of the window

\textgreater setplot(0,1,0,1); // set plot coordinates

\textgreater hold on; // start overwrite mode

\textgreater n=100; X=random(n,2); Y=random(n,2); // get random points

\textgreater colors=rgb(random(n),random(n),random(n)); // get random colors

\textgreater loop 1 to n; color(colors{[}\#{]}); plot(X{[}\#{]},Y{[}\#{]}); end; // plot

\textgreater hold off; // end overwrite mode

\textgreater insimg; // insert to notebook

\begin{figure}
\centering
\pandocbounded{\includegraphics[keepaspectratio]{images/EMT4Plot2D_Hikmatul Utami_23030630082_Matematika B-001.png}}
\caption{images/EMT4Plot2D\_Hikmatul\%20Utami\_23030630082\_Matematika\%20B-001.png}
\end{figure}

\textgreater reset;

Grafik perlu ditahan, karena perintah plot() akan menghapus jendela plot.

Untuk menghapus semua yang kami lakukan, kami menggunakan reset().

Untuk menampilkan gambar hasil plot di layar notebook, perintah plot2d() dapat diakhiri dengan titik dua (:). Cara lainnya adalah perintah plot2d() diakhiri dengan titik koma (;), kemudian menggunakan perintah insimg() untuk menampilkan gambar hasil plot.

Contoh lain, kita menggambar plot sebagai sisipan di plot lain. Hal ini dilakukan dengan mendefinisikan jendela plot yang lebih kecil. Perhatikan bahwa jendela ini tidak memberikan ruang untuk label sumbu di luar jendela plot. Kita harus menambahkan beberapa margin untuk ini sesuai kebutuhan. Perhatikan bahwa kita menyimpan dan memulihkan jendela penuh, dan menahan plot saat ini sementara kita memplot inset.

\textgreater plot2d(``x\^{}3-x'');

\textgreater xw=200; yw=100; ww=300; hw=300;

\textgreater ow=window();

\textgreater window(xw,yw,xw+ww,yw+hw);

\textgreater hold on;

\textgreater barclear(xw-50,yw-10,ww+60,ww+60);

\textgreater plot2d(``x\^{}4-x'',grid=6):

\begin{figure}
\centering
\pandocbounded{\includegraphics[keepaspectratio]{images/EMT4Plot2D_Hikmatul Utami_23030630082_Matematika B-002.png}}
\caption{images/EMT4Plot2D\_Hikmatul\%20Utami\_23030630082\_Matematika\%20B-002.png}
\end{figure}

\textgreater hold off;

\textgreater window(ow);

Plot dengan banyak gambar dicapai dengan cara yang sama. Ada fungsi utilitas figure() untuk ini.

\section{Aspek Plot}\label{aspek-plot}

Plot default menggunakan jendela plot persegi. Anda dapat mengubahnya dengan fungsi aspek(). Jangan lupa untuk mengatur ulang aspeknya nanti. Anda juga dapat mengubah default ini di menu dengan ``Set Aspect'' ke rasio aspek tertentu atau ke ukuran jendela grafik saat ini.

Tapi Anda juga bisa mengubahnya untuk satu plot. Untuk ini, ukuran area plot saat ini diubah, dan jendela diatur sehingga label memiliki cukup ruang.

\textgreater aspect(2); // rasio panjang dan lebar 2:1

\textgreater plot2d({[}``sin(x)'',``cos(x)''{]},0,2pi):

\begin{figure}
\centering
\pandocbounded{\includegraphics[keepaspectratio]{images/EMT4Plot2D_Hikmatul Utami_23030630082_Matematika B-003.png}}
\caption{images/EMT4Plot2D\_Hikmatul\%20Utami\_23030630082\_Matematika\%20B-003.png}
\end{figure}

\textgreater aspect();

\textgreater reset;

Fungsi reset() mengembalikan default plot termasuk rasio aspek.

\chapter{Plot 2D di Euler}\label{plot-2d-di-euler}

EMT Math Toolbox memiliki plot dalam 2D, baik untuk data maupun fungsi. EMT menggunakan fungsi plot2d. Fungsi ini dapat memplot fungsi dan data.

Dimungkinkan untuk membuat plot di Maxima menggunakan Gnuplot atau dengan Python menggunakan Math Plot Lib.

Euler dapat membuat plot 2D + ekspresi + fungsi, variabel, atau kurva berparameter, + vektor nilai x-y, + awan titik di pesawat, + kurva implisit dengan level atau wilayah level. + Fungsi kompleks

Gaya plot mencakup berbagai gaya untuk garis dan titik, plot batang, dan plot berbayang.

\chapter{Plot Ekspresi atau Variabel}\label{plot-ekspresi-atau-variabel}

Ekspresi tunggal dalam ``x'' (misalnya ``4*x\^{}2'') atau nama suatu fungsi (misalnya ``f'') menghasilkan grafik fungsi tersebut.

Berikut adalah contoh paling dasar, yang menggunakan rentang default dan menetapkan rentang y yang tepat agar sesuai dengan plot fungsinya.

Catatan: Jika Anda mengakhiri baris perintah dengan titik dua ``:'', plot akan dimasukkan ke dalam jendela teks. Jika tidak, tekan TAB untuk melihat plot jika jendela plot tertutup.

\textgreater plot2d(``x\^{}2''):

\begin{figure}
\centering
\pandocbounded{\includegraphics[keepaspectratio]{images/EMT4Plot2D_Hikmatul Utami_23030630082_Matematika B-004.png}}
\caption{images/EMT4Plot2D\_Hikmatul\%20Utami\_23030630082\_Matematika\%20B-004.png}
\end{figure}

\textgreater aspect(1.5); plot2d(``x\^{}3-x''):

\begin{figure}
\centering
\pandocbounded{\includegraphics[keepaspectratio]{images/EMT4Plot2D_Hikmatul Utami_23030630082_Matematika B-005.png}}
\caption{images/EMT4Plot2D\_Hikmatul\%20Utami\_23030630082\_Matematika\%20B-005.png}
\end{figure}

\textgreater a:=5.6; plot2d(``exp(-a*x\^{}2)/a''); insimg(30); // menampilkan gambar hasil plot setinggi 25 baris

\begin{figure}
\centering
\pandocbounded{\includegraphics[keepaspectratio]{images/EMT4Plot2D_Hikmatul Utami_23030630082_Matematika B-006.png}}
\caption{images/EMT4Plot2D\_Hikmatul\%20Utami\_23030630082\_Matematika\%20B-006.png}
\end{figure}

Dari beberapa contoh sebelumnya Anda dapat melihat bahwa aslinya gambar plot menggunakan sumbu X dengan rentang nilai dari -2 sampai dengan 2. Untuk mengubah rentang nilai X dan Y, Anda dapat menambahkan nilai-nilai batas X (dan Y) di belakang ekspresi yang digambar.

Rentang plot diatur dengan parameter yang ditetapkan sebagai berikut + a,b: rentang x (default -2,2) + c,d: rentang y (default: skala dengan nilai) + r: alternatifnya radius di sekitar pusat plot + cx,cy: koordinat pusat plot (default 0,0)

\textgreater plot2d(``x\^{}3-x'',-1,2):

\begin{figure}
\centering
\pandocbounded{\includegraphics[keepaspectratio]{images/EMT4Plot2D_Hikmatul Utami_23030630082_Matematika B-007.png}}
\caption{images/EMT4Plot2D\_Hikmatul\%20Utami\_23030630082\_Matematika\%20B-007.png}
\end{figure}

\textgreater plot2d(``sin(x)'',-2*pi,2*pi): // plot sin(x) pada interval {[}-2pi, 2pi{]}

\begin{figure}
\centering
\pandocbounded{\includegraphics[keepaspectratio]{images/EMT4Plot2D_Hikmatul Utami_23030630082_Matematika B-008.png}}
\caption{images/EMT4Plot2D\_Hikmatul\%20Utami\_23030630082\_Matematika\%20B-008.png}
\end{figure}

\textgreater plot2d(``cos(x)'',``sin(3*x)'',xmin=0,xmax=2pi):

\begin{figure}
\centering
\pandocbounded{\includegraphics[keepaspectratio]{images/EMT4Plot2D_Hikmatul Utami_23030630082_Matematika B-009.png}}
\caption{images/EMT4Plot2D\_Hikmatul\%20Utami\_23030630082\_Matematika\%20B-009.png}
\end{figure}

Alternatif untuk titik dua adalah perintah insimg(baris), yang menyisipkan plot yang menempati sejumlah baris teks tertentu.

Dalam opsi, plot dapat diatur agar muncul di jendela terpisah yang dapat diubah ukurannya, di jendela buku catatan.

Lebih banyak gaya dapat dicapai dengan perintah plot tertentu.

Bagaimanapun, tekan tombol tabulator untuk melihat plotnya, jika tersembunyi.

Untuk membagi jendela menjadi beberapa plot, gunakan perintah figure(). Dalam contoh, kita memplot x\^{}1 hingga x\^{}4 menjadi 4 bagian jendela. gambar(0) mengatur ulang jendela default.

\textgreater reset;

\textgreater figure(2,2); \ldots{}\\
\textgreater{} for n=1 to 4; figure(n); plot2d(``x\^{}''+n); end; \ldots{}\\
\textgreater{} figure(0):

\begin{figure}
\centering
\pandocbounded{\includegraphics[keepaspectratio]{images/EMT4Plot2D_Hikmatul Utami_23030630082_Matematika B-010.png}}
\caption{images/EMT4Plot2D\_Hikmatul\%20Utami\_23030630082\_Matematika\%20B-010.png}
\end{figure}

Di plot2d(), ada gaya alternatif yang tersedia dengan grid=x. Untuk gambaran umum, kami menampilkan berbagai gaya kisi dalam satu gambar (lihat di bawah untuk perintah figure()). Gaya grid=0 tidak disertakan. Ini tidak menunjukkan kisi dan bingkai.

\textgreater figure(3,3); \ldots{}\\
\textgreater{} for k=1:9; figure(k); plot2d(``x\^{}3-x'',-2,1,grid=k); end; \ldots{}\\
\textgreater{} figure(0):

\begin{figure}
\centering
\pandocbounded{\includegraphics[keepaspectratio]{images/EMT4Plot2D_Hikmatul Utami_23030630082_Matematika B-011.png}}
\caption{images/EMT4Plot2D\_Hikmatul\%20Utami\_23030630082\_Matematika\%20B-011.png}
\end{figure}

Jika argumen pada plot2d() adalah ekspresi yang diikuti oleh empat angka, angka-angka tersebut adalah rentang x dan y untuk plot tersebut.

Alternatifnya, a, b, c, d dapat ditentukan sebagai parameter yang ditetapkan sebagai a=\ldots{} dll.

Pada contoh berikut, kita mengubah gaya kisi, menambahkan label, dan menggunakan label vertikal untuk sumbu y.

\textgreater aspect(1.5); plot2d(``sin(x)'',0,2pi,-1.2,1.2,grid=3,xl=``x'',yl=``sin(x)''):

\begin{figure}
\centering
\pandocbounded{\includegraphics[keepaspectratio]{images/EMT4Plot2D_Hikmatul Utami_23030630082_Matematika B-012.png}}
\caption{images/EMT4Plot2D\_Hikmatul\%20Utami\_23030630082\_Matematika\%20B-012.png}
\end{figure}

\textgreater plot2d(``sin(x)+cos(2*x)'',0,4pi):

\begin{figure}
\centering
\pandocbounded{\includegraphics[keepaspectratio]{images/EMT4Plot2D_Hikmatul Utami_23030630082_Matematika B-013.png}}
\caption{images/EMT4Plot2D\_Hikmatul\%20Utami\_23030630082\_Matematika\%20B-013.png}
\end{figure}

Gambar yang dihasilkan dengan memasukkan plot ke dalam jendela teks disimpan di direktori yang sama dengan buku catatan, secara default di subdirektori bernama ``gambar''. Mereka juga digunakan oleh ekspor HTML.

Anda cukup menandai gambar apa saja dan menyalinnya ke clipboard dengan Ctrl-C. Tentu saja, Anda juga dapat mengekspor grafik saat ini dengan fungsi di menu File.

Fungsi atau ekspresi di plot2d dievaluasi secara adaptif. Agar lebih cepat, nonaktifkan plot adaptif dengan \textless adaptive dan tentukan jumlah subinterval dengan n=\ldots{} Hal ini hanya diperlukan pada kasus yang jarang terjadi.

\textgreater plot2d(``sign(x)*exp(-x\^{}2)'',-1,1,\textless adaptive,n=10000):

\begin{figure}
\centering
\pandocbounded{\includegraphics[keepaspectratio]{images/EMT4Plot2D_Hikmatul Utami_23030630082_Matematika B-014.png}}
\caption{images/EMT4Plot2D\_Hikmatul\%20Utami\_23030630082\_Matematika\%20B-014.png}
\end{figure}

\textgreater plot2d(``x\^{}x'',r=1.2,cx=1,cy=1):

\begin{figure}
\centering
\pandocbounded{\includegraphics[keepaspectratio]{images/EMT4Plot2D_Hikmatul Utami_23030630082_Matematika B-015.png}}
\caption{images/EMT4Plot2D\_Hikmatul\%20Utami\_23030630082\_Matematika\%20B-015.png}
\end{figure}

Perhatikan bahwa x\^{}x tidak ditentukan untuk x\textless=0. Fungsi plot2d menangkap kesalahan ini, dan mulai membuat plot segera setelah fungsinya ditentukan. Ini berfungsi untuk semua fungsi yang mengembalikan NAN di luar jangkauan definisinya.

\textgreater plot2d(``log(x)'',-0.1,2):

\begin{figure}
\centering
\pandocbounded{\includegraphics[keepaspectratio]{images/EMT4Plot2D_Hikmatul Utami_23030630082_Matematika B-016.png}}
\caption{images/EMT4Plot2D\_Hikmatul\%20Utami\_23030630082\_Matematika\%20B-016.png}
\end{figure}

Parameter square=true (atau \textgreater square) memilih rentang y secara otomatis sehingga hasilnya adalah jendela plot persegi. Perhatikan bahwa secara default, Euler menggunakan spasi persegi di dalam jendela plot.

\textgreater plot2d(``x\^{}3-x'',\textgreater square):

\begin{figure}
\centering
\pandocbounded{\includegraphics[keepaspectratio]{images/EMT4Plot2D_Hikmatul Utami_23030630082_Matematika B-017.png}}
\caption{images/EMT4Plot2D\_Hikmatul\%20Utami\_23030630082\_Matematika\%20B-017.png}
\end{figure}

\textgreater plot2d(`'integrate(``sin(x)*exp(-x\^{}2)'',0,x)'\,',0,2): // plot integral

\begin{figure}
\centering
\pandocbounded{\includegraphics[keepaspectratio]{images/EMT4Plot2D_Hikmatul Utami_23030630082_Matematika B-018.png}}
\caption{images/EMT4Plot2D\_Hikmatul\%20Utami\_23030630082\_Matematika\%20B-018.png}
\end{figure}

Jika Anda memerlukan lebih banyak ruang untuk label y, panggil shrinkwindow() dengan parameter lebih kecil, atau tetapkan nilai positif untuk ``lebih kecil'' di plot2d().

\textgreater plot2d(``gamma(x)'',1,10,yl=``y-values'',smaller=6,\textless vertical):

\begin{figure}
\centering
\pandocbounded{\includegraphics[keepaspectratio]{images/EMT4Plot2D_Hikmatul Utami_23030630082_Matematika B-019.png}}
\caption{images/EMT4Plot2D\_Hikmatul\%20Utami\_23030630082\_Matematika\%20B-019.png}
\end{figure}

Ekspresi simbolik juga dapat digunakan karena disimpan sebagai ekspresi string sederhana.

\textgreater x=linspace(0,2pi,1000); plot2d(sin(5x),cos(7x)):

\begin{figure}
\centering
\pandocbounded{\includegraphics[keepaspectratio]{images/EMT4Plot2D_Hikmatul Utami_23030630082_Matematika B-020.png}}
\caption{images/EMT4Plot2D\_Hikmatul\%20Utami\_23030630082\_Matematika\%20B-020.png}
\end{figure}

\textgreater a:=5.6; expr \&= exp(-a*x\^{}2)/a; // define expression

\textgreater plot2d(expr,-2,2): // plot from -2 to 2

\begin{figure}
\centering
\pandocbounded{\includegraphics[keepaspectratio]{images/EMT4Plot2D_Hikmatul Utami_23030630082_Matematika B-021.png}}
\caption{images/EMT4Plot2D\_Hikmatul\%20Utami\_23030630082\_Matematika\%20B-021.png}
\end{figure}

\textgreater plot2d(expr,r=1,thickness=2): // plot in a square around (0,0)

\begin{figure}
\centering
\pandocbounded{\includegraphics[keepaspectratio]{images/EMT4Plot2D_Hikmatul Utami_23030630082_Matematika B-022.png}}
\caption{images/EMT4Plot2D\_Hikmatul\%20Utami\_23030630082\_Matematika\%20B-022.png}
\end{figure}

\textgreater plot2d(\&diff(expr,x),\textgreater add,style=``--'',color=red): // add another plot

\begin{figure}
\centering
\pandocbounded{\includegraphics[keepaspectratio]{images/EMT4Plot2D_Hikmatul Utami_23030630082_Matematika B-023.png}}
\caption{images/EMT4Plot2D\_Hikmatul\%20Utami\_23030630082\_Matematika\%20B-023.png}
\end{figure}

\textgreater plot2d(\&diff(expr,x,2),a=-2,b=2,c=-2,d=1): // plot in rectangle

\begin{figure}
\centering
\pandocbounded{\includegraphics[keepaspectratio]{images/EMT4Plot2D_Hikmatul Utami_23030630082_Matematika B-024.png}}
\caption{images/EMT4Plot2D\_Hikmatul\%20Utami\_23030630082\_Matematika\%20B-024.png}
\end{figure}

\textgreater plot2d(\&diff(expr,x),a=-2,b=2,\textgreater square): // keep plot square

\begin{figure}
\centering
\pandocbounded{\includegraphics[keepaspectratio]{images/EMT4Plot2D_Hikmatul Utami_23030630082_Matematika B-025.png}}
\caption{images/EMT4Plot2D\_Hikmatul\%20Utami\_23030630082\_Matematika\%20B-025.png}
\end{figure}

\textgreater plot2d(``x\^{}2'',0,1,steps=1,color=red,n=10):

\begin{figure}
\centering
\pandocbounded{\includegraphics[keepaspectratio]{images/EMT4Plot2D_Hikmatul Utami_23030630082_Matematika B-026.png}}
\caption{images/EMT4Plot2D\_Hikmatul\%20Utami\_23030630082\_Matematika\%20B-026.png}
\end{figure}

\textgreater plot2d(``x\^{}2'',\textgreater add,steps=2,color=blue,n=10):

\begin{figure}
\centering
\pandocbounded{\includegraphics[keepaspectratio]{images/EMT4Plot2D_Hikmatul Utami_23030630082_Matematika B-027.png}}
\caption{images/EMT4Plot2D\_Hikmatul\%20Utami\_23030630082\_Matematika\%20B-027.png}
\end{figure}

\chapter{Fungsi dalam satu Parameter}\label{fungsi-dalam-satu-parameter}

Fungsi plot yang paling penting untuk plot planar adalah plot2d(). Fungsi ini diimplementasikan dalam bahasa Euler di file ``plot.e'', yang dimuat di awal program.

Berikut beberapa contoh penggunaan suatu fungsi. Seperti biasa di EMT, fungsi yang berfungsi untuk fungsi atau ekspresi lain, Anda bisa meneruskan parameter tambahan (selain x) yang bukan variabel global ke fungsi dengan parameter titik koma atau dengan kumpulan panggilan.

\textgreater function f(x,a) := x\textsuperscript{2/a+a*x}2-x; // define a function

\textgreater a=0.3; plot2d(``f'',0,1;a): // plot with a=0.3

\begin{figure}
\centering
\pandocbounded{\includegraphics[keepaspectratio]{images/EMT4Plot2D_Hikmatul Utami_23030630082_Matematika B-028.png}}
\caption{images/EMT4Plot2D\_Hikmatul\%20Utami\_23030630082\_Matematika\%20B-028.png}
\end{figure}

\textgreater plot2d(``f'',0,1;0.4): // plot with a=0.4

\begin{figure}
\centering
\pandocbounded{\includegraphics[keepaspectratio]{images/EMT4Plot2D_Hikmatul Utami_23030630082_Matematika B-029.png}}
\caption{images/EMT4Plot2D\_Hikmatul\%20Utami\_23030630082\_Matematika\%20B-029.png}
\end{figure}

\textgreater plot2d(\{\{``f'',0.2\}\},0,1): // plot with a=0.2

\begin{figure}
\centering
\pandocbounded{\includegraphics[keepaspectratio]{images/EMT4Plot2D_Hikmatul Utami_23030630082_Matematika B-030.png}}
\caption{images/EMT4Plot2D\_Hikmatul\%20Utami\_23030630082\_Matematika\%20B-030.png}
\end{figure}

\textgreater plot2d(\{\{``f(x,b)'',b=0.1\}\},0,1): // plot with 0.1

\begin{figure}
\centering
\pandocbounded{\includegraphics[keepaspectratio]{images/EMT4Plot2D_Hikmatul Utami_23030630082_Matematika B-031.png}}
\caption{images/EMT4Plot2D\_Hikmatul\%20Utami\_23030630082\_Matematika\%20B-031.png}
\end{figure}

\textgreater function f(x) := x\^{}3-x; \ldots{}\\
\textgreater{} plot2d(``f'',r=1):

\begin{figure}
\centering
\pandocbounded{\includegraphics[keepaspectratio]{images/EMT4Plot2D_Hikmatul Utami_23030630082_Matematika B-032.png}}
\caption{images/EMT4Plot2D\_Hikmatul\%20Utami\_23030630082\_Matematika\%20B-032.png}
\end{figure}

Berikut ini ringkasan fungsi yang diterima + ekspresi atau ekspresi simbolik di x + fungsi atau fungsi simbolik dengan nama ``f'' + fungsi simbolik hanya dengan nama f

Fungsi plot2d() juga menerima fungsi simbolik. Untuk fungsi simbolik, namanya saja yang berfungsi.

\textgreater function f(x) \&= diff(x\^{}x,x)

\begin{verbatim}
                            x
                           x  (log(x) + 1)
\end{verbatim}

\textgreater plot2d(f,0,2):

\begin{figure}
\centering
\pandocbounded{\includegraphics[keepaspectratio]{images/EMT4Plot2D_Hikmatul Utami_23030630082_Matematika B-033.png}}
\caption{images/EMT4Plot2D\_Hikmatul\%20Utami\_23030630082\_Matematika\%20B-033.png}
\end{figure}

Tentu saja, untuk ekspresi atau ekspresi simbolik, nama variabel sudah cukup untuk memplotnya.

\textgreater expr \&= sin(x)*exp(-x)

\begin{verbatim}
                              - x
                             E    sin(x)
\end{verbatim}

\textgreater plot2d(expr,0,3pi):

\begin{figure}
\centering
\pandocbounded{\includegraphics[keepaspectratio]{images/EMT4Plot2D_Hikmatul Utami_23030630082_Matematika B-034.png}}
\caption{images/EMT4Plot2D\_Hikmatul\%20Utami\_23030630082\_Matematika\%20B-034.png}
\end{figure}

\textgreater function f(x) \&= x\^{}x;

\textgreater plot2d(f,r=1,cx=1,cy=1,color=blue,thickness=2);

\textgreater plot2d(\&diff(f(x),x),\textgreater add,color=red,style=``-.-''):

\begin{figure}
\centering
\pandocbounded{\includegraphics[keepaspectratio]{images/EMT4Plot2D_Hikmatul Utami_23030630082_Matematika B-035.png}}
\caption{images/EMT4Plot2D\_Hikmatul\%20Utami\_23030630082\_Matematika\%20B-035.png}
\end{figure}

Untuk gaya garis ada berbagai pilihan. + gaya=``\ldots{}''. Pilih dari ``-'', ``--'', ``-.'', ``.'', ``.-.'', ``-.-''. + Warna: Lihat di bawah untuk warna. + ketebalan: Defaultnya adalah 1.

Warna dapat dipilih sebagai salah satu warna default, atau sebagai warna RGB. + 0..15: indeks warna default. + konstanta warna: putih, hitam, merah, hijau, biru, cyan, zaitun, abu-abu muda, abu-abu, abu-abu tua, oranye, hijau muda, pirus, biru muda, oranye muda, kuning + rgb(merah,hijau,biru): parameternya real di {[}0,1{]}.

\textgreater plot2d(``exp(-x\^{}2)'',r=2,color=red,thickness=3,style=``--''):

\begin{figure}
\centering
\pandocbounded{\includegraphics[keepaspectratio]{images/EMT4Plot2D_Hikmatul Utami_23030630082_Matematika B-036.png}}
\caption{images/EMT4Plot2D\_Hikmatul\%20Utami\_23030630082\_Matematika\%20B-036.png}
\end{figure}

Here is a view of the predefined colors of EMT.

\textgreater aspect(2); columnsplot(ones(1,16),lab=0:15,grid=0,color=0:15):

\begin{figure}
\centering
\pandocbounded{\includegraphics[keepaspectratio]{images/EMT4Plot2D_Hikmatul Utami_23030630082_Matematika B-037.png}}
\caption{images/EMT4Plot2D\_Hikmatul\%20Utami\_23030630082\_Matematika\%20B-037.png}
\end{figure}

tapi kamu bisa mengganti dengan warna lain.

\textgreater columnsplot(ones(1,16),grid=0,color=rgb(0,0,linspace(0,1,15))):

\begin{figure}
\centering
\pandocbounded{\includegraphics[keepaspectratio]{images/EMT4Plot2D_Hikmatul Utami_23030630082_Matematika B-038.png}}
\caption{images/EMT4Plot2D\_Hikmatul\%20Utami\_23030630082\_Matematika\%20B-038.png}
\end{figure}

\chapter{Menggambar Beberapa Kurva pada bidang koordinat yang sama}\label{menggambar-beberapa-kurva-pada-bidang-koordinat-yang-sama}

Plot lebih dari satu fungsi (multiple function) ke dalam satu jendela dapat dilakukan dengan berbagai cara. Salah satu metodenya adalah menggunakan \textgreater add untuk beberapa panggilan ke plot2d secara keseluruhan, kecuali panggilan pertama. Kami telah menggunakan fitur ini pada contoh di atas.

\textgreater aspect(); plot2d(``cos(x)'',r=2,grid=6); plot2d(``x'',style=``.'',\textgreater add):

\begin{figure}
\centering
\pandocbounded{\includegraphics[keepaspectratio]{images/EMT4Plot2D_Hikmatul Utami_23030630082_Matematika B-039.png}}
\caption{images/EMT4Plot2D\_Hikmatul\%20Utami\_23030630082\_Matematika\%20B-039.png}
\end{figure}

\textgreater aspect(1.5); plot2d(``sin(x)'',0,2pi); plot2d(``cos(x)'',color=blue,style=``--'',\textgreater add):

\begin{figure}
\centering
\pandocbounded{\includegraphics[keepaspectratio]{images/EMT4Plot2D_Hikmatul Utami_23030630082_Matematika B-040.png}}
\caption{images/EMT4Plot2D\_Hikmatul\%20Utami\_23030630082\_Matematika\%20B-040.png}
\end{figure}

Salah satu kegunaan \textgreater add adalah untuk menambahkan titik pada kurva.

\textgreater plot2d(``sin(x)'',0,pi); plot2d(2,sin(2),\textgreater points,\textgreater add):

\begin{figure}
\centering
\pandocbounded{\includegraphics[keepaspectratio]{images/EMT4Plot2D_Hikmatul Utami_23030630082_Matematika B-041.png}}
\caption{images/EMT4Plot2D\_Hikmatul\%20Utami\_23030630082\_Matematika\%20B-041.png}
\end{figure}

Kita tambahkan titik perpotongan dengan label (pada posisi ``cl'' untuk kiri tengah), dan masukkan hasilnya ke dalam buku catatan. Kami juga menambahkan judul pada plot.

\textgreater plot2d({[}``cos(x)'',``x''{]},r=1.1,cx=0.5,cy=0.5, \ldots{}\\
\textgreater{} color={[}black,blue{]},style={[}``-'',``.''{]}, \ldots{}\\
\textgreater{} grid=1);

\textgreater x0=solve(``cos(x)-x'',1); \ldots{}\\
\textgreater{} plot2d(x0,x0,\textgreater points,\textgreater add,title=``Intersection Demo''); \ldots{}\\
\textgreater{} label(``cos(x) = x'',x0,x0,pos=``cl'',offset=20):

\begin{figure}
\centering
\pandocbounded{\includegraphics[keepaspectratio]{images/EMT4Plot2D_Hikmatul Utami_23030630082_Matematika B-042.png}}
\caption{images/EMT4Plot2D\_Hikmatul\%20Utami\_23030630082\_Matematika\%20B-042.png}
\end{figure}

Dalam demo berikut, kita memplot fungsi sin(x)=sin(x)/x dan ekspansi Taylor ke-8 dan ke-16. Kami menghitung perluasan ini menggunakan Maxima melalui ekspresi simbolik.

Plot ini dilakukan dalam perintah multi-baris berikut dengan tiga panggilan ke plot2d(). Yang kedua dan ketiga memiliki kumpulan tanda \textgreater add, yang membuat plot menggunakan rentang sebelumnya.

Kami menambahkan kotak label yang menjelaskan fungsinya.

\textgreater\$taylor(sin(x)/x,x,0,4)

\[\frac{x^4}{120}-\frac{x^2}{6}+1\]\textgreater plot2d(``sinc(x)'',0,4pi,color=green,thickness=2); \ldots{}\\
\textgreater{} plot2d(\&taylor(sin(x)/x,x,0,8),\textgreater add,color=blue,style=``--''); \ldots{}\\
\textgreater{} plot2d(\&taylor(sin(x)/x,x,0,16),\textgreater add,color=red,style=``-.-''); \ldots{}\\
\textgreater{} labelbox({[}``sinc'',``T8'',``T16''{]},styles={[}``-'',``--'',``-.-''{]}, \ldots{}\\
\textgreater{} colors={[}black,blue,red{]}):

\begin{figure}
\centering
\pandocbounded{\includegraphics[keepaspectratio]{images/EMT4Plot2D_Hikmatul Utami_23030630082_Matematika B-044.png}}
\caption{images/EMT4Plot2D\_Hikmatul\%20Utami\_23030630082\_Matematika\%20B-044.png}
\end{figure}

Dalam contoh berikut, kami menghasilkan Polinomial Bernstein.

\[B_i(x) = \binom{n}{i} x^i (1-x)^{n-i}\]\textgreater plot2d(``(1-x)\^{}10'',0,1); // plot first function

\textgreater for i=1 to 10; plot2d(``bin(10,i)*x\textsuperscript{i*(1-x)}(10-i)'',\textgreater add); end;

\textgreater insimg;

\begin{figure}
\centering
\pandocbounded{\includegraphics[keepaspectratio]{images/EMT4Plot2D_Hikmatul Utami_23030630082_Matematika B-046.png}}
\caption{images/EMT4Plot2D\_Hikmatul\%20Utami\_23030630082\_Matematika\%20B-046.png}
\end{figure}

Cara kedua adalah dengan menggunakan pasangan matriks bernilai x dan matriks bernilai y yang berukuran sama.

Kami menghasilkan matriks nilai dengan satu Polinomial Bernstein di setiap baris. Untuk ini, kita cukup menggunakan vektor kolom i. Lihat pendahuluan tentang bahasa matriks untuk mempelajari lebih detail.

\textgreater x=linspace(0,1,500);

\textgreater n=10; k=(0:n)'; // n is row vector, k is column vector

\textgreater y=bin(n,k)*x\textsuperscript{k*(1-x)}(n-k); // y is a matrix then

\textgreater plot2d(x,y):

\begin{figure}
\centering
\pandocbounded{\includegraphics[keepaspectratio]{images/EMT4Plot2D_Hikmatul Utami_23030630082_Matematika B-047.png}}
\caption{images/EMT4Plot2D\_Hikmatul\%20Utami\_23030630082\_Matematika\%20B-047.png}
\end{figure}

Perhatikan bahwa parameter warna dapat berupa vektor. Kemudian setiap warna digunakan untuk setiap baris matriks.

\textgreater x=linspace(0,1,200); y=x\^{}(1:10)'; plot2d(x,y,color=1:10):

\begin{figure}
\centering
\pandocbounded{\includegraphics[keepaspectratio]{images/EMT4Plot2D_Hikmatul Utami_23030630082_Matematika B-048.png}}
\caption{images/EMT4Plot2D\_Hikmatul\%20Utami\_23030630082\_Matematika\%20B-048.png}
\end{figure}

Metode lain adalah menggunakan vektor ekspresi (string). Anda kemudian dapat menggunakan susunan warna, susunan gaya, dan susunan ketebalan dengan panjang yang sama.

\textgreater plot2d({[}``sin(x)'',``cos(x)''{]},0,2pi,color=4:5):

\begin{figure}
\centering
\pandocbounded{\includegraphics[keepaspectratio]{images/EMT4Plot2D_Hikmatul Utami_23030630082_Matematika B-049.png}}
\caption{images/EMT4Plot2D\_Hikmatul\%20Utami\_23030630082\_Matematika\%20B-049.png}
\end{figure}

\textgreater plot2d({[}``sin(x)'',``cos(x)''{]},0,2pi): // plot vector of expressions

\begin{figure}
\centering
\pandocbounded{\includegraphics[keepaspectratio]{images/EMT4Plot2D_Hikmatul Utami_23030630082_Matematika B-050.png}}
\caption{images/EMT4Plot2D\_Hikmatul\%20Utami\_23030630082\_Matematika\%20B-050.png}
\end{figure}

Kita bisa mendapatkan vektor seperti itu dari Maxima menggunakan makelist() dan mxm2str().

\textgreater v \&= makelist(binomial(10,i)*x\textsuperscript{i*(1-x)}(10-i),i,0,10) // make list

\begin{verbatim}
               10            9              8  2             7  3
       [(1 - x)  , 10 (1 - x)  x, 45 (1 - x)  x , 120 (1 - x)  x , 
           6  4             5  5             4  6             3  7
210 (1 - x)  x , 252 (1 - x)  x , 210 (1 - x)  x , 120 (1 - x)  x , 
          2  8              9   10
45 (1 - x)  x , 10 (1 - x) x , x  ]
\end{verbatim}

\textgreater mxm2str(v) // get a vector of strings from the symbolic vector

\begin{verbatim}
(1-x)^10
10*(1-x)^9*x
45*(1-x)^8*x^2
120*(1-x)^7*x^3
210*(1-x)^6*x^4
252*(1-x)^5*x^5
210*(1-x)^4*x^6
120*(1-x)^3*x^7
45*(1-x)^2*x^8
10*(1-x)*x^9
x^10
\end{verbatim}

\textgreater plot2d(mxm2str(v),0,1): // plot functions

\begin{figure}
\centering
\pandocbounded{\includegraphics[keepaspectratio]{images/EMT4Plot2D_Hikmatul Utami_23030630082_Matematika B-051.png}}
\caption{images/EMT4Plot2D\_Hikmatul\%20Utami\_23030630082\_Matematika\%20B-051.png}
\end{figure}

Alternatif lain adalah dengan menggunakan bahasa matriks Euler.

Jika suatu ekspresi menghasilkan matriks fungsi, dengan satu fungsi di setiap baris, semua fungsi tersebut akan diplot ke dalam satu plot.

Untuk ini, gunakan vektor parameter dalam bentuk vektor kolom. Jika array warna ditambahkan maka akan digunakan untuk setiap baris plot.

\textgreater n=(1:10)'; plot2d(``x\^{}n'',0,1,color=1:10):

\begin{figure}
\centering
\pandocbounded{\includegraphics[keepaspectratio]{images/EMT4Plot2D_Hikmatul Utami_23030630082_Matematika B-052.png}}
\caption{images/EMT4Plot2D\_Hikmatul\%20Utami\_23030630082\_Matematika\%20B-052.png}
\end{figure}

Ekspresi dan fungsi satu baris dapat melihat variabel global.

Jika Anda tidak dapat menggunakan variabel global, Anda perlu menggunakan fungsi dengan parameter tambahan, dan meneruskan parameter ini sebagai parameter titik koma.

Berhati-hatilah, untuk meletakkan semua parameter yang ditetapkan di akhir perintah plot2d. Dalam contoh ini kita meneruskan a=5 ke fungsi f, yang kita plot dari -10 hingga 10.

\textgreater function f(x,a) := 1/a*exp(-x\^{}2/a); \ldots{}\\
\textgreater{} plot2d(``f'',-10,10;5,thickness=2,title=``a=5''):

\begin{figure}
\centering
\pandocbounded{\includegraphics[keepaspectratio]{images/EMT4Plot2D_Hikmatul Utami_23030630082_Matematika B-053.png}}
\caption{images/EMT4Plot2D\_Hikmatul\%20Utami\_23030630082\_Matematika\%20B-053.png}
\end{figure}

Alternatifnya, gunakan koleksi dengan nama fungsi dan semua parameter tambahan. Daftar khusus ini disebut kumpulan panggilan, dan ini adalah cara yang lebih disukai untuk meneruskan argumen ke suatu fungsi yang kemudian diteruskan sebagai argumen ke fungsi lain.

Pada contoh berikut, kita menggunakan loop untuk memplot beberapa fungsi (lihat tutorial tentang pemrograman loop).

\textgreater plot2d(\{\{``f'',1\}\},-10,10); \ldots{}\\
\textgreater{} for a=2:10; plot2d(\{\{``f'',a\}\},\textgreater add); end:

\begin{figure}
\centering
\pandocbounded{\includegraphics[keepaspectratio]{images/EMT4Plot2D_Hikmatul Utami_23030630082_Matematika B-054.png}}
\caption{images/EMT4Plot2D\_Hikmatul\%20Utami\_23030630082\_Matematika\%20B-054.png}
\end{figure}

Kita dapat mencapai hasil yang sama dengan cara berikut menggunakan bahasa matriks EMT. Setiap baris matriks f(x,a) merupakan satu fungsi. Selain itu, kita dapat mengatur warna untuk setiap baris matriks. Klik dua kali pada fungsi getspectral() untuk penjelasannya.

\textgreater x=-10:0.01:10; a=(1:10)'; plot2d(x,f(x,a),color=getspectral(a/10)):

\begin{figure}
\centering
\pandocbounded{\includegraphics[keepaspectratio]{images/EMT4Plot2D_Hikmatul Utami_23030630082_Matematika B-055.png}}
\caption{images/EMT4Plot2D\_Hikmatul\%20Utami\_23030630082\_Matematika\%20B-055.png}
\end{figure}

\section{Label Teks}\label{label-teks}

Dekorasi sederhana pun bisa + judul dengan judul = ``\ldots{}'' + label x dan y dengan xl=``\ldots{}'', yl=``\ldots{}'' + label teks lain dengan label(``\ldots{}'',x,y)

Perintah label akan memplot ke plot saat ini pada koordinat plot (x,y). Hal ini memerlukan argumen posisional.

\textgreater plot2d(``x\textsuperscript{3-x'',-1,2,title=''y=x}3-x'',yl=``y'',xl=``x''):

\begin{figure}
\centering
\pandocbounded{\includegraphics[keepaspectratio]{images/EMT4Plot2D_Hikmatul Utami_23030630082_Matematika B-056.png}}
\caption{images/EMT4Plot2D\_Hikmatul\%20Utami\_23030630082\_Matematika\%20B-056.png}
\end{figure}

\textgreater expr := ``log(x)/x''; \ldots{}\\
\textgreater{} plot2d(expr,0.5,5,title=``y=''+expr,xl=``x'',yl=``y''); \ldots{}\\
\textgreater{} label(``(1,0)'',1,0); label(``Max'',E,expr(E),pos=``lc''):

\begin{figure}
\centering
\pandocbounded{\includegraphics[keepaspectratio]{images/EMT4Plot2D_Hikmatul Utami_23030630082_Matematika B-057.png}}
\caption{images/EMT4Plot2D\_Hikmatul\%20Utami\_23030630082\_Matematika\%20B-057.png}
\end{figure}

Ada juga fungsi labelbox(), yang dapat menampilkan fungsi dan teks. Dibutuhkan vektor string dan warna, satu item untuk setiap fungsi.

\textgreater function f(x) \&= x\textsuperscript{2*exp(-x}2); \ldots{}\\
\textgreater{} plot2d(\&f(x),a=-3,b=3,c=-1,d=1); \ldots{}\\
\textgreater{} plot2d(\&diff(f(x),x),\textgreater add,color=blue,style=``--''); \ldots{}\\
\textgreater{} labelbox({[}``function'',``derivative''{]},styles={[}``-'',``--''{]}, \ldots{}\\
\textgreater{} colors={[}black,blue{]},w=0.4):

\begin{figure}
\centering
\pandocbounded{\includegraphics[keepaspectratio]{images/EMT4Plot2D_Hikmatul Utami_23030630082_Matematika B-058.png}}
\caption{images/EMT4Plot2D\_Hikmatul\%20Utami\_23030630082\_Matematika\%20B-058.png}
\end{figure}

Kotak ini berlabuh di kanan atas secara default, tetapi \textgreater kiri berlabuh di kiri atas. Anda dapat memindahkannya ke tempat mana pun yang Anda suka. Posisi jangkar berada di pojok kanan atas kotak, dan angkanya merupakan pecahan dari ukuran jendela grafis. Lebarnya otomatis.

Untuk plot titik, kotak label juga berfungsi. Tambahkan parameter \textgreater points, atau vektor bendera, satu untuk setiap label.

Pada contoh berikut, hanya ada satu fungsi. Jadi kita bisa menggunakan string sebagai pengganti vektor string. Kami mengatur warna teks menjadi hitam untuk contoh ini.

\textgreater n=10; plot2d(0:n,bin(n,0:n),\textgreater addpoints); \ldots{}\\
\textgreater{} labelbox(``Binomials'',styles=``{[}{]}'',\textgreater points,x=0.1,y=0.1, \ldots{}\\
\textgreater{} tcolor=black,\textgreater left):

\begin{figure}
\centering
\pandocbounded{\includegraphics[keepaspectratio]{images/EMT4Plot2D_Hikmatul Utami_23030630082_Matematika B-059.png}}
\caption{images/EMT4Plot2D\_Hikmatul\%20Utami\_23030630082\_Matematika\%20B-059.png}
\end{figure}

Gaya plot ini juga tersedia di statplot(). Seperti di plot2d() warna dapat diatur untuk setiap baris plot. Masih banyak lagi plot khusus untuk keperluan statistik (lihat tutorial tentang statistik).

\textgreater statplot(1:10,random(2,10),color={[}red,blue{]}):

\begin{figure}
\centering
\pandocbounded{\includegraphics[keepaspectratio]{images/EMT4Plot2D_Hikmatul Utami_23030630082_Matematika B-060.png}}
\caption{images/EMT4Plot2D\_Hikmatul\%20Utami\_23030630082\_Matematika\%20B-060.png}
\end{figure}

Fitur serupa adalah fungsi textbox().

Lebarnya secara default adalah lebar maksimal baris teks. Tapi itu bisa diatur oleh pengguna juga.

\textgreater function f(x) \&= exp(-x)*sin(2*pi*x); \ldots{}\\
\textgreater{} plot2d(``f(x)'',0,2pi); \ldots{}\\
\textgreater{} textbox(latex(``\textbackslash text\{Example of a damped oscillation\}\textbackslash{} f(x)=e\^{}\{-x\}sin(2\textbackslash pi x)''),w=0.85):

\begin{figure}
\centering
\pandocbounded{\includegraphics[keepaspectratio]{images/EMT4Plot2D_Hikmatul Utami_23030630082_Matematika B-061.png}}
\caption{images/EMT4Plot2D\_Hikmatul\%20Utami\_23030630082\_Matematika\%20B-061.png}
\end{figure}

Text labels, titles, label boxes and other text can contain Unicode strings (see the syntax of EMT for more about Unicode strings).

\textgreater plot2d(``x\^{}3-x'',title=u''x → x³ - x''):

\begin{figure}
\centering
\pandocbounded{\includegraphics[keepaspectratio]{images/EMT4Plot2D_Hikmatul Utami_23030630082_Matematika B-062.png}}
\caption{images/EMT4Plot2D\_Hikmatul\%20Utami\_23030630082\_Matematika\%20B-062.png}
\end{figure}

The labels on the x- and y-axis can be vertical, as well as the axis.

\textgreater plot2d(``sinc(x)'',0,2pi,xl=``x'',yl=u''x → sinc(x)``,\textgreater vertical):

\begin{figure}
\centering
\pandocbounded{\includegraphics[keepaspectratio]{images/EMT4Plot2D_Hikmatul Utami_23030630082_Matematika B-063.png}}
\caption{images/EMT4Plot2D\_Hikmatul\%20Utami\_23030630082\_Matematika\%20B-063.png}
\end{figure}

\section{LaTeX}\label{latex}

Anda juga dapat memplot rumus LaTeX jika Anda telah menginstal sistem LaTeX. Saya merekomendasikan MiKTeX. Jalur ke biner ``lateks'' dan ``dvipng'' harus berada di jalur sistem, atau Anda harus mengatur LaTeX di menu opsi.

Perhatikan, penguraian LaTeX lambat. Jika Anda ingin menggunakan LaTeX dalam plot animasi, Anda harus memanggil latex() sebelum loop satu kali dan menggunakan hasilnya (gambar dalam matriks RGB).

Pada plot berikut, kami menggunakan LaTeX untuk label x dan y, label, kotak label, dan judul plot.

\textgreater plot2d(``exp(-x)*sin(x)/x'',a=0,b=2pi,c=0,d=1,grid=6,color=blue, \ldots{}\\
\textgreater{} title=latex(``\textbackslash text\{Function \(\\Phi\)\}''), \ldots{}\\
\textgreater{} xl=latex(``\textbackslash phi''),yl=latex(``\textbackslash Phi(\textbackslash phi)'')); \ldots{}\\
\textgreater{} textbox( \ldots{}\\
\textgreater{} latex(``\textbackslash Phi(\textbackslash phi) = e\^{}\{-\textbackslash phi\} \textbackslash frac\{\textbackslash sin(\textbackslash phi)\}\{\textbackslash phi\}''),x=0.8,y=0.5); \ldots{}\\
\textgreater{} label(latex(``\textbackslash Phi'',color=blue),1,0.4):

\begin{figure}
\centering
\pandocbounded{\includegraphics[keepaspectratio]{images/EMT4Plot2D_Hikmatul Utami_23030630082_Matematika B-064.png}}
\caption{images/EMT4Plot2D\_Hikmatul\%20Utami\_23030630082\_Matematika\%20B-064.png}
\end{figure}

Seringkali, kita menginginkan spasi dan label teks yang tidak konformal pada sumbu x. Kita bisa menggunakan xaxis() dan yaxis() seperti yang akan kita tunjukkan nanti.

Cara termudah adalah membuat plot kosong dengan bingkai menggunakan grid=4, lalu menambahkan grid dengan ygrid() dan xgrid(). Pada contoh berikut, kami menggunakan tiga string LaTeX untuk label pada sumbu x dengan xtick().

\textgreater plot2d(``sinc(x)'',0,2pi,grid=4,\textless ticks); \ldots{}\\
\textgreater{} ygrid(-2:0.5:2,grid=6); \ldots{}\\
\textgreater{} xgrid({[}0:2{]}*pi,\textless ticks,grid=6); \ldots{}\\
\textgreater{} xtick({[}0,pi,2pi{]},{[}``0'',``\textbackslash pi'',``2\textbackslash pi''{]},\textgreater latex):

\begin{figure}
\centering
\pandocbounded{\includegraphics[keepaspectratio]{images/EMT4Plot2D_Hikmatul Utami_23030630082_Matematika B-065.png}}
\caption{images/EMT4Plot2D\_Hikmatul\%20Utami\_23030630082\_Matematika\%20B-065.png}
\end{figure}

Tentu saja fungsinya juga bisa digunakan.

\textgreater function map f(x) \ldots{}

\begin{verbatim}
if x>0 then return x^4
else return x^2
endif
endfunction
\end{verbatim}

Parameter ``peta'' membantu menggunakan fungsi untuk vektor. Untuk plot, itu tidak diperlukan. Namun untuk menunjukkan bahwa vektorisasi berguna, kami menambahkan beberapa poin penting ke plot pada x=-1, x=0, dan x=1.

Pada plot berikut, kami juga memasukkan beberapa kode LaTeX. Kami menggunakannya untuk dua label dan kotak teks. Tentu saja, Anda hanya dapat menggunakan LaTeX jika Anda telah menginstal LaTeX dengan benar.

\textgreater plot2d(``f'',-1,1,xl=``x'',yl=``f(x)'',grid=6); \ldots{}\\
\textgreater{} plot2d({[}-1,0,1{]},f({[}-1,0,1{]}),\textgreater points,\textgreater add); \ldots{}\\
\textgreater{} label(latex(``x\^{}3''),0.72,f(0.72)); \ldots{}\\
\textgreater{} label(latex(``x\^{}2''),-0.52,f(-0.52),pos=``ll''); \ldots{}\\
\textgreater{} textbox( \ldots{}\\
\textgreater{} latex(``f(x)=\textbackslash begin\{cases\} x\^{}3 \& x\textgreater0 \textbackslash\textbackslash{} x\^{}2 \& x \textbackslash le 0\textbackslash end\{cases\}''), \ldots{}\\
\textgreater{} x=0.7,y=0.2):

\begin{figure}
\centering
\pandocbounded{\includegraphics[keepaspectratio]{images/EMT4Plot2D_Hikmatul Utami_23030630082_Matematika B-066.png}}
\caption{images/EMT4Plot2D\_Hikmatul\%20Utami\_23030630082\_Matematika\%20B-066.png}
\end{figure}

\section{Interaksi Pengguna}\label{interaksi-pengguna}

Saat memplot suatu fungsi atau ekspresi, parameter \textgreater pengguna memungkinkan pengguna untuk memperbesar dan menggeser plot dengan tombol kursor atau mouse. Pengguna bisa + perbesar dengan + atau - + pindahkan plot dengan tombol kursor + pilih jendela plot dengan mouse + atur ulang tampilan dengan spasi + keluar dengan kembali

Tombol spasi akan mengatur ulang plot ke jendela plot aslinya.

Saat memplot data, flag \textgreater user hanya akan menunggu penekanan tombol.

\textgreater plot2d(\{\{``x\^{}3-a*x'',a=1\}\},\textgreater user,title=``Press any key!''):

\begin{figure}
\centering
\pandocbounded{\includegraphics[keepaspectratio]{images/EMT4Plot2D_Hikmatul Utami_23030630082_Matematika B-067.png}}
\caption{images/EMT4Plot2D\_Hikmatul\%20Utami\_23030630082\_Matematika\%20B-067.png}
\end{figure}

\textgreater plot2d(``exp(x)*sin(x)'',user=true, \ldots{}\\
\textgreater{} title=``+/- or cursor keys (return to exit)''):

\begin{figure}
\centering
\pandocbounded{\includegraphics[keepaspectratio]{images/EMT4Plot2D_Hikmatul Utami_23030630082_Matematika B-068.png}}
\caption{images/EMT4Plot2D\_Hikmatul\%20Utami\_23030630082\_Matematika\%20B-068.png}
\end{figure}

Berikut ini menunjukkan cara interaksi pengguna tingkat lanjut (lihat tutorial tentang pemrograman untuk detailnya).

Fungsi bawaan mousedrag() menunggu aktivitas mouse atau keyboard. Ini melaporkan mouse ke bawah, gerakan mouse atau mouse ke atas, dan penekanan tombol. Fungsi dragpoints() memanfaatkan ini, dan memungkinkan pengguna menyeret titik mana pun dalam plot.

Kita membutuhkan fungsi plot terlebih dahulu. Misalnya, kita melakukan interpolasi pada 5 titik dengan polinomial. Fungsi tersebut harus diplot ke dalam area plot yang tetap.

\textgreater function plotf(xp,yp,select) \ldots{}

\begin{verbatim}
  d=interp(xp,yp);
  plot2d("interpval(xp,d,x)";d,xp,r=2);
  plot2d(xp,yp,>points,>add);
  if select>0 then
    plot2d(xp[select],yp[select],color=red,>points,>add);
  endif;
  title("Drag one point, or press space or return!");
endfunction
\end{verbatim}

Perhatikan parameter titik koma di plot2d (d dan xp), yang diteruskan ke evaluasi fungsi interp(). Tanpa ini, kita harus menulis fungsi plotinterp() terlebih dahulu, mengakses nilainya secara global.

Sekarang kita menghasilkan beberapa nilai acak, dan membiarkan pengguna menyeret titiknya.

\textgreater t=-1:0.5:1; dragpoints(``plotf'',t,random(size(t))-0.5):

\begin{figure}
\centering
\pandocbounded{\includegraphics[keepaspectratio]{images/EMT4Plot2D_Hikmatul Utami_23030630082_Matematika B-069.png}}
\caption{images/EMT4Plot2D\_Hikmatul\%20Utami\_23030630082\_Matematika\%20B-069.png}
\end{figure}

Ada juga fungsi yang memplot fungsi lain bergantung pada vektor parameter, dan memungkinkan pengguna menyesuaikan parameter ini.

Pertama kita membutuhkan fungsi plot.

\textgreater function plotf({[}a,b{]}) := plot2d(``exp(a*x)*cos(2pi*b*x)'',0,2pi;a,b);

Kemudian kita memerlukan nama untuk parameter, nilai awal dan matriks rentang nx2, opsional garis judul.

Ada penggeser interaktif, yang dapat menetapkan nilai oleh pengguna. Fungsi dragvalues() menyediakan ini.

\textgreater dragvalues(``plotf'',{[}``a'',``b''{]},{[}-1,2{]},{[}{[}-2,2{]};{[}1,10{]}{]}, \ldots{}\\
\textgreater{} heading=``Drag these values:'',hcolor=black):

\begin{figure}
\centering
\pandocbounded{\includegraphics[keepaspectratio]{images/EMT4Plot2D_Hikmatul Utami_23030630082_Matematika B-070.png}}
\caption{images/EMT4Plot2D\_Hikmatul\%20Utami\_23030630082\_Matematika\%20B-070.png}
\end{figure}

Dimungkinkan untuk membatasi nilai yang diseret menjadi bilangan bulat. Sebagai contoh, kita menulis fungsi plot, yang memplot polinomial Taylor berderajat n ke fungsi kosinus.

\textgreater function plotf(n) \ldots{}

\begin{verbatim}
plot2d("cos(x)",0,2pi,>square,grid=6);
plot2d(&"taylor(cos(x),x,0,@n)",color=blue,>add);
textbox("Taylor polynomial of degree "+n,0.1,0.02,style="t",>left);
endfunction
\end{verbatim}

Sekarang kita izinkan derajat n bervariasi dari 0 hingga 20 dalam 20 perhentian. Hasil dragvalues() digunakan untuk memplot sketsa dengan n ini, dan untuk memasukkan plot ke dalam buku catatan.

\textgreater nd=dragvalues(``plotf'',``degree'',2,{[}0,20{]},20,y=0.8, \ldots{}\\
\textgreater{} heading=``Drag the value:''); \ldots{}\\
\textgreater{} plotf(nd):

\begin{figure}
\centering
\pandocbounded{\includegraphics[keepaspectratio]{images/EMT4Plot2D_Hikmatul Utami_23030630082_Matematika B-071.png}}
\caption{images/EMT4Plot2D\_Hikmatul\%20Utami\_23030630082\_Matematika\%20B-071.png}
\end{figure}

Berikut ini adalah demonstrasi sederhana dari fungsinya. Pengguna dapat menggambar jendela plot, meninggalkan jejak titik.

\textgreater function dragtest \ldots{}

\begin{verbatim}
  plot2d(none,r=1,title="Drag with the mouse, or press any key!");
  start=0;
  repeat
    {flag,m,time}=mousedrag();
    if flag==0 then return; endif;
    if flag==2 then
      hold on; mark(m[1],m[2]); hold off;
    endif;
  end
endfunction
\end{verbatim}

\textgreater dragtest // lihat hasilnya dan cobalah lakukan!

\section{Gaya~Plot~2D}\label{gaya-plot-2d}

Secara default, EMT menghitung tick sumbu otomatis dan menambahkan label ke setiap tick. Ini dapat diubah dengan parameter grid. Gaya default sumbu dan label dapat diubah. Selain itu, label dan judul dapat ditambahkan secara manual. Untuk menyetel ulang ke gaya default, gunakan reset().

\textgreater aspect();

\textgreater figure(3,4); \ldots{}\\
\textgreater{} figure(1); plot2d(``x\^{}3-x'',grid=0); \ldots{} // no grid, frame or axis

\textgreater{} figure(2); plot2d(``x\^{}3-x'',grid=1); \ldots{} // x-y-axis

\textgreater{} figure(3); plot2d(``x\^{}3-x'',grid=2); \ldots{} // default ticks

\textgreater{} figure(4); plot2d(``x\^{}3-x'',grid=3); \ldots{} // x-y- axis with labels inside

\textgreater{} figure(5); plot2d(``x\^{}3-x'',grid=4); \ldots{} // no ticks, only labels

\textgreater{} figure(6); plot2d(``x\^{}3-x'',grid=5); \ldots{} // default, but no margin

\textgreater{} figure(7); plot2d(``x\^{}3-x'',grid=6); \ldots{} // axes only

\textgreater{} figure(8); plot2d(``x\^{}3-x'',grid=7); \ldots{} // axes only, ticks at axis

\textgreater{} figure(9); plot2d(``x\^{}3-x'',grid=8); \ldots{} // axes only, finer ticks at axis

\textgreater{} figure(10); plot2d(``x\^{}3-x'',grid=9); \ldots{} // default, small ticks inside

\textgreater{} figure(11); plot2d(``x\^{}3-x'',grid=10); \ldots// no ticks, axes only

\textgreater{} figure(0):

\begin{figure}
\centering
\pandocbounded{\includegraphics[keepaspectratio]{images/EMT4Plot2D_Hikmatul Utami_23030630082_Matematika B-072.png}}
\caption{images/EMT4Plot2D\_Hikmatul\%20Utami\_23030630082\_Matematika\%20B-072.png}
\end{figure}

Parameter \textless frame mematikan frame, dan framecolor=blue mengatur frame menjadi warna biru.

Jika Anda menginginkan tanda centang Anda sendiri, Anda dapat menggunakan style=0, dan menambahkan semuanya nanti.

\textgreater aspect(1.5);

\textgreater plot2d(``x\^{}3-x'',grid=0); // plot

\textgreater frame; xgrid({[}-1,0,1{]}); ygrid(0): // add frame and grid

\begin{figure}
\centering
\pandocbounded{\includegraphics[keepaspectratio]{images/EMT4Plot2D_Hikmatul Utami_23030630082_Matematika B-073.png}}
\caption{images/EMT4Plot2D\_Hikmatul\%20Utami\_23030630082\_Matematika\%20B-073.png}
\end{figure}

Untuk judul plot dan label sumbu, lihat contoh berikut.

\textgreater plot2d(``exp(x)'',-1,1);

\textgreater textcolor(black); // set the text color to black

\textgreater title(latex(``y=e\^{}x'')); // title above the plot

\textgreater xlabel(latex(``x'')); // ``x'' for x-axis

\textgreater ylabel(latex(``y''),\textgreater vertical); // vertical ``y'' for y-axis

\textgreater label(latex(``(0,1)''),0,1,color=blue): // label a point

\begin{figure}
\centering
\pandocbounded{\includegraphics[keepaspectratio]{images/EMT4Plot2D_Hikmatul Utami_23030630082_Matematika B-074.png}}
\caption{images/EMT4Plot2D\_Hikmatul\%20Utami\_23030630082\_Matematika\%20B-074.png}
\end{figure}

Sumbu dapat digambar secara terpisah dengan xaxis() dan yaxis().

\textgreater plot2d(``x\^{}3-x'',\textless grid,\textless frame);

\textgreater xaxis(0,xx=-2:1,style=``-\textgreater{}''); yaxis(0,yy=-5:5,style=``-\textgreater{}''):

\begin{figure}
\centering
\pandocbounded{\includegraphics[keepaspectratio]{images/EMT4Plot2D_Hikmatul Utami_23030630082_Matematika B-075.png}}
\caption{images/EMT4Plot2D\_Hikmatul\%20Utami\_23030630082\_Matematika\%20B-075.png}
\end{figure}

Teks pada plot dapat diatur dengan label(). Dalam contoh berikut, ``lc'' berarti bagian tengah bawah. Ini menetapkan posisi label relatif terhadap koordinat plot.

\textgreater function f(x) \&= x\^{}3-x

\begin{verbatim}
                                 3
                                x  - x
\end{verbatim}

\textgreater plot2d(f,-1,1,\textgreater square);

\textgreater x0=fmin(f,0,1); // compute point of minimum

\textgreater label(``Rel. Min.'',x0,f(x0),pos=``lc''): // add a label there

\begin{figure}
\centering
\pandocbounded{\includegraphics[keepaspectratio]{images/EMT4Plot2D_Hikmatul Utami_23030630082_Matematika B-076.png}}
\caption{images/EMT4Plot2D\_Hikmatul\%20Utami\_23030630082\_Matematika\%20B-076.png}
\end{figure}

Ada juga kotak teks.

\textgreater plot2d(\&f(x),-1,1,-2,2); // function

\textgreater plot2d(\&diff(f(x),x),\textgreater add,style=``--'',color=red); // derivative

\textgreater labelbox({[}``f'',``f'''{]},{[}``-'',``--''{]},{[}black,red{]}): // label box

\begin{figure}
\centering
\pandocbounded{\includegraphics[keepaspectratio]{images/EMT4Plot2D_Hikmatul Utami_23030630082_Matematika B-077.png}}
\caption{images/EMT4Plot2D\_Hikmatul\%20Utami\_23030630082\_Matematika\%20B-077.png}
\end{figure}

\textgreater plot2d({[}``exp(x)'',``1+x''{]},color={[}black,blue{]},style={[}``-'',``-.-''{]}):

\begin{figure}
\centering
\pandocbounded{\includegraphics[keepaspectratio]{images/EMT4Plot2D_Hikmatul Utami_23030630082_Matematika B-078.png}}
\caption{images/EMT4Plot2D\_Hikmatul\%20Utami\_23030630082\_Matematika\%20B-078.png}
\end{figure}

\textgreater gridstyle(``-\textgreater{}'',color=gray,textcolor=gray,framecolor=gray); \ldots{}\\
\textgreater{} plot2d(``x\^{}3-x'',grid=1); \ldots{}\\
\textgreater{} settitle(``y=x\^{}3-x'',color=black); \ldots{}\\
\textgreater{} label(``x'',2,0,pos=``bc'',color=gray); \ldots{}\\
\textgreater{} label(``y'',0,6,pos=``cl'',color=gray); \ldots{}\\
\textgreater{} reset():

\begin{figure}
\centering
\pandocbounded{\includegraphics[keepaspectratio]{images/EMT4Plot2D_Hikmatul Utami_23030630082_Matematika B-079.png}}
\caption{images/EMT4Plot2D\_Hikmatul\%20Utami\_23030630082\_Matematika\%20B-079.png}
\end{figure}

Untuk kontrol lebih lanjut, sumbu x dan sumbu y dapat dilakukan secara manual.

Perintah fullwindow() memperluas jendela plot karena kita tidak lagi memerlukan tempat untuk label di luar jendela plot. Gunakan shrinkwindow() atau reset() untuk menyetel ulang ke default.

\textgreater fullwindow; \ldots{}\\
\textgreater{} gridstyle(color=darkgray,textcolor=darkgray); \ldots{}\\
\textgreater{} plot2d({[}``2\textsuperscript{x'',''1'',''2}(-x)''{]},a=-2,b=2,c=0,d=4,\textless grid,color=4:6,\textless frame); \ldots{}\\
\textgreater{} xaxis(0,-2:1,style=``-\textgreater{}''); xaxis(0,2,``x'',\textless axis); \ldots{}\\
\textgreater{} yaxis(0,4,``y'',style=``-\textgreater{}''); \ldots{}\\
\textgreater{} yaxis(-2,1:4,\textgreater left); \ldots{}\\
\textgreater{} yaxis(2,2\^{}(-2:2),style=``.'',\textless left); \ldots{}\\
\textgreater{} labelbox({[}``2\textsuperscript{x'',''1'',''2}-x''{]},colors=4:6,x=0.8,y=0.2); \ldots{}\\
\textgreater{} reset:

\begin{figure}
\centering
\pandocbounded{\includegraphics[keepaspectratio]{images/EMT4Plot2D_Hikmatul Utami_23030630082_Matematika B-080.png}}
\caption{images/EMT4Plot2D\_Hikmatul\%20Utami\_23030630082\_Matematika\%20B-080.png}
\end{figure}

Berikut adalah contoh lain, di mana string Unicode digunakan dan sumbunya berada di luar area plot.

\textgreater aspect(1.5);

\textgreater plot2d({[}``sin(x)'',``cos(x)''{]},0,2pi,color={[}red,green{]},\textless grid,\textless frame); \ldots{}\\
\textgreater{} xaxis(-1.1,(0:2)*pi,xt={[}``0'',u''π``,u''2π''{]},style=``-'',\textgreater ticks,\textgreater zero); \ldots{}\\
\textgreater{} xgrid((0:0.5:2)*pi,\textless ticks); \ldots{}\\
\textgreater{} yaxis(-0.1*pi,-1:0.2:1,style=``-'',\textgreater zero,\textgreater grid); \ldots{}\\
\textgreater{} labelbox({[}``sin'',``cos''{]},colors={[}red,green{]},x=0.5,y=0.2,\textgreater left); \ldots{}\\
\textgreater{} xlabel(u''φ``); ylabel(u''f(φ)``):

\begin{figure}
\centering
\pandocbounded{\includegraphics[keepaspectratio]{images/EMT4Plot2D_Hikmatul Utami_23030630082_Matematika B-081.png}}
\caption{images/EMT4Plot2D\_Hikmatul\%20Utami\_23030630082\_Matematika\%20B-081.png}
\end{figure}

\chapter{Merencanakan Data 2D}\label{merencanakan-data-2d}

Jika x dan y adalah vektor data, maka data tersebut akan digunakan sebagai koordinat x dan y pada suatu kurva. Dalam hal ini, a, b, c, dan d, atau radius r dapat ditentukan, atau jendela plot akan menyesuaikan secara otomatis dengan data. Alternatifnya, \textgreater persegi dapat diatur untuk mempertahankan rasio aspek persegi.

Merencanakan ekspresi hanyalah singkatan dari plot data. Untuk plot data, Anda memerlukan satu atau beberapa baris nilai x, dan satu atau beberapa baris nilai y. Dari rentang dan nilai x, fungsi plot2d akan menghitung data yang akan diplot, secara default dengan evaluasi fungsi yang adaptif. Untuk plot titik gunakan ``\textgreater titik'', untuk garis dan titik campuran gunakan ``\textgreater addpoints''.

Tapi Anda bisa memasukkan data secara langsung. + Gunakan vektor baris untuk x dan y untuk satu fungsi. + Matriks untuk x dan y diplot baris demi baris.

Berikut adalah contoh dengan satu baris untuk x dan y.

\textgreater x=-10:0.1:10; y=exp(-x\^{}2)*x; plot2d(x,y):

\begin{figure}
\centering
\pandocbounded{\includegraphics[keepaspectratio]{images/EMT4Plot2D_Hikmatul Utami_23030630082_Matematika B-082.png}}
\caption{images/EMT4Plot2D\_Hikmatul\%20Utami\_23030630082\_Matematika\%20B-082.png}
\end{figure}

Data juga dapat diplot sebagai poin. Gunakan points=true untuk ini. Plotnya berfungsi seperti poligon, tetapi hanya menggambar sudutnya saja. + style=``\ldots{}'': Pilih dari ``{[}{]}'', ``\textless\textgreater{}'', ``o'', ``.'', ``..'', ``+'', ``*``,''{[}{]}\#``,''\textless{} \textgreater\#``,''o\#``,''..\#``,''\#``,''\textbar``.

Untuk memplot kumpulan titik, gunakan \textgreater titik. Jika warna merupakan vektor warna, masing-masing titik

mendapat warna berbeda. Untuk matriks koordinat dan vektor kolom, warna diterapkan pada baris matriks.

Parameter \textgreater addpoints menambahkan titik ke segmen garis untuk plot data.

\textgreater xdata={[}1,1.5,2.5,3,4{]}; ydata={[}3,3.1,2.8,2.9,2.7{]}; // data

\textgreater plot2d(xdata,ydata,a=0.5,b=4.5,c=2.5,d=3.5,style=``.''); // lines

\textgreater plot2d(xdata,ydata,\textgreater points,\textgreater add,style=``o''): // add points

\begin{figure}
\centering
\pandocbounded{\includegraphics[keepaspectratio]{images/EMT4Plot2D_Hikmatul Utami_23030630082_Matematika B-083.png}}
\caption{images/EMT4Plot2D\_Hikmatul\%20Utami\_23030630082\_Matematika\%20B-083.png}
\end{figure}

\textgreater p=polyfit(xdata,ydata,1); // get regression line

\textgreater plot2d(``polyval(p,x)'',\textgreater add,color=red): // add plot of line

\begin{figure}
\centering
\pandocbounded{\includegraphics[keepaspectratio]{images/EMT4Plot2D_Hikmatul Utami_23030630082_Matematika B-084.png}}
\caption{images/EMT4Plot2D\_Hikmatul\%20Utami\_23030630082\_Matematika\%20B-084.png}
\end{figure}

\chapter{Menggambar Daerah Yang Dibatasi Kurva}\label{menggambar-daerah-yang-dibatasi-kurva}

Plot data sebenarnya berbentuk poligon. Kita juga dapat memplot kurva atau kurva terisi. + terisi=benar mengisi plot. + style=``\ldots{}'': Pilih dari ``\#'', ``/'', ``",''/``. + Fillcolor : Lihat di atas untuk mengetahui warna yang tersedia.

Warna isian ditentukan oleh argumen ``fillcolor'', dan pada \textless outline opsional, mencegah menggambar batas untuk semua gaya kecuali gaya default.

\textgreater t=linspace(0,2pi,1000); // parameter for curve

\textgreater x=sin(t)*exp(t/pi); y=cos(t)*exp(t/pi); // x(t) and y(t)

\textgreater figure(1,2); aspect(16/9)

\textgreater figure(1); plot2d(x,y,r=10); // plot curve

\textgreater figure(2); plot2d(x,y,r=10,\textgreater filled,style=``/'',fillcolor=red); // fill curve

\textgreater figure(0):

\begin{figure}
\centering
\pandocbounded{\includegraphics[keepaspectratio]{images/EMT4Plot2D_Hikmatul Utami_23030630082_Matematika B-085.png}}
\caption{images/EMT4Plot2D\_Hikmatul\%20Utami\_23030630082\_Matematika\%20B-085.png}
\end{figure}

Dalam contoh berikut kita memplot elips terisi dan dua segi enam terisi menggunakan kurva tertutup dengan 6 titik dengan gaya isian berbeda.

\textgreater x=linspace(0,2pi,1000); plot2d(sin(x),cos(x)*0.5,r=1,\textgreater filled,style=``/''):

\begin{figure}
\centering
\pandocbounded{\includegraphics[keepaspectratio]{images/EMT4Plot2D_Hikmatul Utami_23030630082_Matematika B-086.png}}
\caption{images/EMT4Plot2D\_Hikmatul\%20Utami\_23030630082\_Matematika\%20B-086.png}
\end{figure}

\textgreater t=linspace(0,2pi,6); \ldots{}\\
\textgreater{} plot2d(cos(t),sin(t),\textgreater filled,style=``/'',fillcolor=red,r=1.2):

\begin{figure}
\centering
\pandocbounded{\includegraphics[keepaspectratio]{images/EMT4Plot2D_Hikmatul Utami_23030630082_Matematika B-087.png}}
\caption{images/EMT4Plot2D\_Hikmatul\%20Utami\_23030630082\_Matematika\%20B-087.png}
\end{figure}

\textgreater t=linspace(0,2pi,6); plot2d(cos(t),sin(t),\textgreater filled,style=``\#''):

\begin{figure}
\centering
\pandocbounded{\includegraphics[keepaspectratio]{images/EMT4Plot2D_Hikmatul Utami_23030630082_Matematika B-088.png}}
\caption{images/EMT4Plot2D\_Hikmatul\%20Utami\_23030630082\_Matematika\%20B-088.png}
\end{figure}

Contoh lainnya adalah septagon yang kita buat dengan 7 titik pada lingkaran satuan.

\textgreater t=linspace(0,2pi,7); \ldots{}\\
\textgreater{} plot2d(cos(t),sin(t),r=1,\textgreater filled,style=``/'',fillcolor=red):

\begin{figure}
\centering
\pandocbounded{\includegraphics[keepaspectratio]{images/EMT4Plot2D_Hikmatul Utami_23030630082_Matematika B-089.png}}
\caption{images/EMT4Plot2D\_Hikmatul\%20Utami\_23030630082\_Matematika\%20B-089.png}
\end{figure}

Berikut adalah himpunan nilai maksimal dari empat kondisi linier yang kurang dari atau sama dengan 3. Ini adalah A{[}k{]}.v\textless=3 untuk semua baris A. Untuk mendapatkan sudut yang bagus, kita menggunakan n yang relatif besar.

\textgreater A={[}2,1;1,2;-1,0;0,-1{]};

\textgreater function f(x,y) := max({[}x,y{]}.A');

\textgreater plot2d(``f'',r=4,level={[}0;3{]},color=green,n=111):

\begin{figure}
\centering
\pandocbounded{\includegraphics[keepaspectratio]{images/EMT4Plot2D_Hikmatul Utami_23030630082_Matematika B-090.png}}
\caption{images/EMT4Plot2D\_Hikmatul\%20Utami\_23030630082\_Matematika\%20B-090.png}
\end{figure}

Poin utama dari bahasa matriks adalah memungkinkan pembuatan tabel fungsi dengan mudah.

\textgreater t=linspace(0,2pi,1000); x=cos(3*t); y=sin(4*t);

Kami sekarang memiliki nilai vektor x dan y. plot2d() dapat memplot nilai-nilai ini sebagai kurva yang menghubungkan titik-titik tersebut. Plotnya bisa diisi. Dalam hal ini menghasilkan hasil yang bagus karena aturan belitan, yang digunakan untuk isi.

\textgreater plot2d(x,y,\textless grid,\textless frame,\textgreater filled):

\begin{figure}
\centering
\pandocbounded{\includegraphics[keepaspectratio]{images/EMT4Plot2D_Hikmatul Utami_23030630082_Matematika B-091.png}}
\caption{images/EMT4Plot2D\_Hikmatul\%20Utami\_23030630082\_Matematika\%20B-091.png}
\end{figure}

Vektor interval diplot terhadap nilai x sebagai wilayah terisi antara nilai interval yang lebih rendah dan lebih tinggi. Hal ini dapat berguna untuk memplot kesalahan perhitungan. Tapi itu bisa juga dapat digunakan untuk memplot kesalahan statistik.

\textgreater t=0:0.1:1; \ldots{}\\
\textgreater{} plot2d(t,interval(t-random(size(t)),t+random(size(t))),style=``\textbar{}''); \ldots{}\\
\textgreater{} plot2d(t,t,add=true):

\begin{figure}
\centering
\pandocbounded{\includegraphics[keepaspectratio]{images/EMT4Plot2D_Hikmatul Utami_23030630082_Matematika B-092.png}}
\caption{images/EMT4Plot2D\_Hikmatul\%20Utami\_23030630082\_Matematika\%20B-092.png}
\end{figure}

Jika x adalah vektor yang diurutkan, dan y adalah vektor interval, maka plot2d akan memplot rentang interval yang terisi pada bidang. Gaya isiannya sama dengan gaya poligon.

\textgreater t=-1:0.01:1; x=\textsubscript{t-0.01,t+0.01}; y=x\^{}3-x;

\textgreater plot2d(t,y):

\begin{figure}
\centering
\pandocbounded{\includegraphics[keepaspectratio]{images/EMT4Plot2D_Hikmatul Utami_23030630082_Matematika B-093.png}}
\caption{images/EMT4Plot2D\_Hikmatul\%20Utami\_23030630082\_Matematika\%20B-093.png}
\end{figure}

Dimungkinkan untuk mengisi wilayah nilai untuk fungsi tertentu. Untuk ini, level harus berupa matriks 2xn. Baris pertama adalah batas bawah dan baris kedua berisi batas atas.

\textgreater expr := ``2*x\textsuperscript{2+x*y+3*y}4+y''; // define an expression f(x,y)

\textgreater plot2d(expr,level={[}0;1{]},style=``-'',color=blue): // 0 \textless= f(x,y) \textless= 1

\begin{figure}
\centering
\pandocbounded{\includegraphics[keepaspectratio]{images/EMT4Plot2D_Hikmatul Utami_23030630082_Matematika B-094.png}}
\caption{images/EMT4Plot2D\_Hikmatul\%20Utami\_23030630082\_Matematika\%20B-094.png}
\end{figure}

Kita juga dapat mengisi rentang nilai seperti

\[-1 \le (x^2+y^2)^2-x^2+y^2 \le 0.\]\textgreater plot2d(``(x\textsuperscript{2+y}2)\textsuperscript{2-x}2+y\^{}2'',r=1.2,level={[}-1;0{]},style=``/''):

\begin{figure}
\centering
\pandocbounded{\includegraphics[keepaspectratio]{images/EMT4Plot2D_Hikmatul Utami_23030630082_Matematika B-096.png}}
\caption{images/EMT4Plot2D\_Hikmatul\%20Utami\_23030630082\_Matematika\%20B-096.png}
\end{figure}

\textgreater plot2d(``cos(x)'',``sin(x)\^{}3'',xmin=0,xmax=2pi,\textgreater filled,style=``/''):

\begin{figure}
\centering
\pandocbounded{\includegraphics[keepaspectratio]{images/EMT4Plot2D_Hikmatul Utami_23030630082_Matematika B-097.png}}
\caption{images/EMT4Plot2D\_Hikmatul\%20Utami\_23030630082\_Matematika\%20B-097.png}
\end{figure}

\chapter{Grafik Fungsi Parametrik}\label{grafik-fungsi-parametrik}

Nilai x tidak perlu diurutkan. (x,y) hanya menggambarkan sebuah kurva. Jika x diurutkan, kurva tersebut merupakan grafik suatu fungsi.

Dalam contoh berikut, kita memplot spiral

\[\gamma(t) = t \cdot (\cos(2\pi t),\sin(2\pi t))\]Kita perlu menggunakan banyak titik untuk tampilan yang halus atau fungsi adaptif() untuk mengevaluasi ekspresi (lihat fungsi adaptif() untuk lebih jelasnya).

\textgreater t=linspace(0,1,1000); \ldots{}\\
\textgreater{} plot2d(t*cos(2*pi*t),t*sin(2*pi*t),r=1):

\begin{figure}
\centering
\pandocbounded{\includegraphics[keepaspectratio]{images/EMT4Plot2D_Hikmatul Utami_23030630082_Matematika B-099.png}}
\caption{images/EMT4Plot2D\_Hikmatul\%20Utami\_23030630082\_Matematika\%20B-099.png}
\end{figure}

Sebagai alternatif, dimungkinkan untuk menggunakan dua ekspresi untuk kurva. Berikut ini plot kurva yang sama seperti di atas.

\textgreater plot2d(``x*cos(2*pi*x)'',``x*sin(2*pi*x)'',xmin=0,xmax=1,r=1):

\begin{figure}
\centering
\pandocbounded{\includegraphics[keepaspectratio]{images/EMT4Plot2D_Hikmatul Utami_23030630082_Matematika B-100.png}}
\caption{images/EMT4Plot2D\_Hikmatul\%20Utami\_23030630082\_Matematika\%20B-100.png}
\end{figure}

\textgreater t=linspace(0,1,1000); r=exp(-t); x=r*cos(2pi*t); y=r*sin(2pi*t);

\textgreater plot2d(x,y,r=1):

\begin{figure}
\centering
\pandocbounded{\includegraphics[keepaspectratio]{images/EMT4Plot2D_Hikmatul Utami_23030630082_Matematika B-101.png}}
\caption{images/EMT4Plot2D\_Hikmatul\%20Utami\_23030630082\_Matematika\%20B-101.png}
\end{figure}

Pada contoh berikutnya, kita memplot kurvanya

\[\gamma(t) = (r(t) \cos(t), r(t) \sin(t))\]with

\[r(t) = 1 + \dfrac{\sin(3t)}{2}.\]\textgreater t=linspace(0,2pi,1000); r=1+sin(3*t)/2; x=r*cos(t); y=r*sin(t); \ldots{}\\
\textgreater{} plot2d(x,y,\textgreater filled,fillcolor=red,style=``/'',r=1.5):

\begin{figure}
\centering
\pandocbounded{\includegraphics[keepaspectratio]{images/EMT4Plot2D_Hikmatul Utami_23030630082_Matematika B-104.png}}
\caption{images/EMT4Plot2D\_Hikmatul\%20Utami\_23030630082\_Matematika\%20B-104.png}
\end{figure}

\chapter{Menggambar Grafik Bilangan Kompleks}\label{menggambar-grafik-bilangan-kompleks}

Serangkaian bilangan kompleks juga dapat diplot. Kemudian titik-titik grid akan dihubungkan. Jika sejumlah garis kisi ditentukan (atau vektor garis kisi 1x2) dalam argumen cgrid, hanya garis kisi tersebut yang terlihat.

Matriks bilangan kompleks secara otomatis akan diplot sebagai kisi-kisi pada bidang kompleks.

Pada contoh berikut, kita memplot gambar lingkaran satuan di bawah fungsi eksponensial. Parameter cgrid menyembunyikan beberapa kurva grid.

\textgreater aspect(); r=linspace(0,1,50); a=linspace(0,2pi,80)'; z=r*exp(I*a);\ldots{}\\
\textgreater{} plot2d(z,a=-1.25,b=1.25,c=-1.25,d=1.25,cgrid=10):

\begin{figure}
\centering
\pandocbounded{\includegraphics[keepaspectratio]{images/EMT4Plot2D_Hikmatul Utami_23030630082_Matematika B-105.png}}
\caption{images/EMT4Plot2D\_Hikmatul\%20Utami\_23030630082\_Matematika\%20B-105.png}
\end{figure}

\textgreater aspect(1.25); r=linspace(0,1,50); a=linspace(0,2pi,200)'; z=r*exp(I*a);

\textgreater plot2d(exp(z),cgrid={[}40,10{]}):

\begin{figure}
\centering
\pandocbounded{\includegraphics[keepaspectratio]{images/EMT4Plot2D_Hikmatul Utami_23030630082_Matematika B-106.png}}
\caption{images/EMT4Plot2D\_Hikmatul\%20Utami\_23030630082\_Matematika\%20B-106.png}
\end{figure}

\textgreater r=linspace(0,1,10); a=linspace(0,2pi,40)'; z=r*exp(I*a);

\textgreater plot2d(exp(z),\textgreater points,\textgreater add):

\begin{figure}
\centering
\pandocbounded{\includegraphics[keepaspectratio]{images/EMT4Plot2D_Hikmatul Utami_23030630082_Matematika B-107.png}}
\caption{images/EMT4Plot2D\_Hikmatul\%20Utami\_23030630082\_Matematika\%20B-107.png}
\end{figure}

Vektor bilangan kompleks secara otomatis diplot sebagai kurva pada bidang kompleks dengan bagian nyata dan bagian imajiner

Dalam contoh, kita memplot lingkaran satuan dengan

\[\gamma(t) = e^{it}\]\textgreater t=linspace(0,2pi,1000); \ldots{}\\
\textgreater{} plot2d(exp(I*t)+exp(4*I*t),r=2):

\begin{figure}
\centering
\pandocbounded{\includegraphics[keepaspectratio]{images/EMT4Plot2D_Hikmatul Utami_23030630082_Matematika B-109.png}}
\caption{images/EMT4Plot2D\_Hikmatul\%20Utami\_23030630082\_Matematika\%20B-109.png}
\end{figure}

\chapter{Plot Statistik}\label{plot-statistik}

Ada banyak fungsi yang dikhususkan pada plot statistik. Salah satu plot yang sering digunakan adalah plot kolom.

Jumlah kumulatif dari nilai terdistribusi normal 0-1 menghasilkan jalan acak.

\textgreater plot2d(cumsum(randnormal(1,1000))):

\begin{figure}
\centering
\pandocbounded{\includegraphics[keepaspectratio]{images/EMT4Plot2D_Hikmatul Utami_23030630082_Matematika B-110.png}}
\caption{images/EMT4Plot2D\_Hikmatul\%20Utami\_23030630082\_Matematika\%20B-110.png}
\end{figure}

Penggunaan dua baris menunjukkan jalan dalam dua dimensi.

\textgreater X=cumsum(randnormal(2,1000)); plot2d(X{[}1{]},X{[}2{]}):

\begin{figure}
\centering
\pandocbounded{\includegraphics[keepaspectratio]{images/EMT4Plot2D_Hikmatul Utami_23030630082_Matematika B-111.png}}
\caption{images/EMT4Plot2D\_Hikmatul\%20Utami\_23030630082\_Matematika\%20B-111.png}
\end{figure}

\textgreater columnsplot(cumsum(random(10)),style=``/'',color=blue):

\begin{figure}
\centering
\pandocbounded{\includegraphics[keepaspectratio]{images/EMT4Plot2D_Hikmatul Utami_23030630082_Matematika B-112.png}}
\caption{images/EMT4Plot2D\_Hikmatul\%20Utami\_23030630082\_Matematika\%20B-112.png}
\end{figure}

Itu juga dapat menampilkan string sebagai label.

\textgreater months={[}``Jan'',``Feb'',``Mar'',``Apr'',``May'',``Jun'', \ldots{}\\
\textgreater{} ``Jul'',``Aug'',``Sep'',``Oct'',``Nov'',``Dec''{]};

\textgreater values={[}10,12,12,18,22,28,30,26,22,18,12,8{]};

\textgreater columnsplot(values,lab=months,color=red,style=``-'');

\textgreater title(``Temperature''):

\begin{figure}
\centering
\pandocbounded{\includegraphics[keepaspectratio]{images/EMT4Plot2D_Hikmatul Utami_23030630082_Matematika B-113.png}}
\caption{images/EMT4Plot2D\_Hikmatul\%20Utami\_23030630082\_Matematika\%20B-113.png}
\end{figure}

\textgreater k=0:10;

\textgreater plot2d(k,bin(10,k),\textgreater bar):

\begin{figure}
\centering
\pandocbounded{\includegraphics[keepaspectratio]{images/EMT4Plot2D_Hikmatul Utami_23030630082_Matematika B-114.png}}
\caption{images/EMT4Plot2D\_Hikmatul\%20Utami\_23030630082\_Matematika\%20B-114.png}
\end{figure}

\textgreater plot2d(k,bin(10,k)); plot2d(k,bin(10,k),\textgreater points,\textgreater add):

\begin{figure}
\centering
\pandocbounded{\includegraphics[keepaspectratio]{images/EMT4Plot2D_Hikmatul Utami_23030630082_Matematika B-115.png}}
\caption{images/EMT4Plot2D\_Hikmatul\%20Utami\_23030630082\_Matematika\%20B-115.png}
\end{figure}

\textgreater plot2d(normal(1000),normal(1000),\textgreater points,grid=6,style=``..''):

\begin{figure}
\centering
\pandocbounded{\includegraphics[keepaspectratio]{images/EMT4Plot2D_Hikmatul Utami_23030630082_Matematika B-116.png}}
\caption{images/EMT4Plot2D\_Hikmatul\%20Utami\_23030630082\_Matematika\%20B-116.png}
\end{figure}

\textgreater plot2d(normal(1,1000),\textgreater distribution,style=``O''):

\begin{figure}
\centering
\pandocbounded{\includegraphics[keepaspectratio]{images/EMT4Plot2D_Hikmatul Utami_23030630082_Matematika B-117.png}}
\caption{images/EMT4Plot2D\_Hikmatul\%20Utami\_23030630082\_Matematika\%20B-117.png}
\end{figure}

\textgreater plot2d(``qnormal'',0,5;2.5,0.5,\textgreater filled):

\begin{figure}
\centering
\pandocbounded{\includegraphics[keepaspectratio]{images/EMT4Plot2D_Hikmatul Utami_23030630082_Matematika B-118.png}}
\caption{images/EMT4Plot2D\_Hikmatul\%20Utami\_23030630082\_Matematika\%20B-118.png}
\end{figure}

Untuk memplot distribusi statistik eksperimental, Anda dapat menggunakan distribution=n dengan plot2d.

\textgreater w=randexponential(1,1000); // exponential distribution

\textgreater plot2d(w,\textgreater distribution): // or distribution=n with n intervals

\begin{figure}
\centering
\pandocbounded{\includegraphics[keepaspectratio]{images/EMT4Plot2D_Hikmatul Utami_23030630082_Matematika B-119.png}}
\caption{images/EMT4Plot2D\_Hikmatul\%20Utami\_23030630082\_Matematika\%20B-119.png}
\end{figure}

Atau Anda dapat menghitung distribusi dari data dan memplot hasilnya dengan \textgreater bar di plot3d, atau dengan plot kolom.

\textgreater w=normal(1000); // 0-1-normal distribution

\textgreater\{x,y\}=histo(w,10,v={[}-6,-4,-2,-1,0,1,2,4,6{]}); // interval bounds v

\textgreater plot2d(x,y,\textgreater bar):

\begin{figure}
\centering
\pandocbounded{\includegraphics[keepaspectratio]{images/EMT4Plot2D_Hikmatul Utami_23030630082_Matematika B-120.png}}
\caption{images/EMT4Plot2D\_Hikmatul\%20Utami\_23030630082\_Matematika\%20B-120.png}
\end{figure}

Fungsi statplot() mengatur gaya dengan string sederhana.

\textgreater statplot(1:10,cumsum(random(10)),``b''):

\begin{figure}
\centering
\pandocbounded{\includegraphics[keepaspectratio]{images/EMT4Plot2D_Hikmatul Utami_23030630082_Matematika B-121.png}}
\caption{images/EMT4Plot2D\_Hikmatul\%20Utami\_23030630082\_Matematika\%20B-121.png}
\end{figure}

\textgreater n=10; i=0:n; \ldots{}\\
\textgreater{} plot2d(i,bin(n,i)/2\^{}n,a=0,b=10,c=0,d=0.3); \ldots{}\\
\textgreater{} plot2d(i,bin(n,i)/2\^{}n,points=true,style=``ow'',add=true,color=blue):

\begin{figure}
\centering
\pandocbounded{\includegraphics[keepaspectratio]{images/EMT4Plot2D_Hikmatul Utami_23030630082_Matematika B-122.png}}
\caption{images/EMT4Plot2D\_Hikmatul\%20Utami\_23030630082\_Matematika\%20B-122.png}
\end{figure}

Selain itu, data dapat diplot sebagai batang. Dalam hal ini, x harus diurutkan dan satu elemen lebih panjang dari y. Batangnya akan memanjang dari x{[}i{]} hingga x{[}i+1{]} dengan nilai y{[}i{]}. Jika x berukuran sama dengan y, maka x akan diperpanjang satu elemen dengan spasi terakhir.

Gaya isian dapat digunakan seperti di atas.

\textgreater n=10; k=bin(n,0:n); \ldots{}\\
\textgreater{} plot2d(-0.5:n+0.5,k,bar=true,fillcolor=lightgray):

\begin{figure}
\centering
\pandocbounded{\includegraphics[keepaspectratio]{images/EMT4Plot2D_Hikmatul Utami_23030630082_Matematika B-123.png}}
\caption{images/EMT4Plot2D\_Hikmatul\%20Utami\_23030630082\_Matematika\%20B-123.png}
\end{figure}

Data untuk plot batang (batang=1) dan histogram (histogram=1) dapat diberikan secara eksplisit dalam xv dan yv, atau dapat dihitung dari distribusi empiris dalam xv dengan \textgreater distribusi (atau distribusi=n). Histogram nilai xv akan dihitung secara otomatis dengan \textgreater histogram. Jika \textgreater even ditentukan, nilai xv akan dihitung dalam interval bilangan bulat.

\textgreater plot2d(normal(10000),distribution=50):

\begin{figure}
\centering
\pandocbounded{\includegraphics[keepaspectratio]{images/EMT4Plot2D_Hikmatul Utami_23030630082_Matematika B-124.png}}
\caption{images/EMT4Plot2D\_Hikmatul\%20Utami\_23030630082\_Matematika\%20B-124.png}
\end{figure}

\textgreater k=0:10; m=bin(10,k); x=(0:11)-0.5; plot2d(x,m,\textgreater bar):

\begin{figure}
\centering
\pandocbounded{\includegraphics[keepaspectratio]{images/EMT4Plot2D_Hikmatul Utami_23030630082_Matematika B-125.png}}
\caption{images/EMT4Plot2D\_Hikmatul\%20Utami\_23030630082\_Matematika\%20B-125.png}
\end{figure}

\textgreater columnsplot(m,k):

\begin{figure}
\centering
\pandocbounded{\includegraphics[keepaspectratio]{images/EMT4Plot2D_Hikmatul Utami_23030630082_Matematika B-126.png}}
\caption{images/EMT4Plot2D\_Hikmatul\%20Utami\_23030630082\_Matematika\%20B-126.png}
\end{figure}

\textgreater plot2d(random(600)*6,histogram=6):

\begin{figure}
\centering
\pandocbounded{\includegraphics[keepaspectratio]{images/EMT4Plot2D_Hikmatul Utami_23030630082_Matematika B-127.png}}
\caption{images/EMT4Plot2D\_Hikmatul\%20Utami\_23030630082\_Matematika\%20B-127.png}
\end{figure}

Untuk distribusi, terdapat parameter distribution=n, yang menghitung nilai secara otomatis dan mencetak distribusi relatif dengan n sub-interval.

\textgreater plot2d(normal(1,1000),distribution=10,style=``\textbackslash/''):

\begin{figure}
\centering
\pandocbounded{\includegraphics[keepaspectratio]{images/EMT4Plot2D_Hikmatul Utami_23030630082_Matematika B-128.png}}
\caption{images/EMT4Plot2D\_Hikmatul\%20Utami\_23030630082\_Matematika\%20B-128.png}
\end{figure}

Dengan parameter even=true, ini akan menggunakan interval bilangan bulat.

\textgreater plot2d(intrandom(1,1000,10),distribution=10,even=true):

\begin{figure}
\centering
\pandocbounded{\includegraphics[keepaspectratio]{images/EMT4Plot2D_Hikmatul Utami_23030630082_Matematika B-129.png}}
\caption{images/EMT4Plot2D\_Hikmatul\%20Utami\_23030630082\_Matematika\%20B-129.png}
\end{figure}

Perhatikan bahwa ada banyak plot statistik yang mungkin berguna. Silahkan lihat tutorial tentang statistik.

\textgreater columnsplot(getmultiplicities(1:6,intrandom(1,6000,6))):

\begin{figure}
\centering
\pandocbounded{\includegraphics[keepaspectratio]{images/EMT4Plot2D_Hikmatul Utami_23030630082_Matematika B-130.png}}
\caption{images/EMT4Plot2D\_Hikmatul\%20Utami\_23030630082\_Matematika\%20B-130.png}
\end{figure}

\textgreater plot2d(normal(1,1000),\textgreater distribution); \ldots{}\\
\textgreater{} plot2d(``qnormal(x)'',color=red,thickness=2,\textgreater add):

\begin{figure}
\centering
\pandocbounded{\includegraphics[keepaspectratio]{images/EMT4Plot2D_Hikmatul Utami_23030630082_Matematika B-131.png}}
\caption{images/EMT4Plot2D\_Hikmatul\%20Utami\_23030630082\_Matematika\%20B-131.png}
\end{figure}

Ada juga banyak plot khusus untuk statistik. Plot kotak menunjukkan kuartil distribusi ini dan banyak outlier. Menurut definisinya, outlier dalam plot kotak adalah data yang melebihi 1,5 kali rentang 50\% tengah plot.

\textgreater M=normal(5,1000); boxplot(quartiles(M)):

\begin{figure}
\centering
\pandocbounded{\includegraphics[keepaspectratio]{images/EMT4Plot2D_Hikmatul Utami_23030630082_Matematika B-132.png}}
\caption{images/EMT4Plot2D\_Hikmatul\%20Utami\_23030630082\_Matematika\%20B-132.png}
\end{figure}

\chapter{Fungsi Implisit}\label{fungsi-implisit}

Plot implisit menunjukkan penyelesaian garis level f(x,y)=level, dengan ``level'' dapat berupa nilai tunggal atau vektor nilai. Jika level = ``auto'', akan ada garis level nc, yang akan tersebar antara fungsi minimum dan maksimum secara merata. Warna yang lebih gelap atau lebih terang dapat ditambahkan dengan \textgreater hue untuk menunjukkan nilai fungsi. Untuk fungsi implisit, xv harus berupa fungsi atau ekspresi parameter x dan y, atau alternatifnya, xv dapat berupa matriks nilai.

Euler dapat menandai garis level

\[f(x,y) = c\]dari fungsi apa pun.

Untuk menggambar himpunan f(x,y)=c untuk satu atau lebih konstanta c, Anda dapat menggunakan plot2d() dengan plot implisitnya pada bidang. Parameter c adalah level=c, dimana c dapat berupa vektor garis level. Selain itu, skema warna dapat digambar di latar belakang untuk menunjukkan nilai fungsi setiap titik dalam plot. Parameter ``n'' menentukan kehalusan plot.

\textgreater aspect(1.5);

\textgreater plot2d(``x\textsuperscript{2+y}2-x*y-x'',r=1.5,level=0,contourcolor=red):

\begin{figure}
\centering
\pandocbounded{\includegraphics[keepaspectratio]{images/EMT4Plot2D_Hikmatul Utami_23030630082_Matematika B-134.png}}
\caption{images/EMT4Plot2D\_Hikmatul\%20Utami\_23030630082\_Matematika\%20B-134.png}
\end{figure}

\textgreater expr := ``2*x\textsuperscript{2+x*y+3*y}4+y''; // define an expression f(x,y)

\textgreater plot2d(expr,level=0): // Solutions of f(x,y)=0

\begin{figure}
\centering
\pandocbounded{\includegraphics[keepaspectratio]{images/EMT4Plot2D_Hikmatul Utami_23030630082_Matematika B-135.png}}
\caption{images/EMT4Plot2D\_Hikmatul\%20Utami\_23030630082\_Matematika\%20B-135.png}
\end{figure}

\textgreater plot2d(expr,level=0:0.5:20,\textgreater hue,contourcolor=white,n=200): // nice

\begin{figure}
\centering
\pandocbounded{\includegraphics[keepaspectratio]{images/EMT4Plot2D_Hikmatul Utami_23030630082_Matematika B-136.png}}
\caption{images/EMT4Plot2D\_Hikmatul\%20Utami\_23030630082\_Matematika\%20B-136.png}
\end{figure}

\textgreater plot2d(expr,level=0:0.5:20,\textgreater hue,\textgreater spectral,n=200,grid=4): // nicer

\begin{figure}
\centering
\pandocbounded{\includegraphics[keepaspectratio]{images/EMT4Plot2D_Hikmatul Utami_23030630082_Matematika B-137.png}}
\caption{images/EMT4Plot2D\_Hikmatul\%20Utami\_23030630082\_Matematika\%20B-137.png}
\end{figure}

Ini juga berfungsi untuk plot data. Namun Anda harus menentukan rentangnya untuk label sumbu.

\textgreater x=-2:0.05:1; y=x'; z=expr(x,y);

\textgreater plot2d(z,level=0,a=-1,b=2,c=-2,d=1,\textgreater hue):

\begin{figure}
\centering
\pandocbounded{\includegraphics[keepaspectratio]{images/EMT4Plot2D_Hikmatul Utami_23030630082_Matematika B-138.png}}
\caption{images/EMT4Plot2D\_Hikmatul\%20Utami\_23030630082\_Matematika\%20B-138.png}
\end{figure}

\textgreater plot2d(``x\textsuperscript{3-y}2'',\textgreater contour,\textgreater hue,\textgreater spectral):

\begin{figure}
\centering
\pandocbounded{\includegraphics[keepaspectratio]{images/EMT4Plot2D_Hikmatul Utami_23030630082_Matematika B-139.png}}
\caption{images/EMT4Plot2D\_Hikmatul\%20Utami\_23030630082\_Matematika\%20B-139.png}
\end{figure}

\textgreater plot2d(``x\textsuperscript{3-y}2'',level=0,contourwidth=3,\textgreater add,contourcolor=red):

\begin{figure}
\centering
\pandocbounded{\includegraphics[keepaspectratio]{images/EMT4Plot2D_Hikmatul Utami_23030630082_Matematika B-140.png}}
\caption{images/EMT4Plot2D\_Hikmatul\%20Utami\_23030630082\_Matematika\%20B-140.png}
\end{figure}

\textgreater z=z+normal(size(z))*0.2;

\textgreater plot2d(z,level=0.5,a=-1,b=2,c=-2,d=1):

\begin{figure}
\centering
\pandocbounded{\includegraphics[keepaspectratio]{images/EMT4Plot2D_Hikmatul Utami_23030630082_Matematika B-141.png}}
\caption{images/EMT4Plot2D\_Hikmatul\%20Utami\_23030630082\_Matematika\%20B-141.png}
\end{figure}

\textgreater plot2d(expr,level={[}0:0.2:5;0.05:0.2:5.05{]},color=lightgray):

\begin{figure}
\centering
\pandocbounded{\includegraphics[keepaspectratio]{images/EMT4Plot2D_Hikmatul Utami_23030630082_Matematika B-142.png}}
\caption{images/EMT4Plot2D\_Hikmatul\%20Utami\_23030630082\_Matematika\%20B-142.png}
\end{figure}

\textgreater plot2d(``x\textsuperscript{2+y}3+x*y'',level=1,r=4,n=100):

\begin{figure}
\centering
\pandocbounded{\includegraphics[keepaspectratio]{images/EMT4Plot2D_Hikmatul Utami_23030630082_Matematika B-143.png}}
\caption{images/EMT4Plot2D\_Hikmatul\%20Utami\_23030630082\_Matematika\%20B-143.png}
\end{figure}

\textgreater plot2d(``x\textsuperscript{2+2*y}2-x*y'',level=0:0.1:10,n=100,contourcolor=white,\textgreater hue):

\begin{figure}
\centering
\pandocbounded{\includegraphics[keepaspectratio]{images/EMT4Plot2D_Hikmatul Utami_23030630082_Matematika B-144.png}}
\caption{images/EMT4Plot2D\_Hikmatul\%20Utami\_23030630082\_Matematika\%20B-144.png}
\end{figure}

Dimungkinkan juga untuk mengisi set

\[a \le f(x,y) \le b\]dengan rentang level.

Dimungkinkan untuk mengisi wilayah nilai untuk fungsi tertentu. Untuk ini, level harus berupa matriks 2xn. Baris pertama adalah batas bawah dan baris kedua berisi batas atas.

\textgreater plot2d(expr,level={[}0;1{]},style=``-'',color=blue): // 0 \textless= f(x,y) \textless= 1

\begin{figure}
\centering
\pandocbounded{\includegraphics[keepaspectratio]{images/EMT4Plot2D_Hikmatul Utami_23030630082_Matematika B-146.png}}
\caption{images/EMT4Plot2D\_Hikmatul\%20Utami\_23030630082\_Matematika\%20B-146.png}
\end{figure}

Plot implisit juga dapat menunjukkan rentang level. Maka level harus berupa matriks interval level 2xn, di mana baris pertama berisi awal dan baris kedua berisi akhir setiap interval. Alternatifnya, vektor baris sederhana dapat digunakan untuk level, dan parameter dl memperluas nilai level ke interval.

\textgreater plot2d(``x\textsuperscript{4+y}4'',r=1.5,level={[}0;1{]},color=blue,style=``/''):

\begin{figure}
\centering
\pandocbounded{\includegraphics[keepaspectratio]{images/EMT4Plot2D_Hikmatul Utami_23030630082_Matematika B-147.png}}
\caption{images/EMT4Plot2D\_Hikmatul\%20Utami\_23030630082\_Matematika\%20B-147.png}
\end{figure}

\textgreater plot2d(``x\textsuperscript{2+y}3+x*y'',level={[}0,2,4;1,3,5{]},style=``/'',r=2,n=100):

\begin{figure}
\centering
\pandocbounded{\includegraphics[keepaspectratio]{images/EMT4Plot2D_Hikmatul Utami_23030630082_Matematika B-148.png}}
\caption{images/EMT4Plot2D\_Hikmatul\%20Utami\_23030630082\_Matematika\%20B-148.png}
\end{figure}

\textgreater plot2d(``x\textsuperscript{2+y}3+x*y'',level=-10:20,r=2,style=``-'',dl=0.1,n=100):

\begin{figure}
\centering
\pandocbounded{\includegraphics[keepaspectratio]{images/EMT4Plot2D_Hikmatul Utami_23030630082_Matematika B-149.png}}
\caption{images/EMT4Plot2D\_Hikmatul\%20Utami\_23030630082\_Matematika\%20B-149.png}
\end{figure}

\textgreater plot2d(``sin(x)*cos(y)'',r=pi,\textgreater hue,\textgreater levels,n=100):

\begin{figure}
\centering
\pandocbounded{\includegraphics[keepaspectratio]{images/EMT4Plot2D_Hikmatul Utami_23030630082_Matematika B-150.png}}
\caption{images/EMT4Plot2D\_Hikmatul\%20Utami\_23030630082\_Matematika\%20B-150.png}
\end{figure}

Dimungkinkan juga untuk menandai suatu wilayah

\[a \le f(x,y) \le b.\]Hal ini dilakukan dengan menambahkan level dengan dua baris.

\textgreater plot2d(``(x\textsuperscript{2+y}2-1)\textsuperscript{3-x}2*y\^{}3'',r=1.3, \ldots{}\\
\textgreater{} style=``\#'',color=red,\textless outline, \ldots{}\\
\textgreater{} level={[}-2;0{]},n=100):

\begin{figure}
\centering
\pandocbounded{\includegraphics[keepaspectratio]{images/EMT4Plot2D_Hikmatul Utami_23030630082_Matematika B-152.png}}
\caption{images/EMT4Plot2D\_Hikmatul\%20Utami\_23030630082\_Matematika\%20B-152.png}
\end{figure}

Dimungkinkan untuk menentukan level tertentu. Misalnya, kita dapat memplot solusi persamaan seperti \[x^3-xy+x^2y^2=6\]\textgreater plot2d(``x\textsuperscript{3-x*y+x}2*y\^{}2'',r=6,level=1,n=100):

\begin{figure}
\centering
\pandocbounded{\includegraphics[keepaspectratio]{images/EMT4Plot2D_Hikmatul Utami_23030630082_Matematika B-154.png}}
\caption{images/EMT4Plot2D\_Hikmatul\%20Utami\_23030630082\_Matematika\%20B-154.png}
\end{figure}

\textgreater function starplot1 (v, style=``/'', color=green, lab=none) \ldots{}

\begin{verbatim}
  if !holding() then clg; endif;
  w=window(); window(0,0,1024,1024);
  h=holding(1);
  r=max(abs(v))*1.2;
  setplot(-r,r,-r,r);
  n=cols(v); t=linspace(0,2pi,n);
  v=v|v[1]; c=v*cos(t); s=v*sin(t);
  cl=barcolor(color); st=barstyle(style);
  loop 1 to n
    polygon([0,c[#],c[#+1]],[0,s[#],s[#+1]],1);
    if lab!=none then
      rlab=v[#]+r*0.1;
      {col,row}=toscreen(cos(t[#])*rlab,sin(t[#])*rlab);
      ctext(""+lab[#],col,row-textheight()/2);
    endif;
  end;
  barcolor(cl); barstyle(st);
  holding(h);
  window(w);
endfunction
\end{verbatim}

Dimungkinkan untuk menentukan level tertentu. Misalnya, kita dapat memplot solusi persamaan seperti

\[x^3-xy+x^2y^2=6\]\textgreater reset; starplot1(normal(1,10)+5,color=red,lab=1:10):

\begin{figure}
\centering
\pandocbounded{\includegraphics[keepaspectratio]{images/EMT4Plot2D_Hikmatul Utami_23030630082_Matematika B-156.png}}
\caption{images/EMT4Plot2D\_Hikmatul\%20Utami\_23030630082\_Matematika\%20B-156.png}
\end{figure}

Terkadang, Anda mungkin ingin merencanakan sesuatu yang plot2d tidak bisa lakukan, tapi hampir.

Dalam fungsi berikut, kita membuat plot impuls logaritmik. plot2d dapat melakukan plot logaritmik, tetapi tidak untuk batang impuls.

\textgreater function logimpulseplot1 (x,y) \ldots{}

\begin{verbatim}
  {x0,y0}=makeimpulse(x,log(y)/log(10));
  plot2d(x0,y0,>bar,grid=0);
  h=holding(1);
  frame();
  xgrid(ticks(x));
  p=plot();
  for i=-10 to 10;
    if i<=p[4] and i>=p[3] then
       ygrid(i,yt="10^"+i);
    endif;
  end;
  holding(h);
endfunction
\end{verbatim}

Let us test it with exponentially distributed values.

\textgreater aspect(1.5); x=1:10; y=-log(random(size(x)))*200; \ldots{}\\
\textgreater{} logimpulseplot1(x,y):

\begin{figure}
\centering
\pandocbounded{\includegraphics[keepaspectratio]{images/EMT4Plot2D_Hikmatul Utami_23030630082_Matematika B-157.png}}
\caption{images/EMT4Plot2D\_Hikmatul\%20Utami\_23030630082\_Matematika\%20B-157.png}
\end{figure}

Mari kita menganimasikan kurva 2D menggunakan plot langsung. Perintah plot(x,y) hanya memplot kurva ke dalam jendela plot. setplot(a,b,c,d) menyetel jendela ini.

Fungsi wait(0) memaksa plot muncul di jendela grafis. Jika tidak, pengundian ulang akan dilakukan dalam interval waktu yang jarang.

\textgreater function animliss (n,m) \ldots{}

\begin{verbatim}
t=linspace(0,2pi,500);
f=0;
c=framecolor(0);
l=linewidth(2);
setplot(-1,1,-1,1);
repeat
  clg;
  plot(sin(n*t),cos(m*t+f));
  wait(0);
  if testkey() then break; endif;
  f=f+0.02;
end;
framecolor(c);
linewidth(l);
endfunction
\end{verbatim}

Press any key to stop this animation.

\textgreater animliss(2,3); // lihat hasilnya, jika sudah puas, tekan ENTER

\chapter{Plot Logaritmik}\label{plot-logaritmik}

EMT menggunakan parameter ``logplot'' untuk skala logaritmik.

Plot logaritma dapat diplot menggunakan skala logaritma di y dengan logplot=1, atau menggunakan skala logaritma di x dan y dengan logplot=2, atau di x dengan logplot=3.

\begin{itemize}
\tightlist
\item
  logplot=1: y-logaritma\\
\item
  logplot=2: x-y-logaritma\\
\item
  logplot=3: x-logaritma
\end{itemize}

\textgreater plot2d(``exp(x\textsuperscript{3-x)*x}2'',1,5,logplot=1):

\begin{figure}
\centering
\pandocbounded{\includegraphics[keepaspectratio]{images/EMT4Plot2D_Hikmatul Utami_23030630082_Matematika B-158.png}}
\caption{images/EMT4Plot2D\_Hikmatul\%20Utami\_23030630082\_Matematika\%20B-158.png}
\end{figure}

\textgreater plot2d(``exp(x+sin(x))'',0,100,logplot=1):

\begin{figure}
\centering
\pandocbounded{\includegraphics[keepaspectratio]{images/EMT4Plot2D_Hikmatul Utami_23030630082_Matematika B-159.png}}
\caption{images/EMT4Plot2D\_Hikmatul\%20Utami\_23030630082\_Matematika\%20B-159.png}
\end{figure}

\textgreater plot2d(``exp(x+sin(x))'',10,100,logplot=2):

\begin{figure}
\centering
\pandocbounded{\includegraphics[keepaspectratio]{images/EMT4Plot2D_Hikmatul Utami_23030630082_Matematika B-160.png}}
\caption{images/EMT4Plot2D\_Hikmatul\%20Utami\_23030630082\_Matematika\%20B-160.png}
\end{figure}

\textgreater plot2d(``gamma(x)'',1,10,logplot=1):

\begin{figure}
\centering
\pandocbounded{\includegraphics[keepaspectratio]{images/EMT4Plot2D_Hikmatul Utami_23030630082_Matematika B-161.png}}
\caption{images/EMT4Plot2D\_Hikmatul\%20Utami\_23030630082\_Matematika\%20B-161.png}
\end{figure}

\textgreater plot2d(``log(x*(2+sin(x/100)))'',10,1000,logplot=3):

\begin{figure}
\centering
\pandocbounded{\includegraphics[keepaspectratio]{images/EMT4Plot2D_Hikmatul Utami_23030630082_Matematika B-162.png}}
\caption{images/EMT4Plot2D\_Hikmatul\%20Utami\_23030630082\_Matematika\%20B-162.png}
\end{figure}

Ini juga berfungsi dengan plot data.

\textgreater x=10\^{}(1:20); y=x\^{}2-x;

\textgreater plot2d(x,y,logplot=2):

\begin{figure}
\centering
\pandocbounded{\includegraphics[keepaspectratio]{images/EMT4Plot2D_Hikmatul Utami_23030630082_Matematika B-163.png}}
\caption{images/EMT4Plot2D\_Hikmatul\%20Utami\_23030630082\_Matematika\%20B-163.png}
\end{figure}

\section{CONTOH SOAL}\label{contoh-soal}

1.Buatlah sebuah program yang mensimulasikan 60 data dari distribusi Poisson dengan rata-rata rambda =3 menggunakan kode EMT. Kemudian, buat histogram dari data tersebut dengan 8 interval, menggunakan batas interval berikut:{[}-1,0,1,2,3,4,5,6{]}.

\textgreater w=normal(60); // Poisson distribution with lambda = 3

\textgreater\{x,y\}=histo(w,8,v={[}-1,0,1,2,3,4,5,6{]}); // interval bounds v

\textgreater plot2d(x,y,\textgreater bar):

\begin{figure}
\centering
\pandocbounded{\includegraphics[keepaspectratio]{images/EMT4Plot2D_Hikmatul Utami_23030630082_Matematika B-164.png}}
\caption{images/EMT4Plot2D\_Hikmatul\%20Utami\_23030630082\_Matematika\%20B-164.png}
\end{figure}

2.Buatlah sebuah program yang menghitung dan memplot grafik dari fungsi y = 12x\^{}2-x+1 untuk x yang merupakan nilai dari 1 hingga 15.

\textgreater x=1:15; // nilai x dari 1 hingga 15

\textgreater y=12.*x.\^{}2 - x + 1; // fungsi y = 12x\^{}2 - x + 1

\textgreater plot2d(x,y,logplot=2): // memplot dengan skala logaritmik pada sumbu y

\begin{figure}
\centering
\pandocbounded{\includegraphics[keepaspectratio]{images/EMT4Plot2D_Hikmatul Utami_23030630082_Matematika B-165.png}}
\caption{images/EMT4Plot2D\_Hikmatul\%20Utami\_23030630082\_Matematika\%20B-165.png}
\end{figure}

\begin{enumerate}
\def\labelenumi{\arabic{enumi}.}
\setcounter{enumi}{2}
\tightlist
\item
  Gambar grafik fungsi y=x\^{}3-3
\end{enumerate}

\textgreater gridstyle(``-\textgreater{}'', color=green, textcolor=green, framecolor=green);

\textgreater plot2d(``x\^{}3-x'', grid=1):

\begin{figure}
\centering
\pandocbounded{\includegraphics[keepaspectratio]{images/EMT4Plot2D_Hikmatul Utami_23030630082_Matematika B-166.png}}
\caption{images/EMT4Plot2D\_Hikmatul\%20Utami\_23030630082\_Matematika\%20B-166.png}
\end{figure}

\textgreater label(``x'', 2, 0, pos=``bc'', color=black);

\textgreater label(``y'', 0, 6, pos=``cl'', color=black);

\textgreater reset();

\section{LATIHAN ATAU MENCOBA}\label{latihan-atau-mencoba}

\textgreater plot2d(``(x\textsuperscript{2+y}2-1)\textsuperscript{3-x}2*y\^{}3'',r=1.3, \ldots{}\\
\textgreater{} style=``\#'', color=orange, \textless outline, \ldots{}\\
\textgreater{} level={[}-1.5; 0{]}, n=100):

\begin{figure}
\centering
\pandocbounded{\includegraphics[keepaspectratio]{images/EMT4Plot2D_Hikmatul Utami_23030630082_Matematika B-167.png}}
\caption{images/EMT4Plot2D\_Hikmatul\%20Utami\_23030630082\_Matematika\%20B-167.png}
\end{figure}

\textgreater plot2d(cumsum(randnormal(5,60))):

\begin{figure}
\centering
\pandocbounded{\includegraphics[keepaspectratio]{images/EMT4Plot2D_Hikmatul Utami_23030630082_Matematika B-168.png}}
\caption{images/EMT4Plot2D\_Hikmatul\%20Utami\_23030630082\_Matematika\%20B-168.png}
\end{figure}

\chapter{Rujukan Lengkap Fungsi plot2d()}\label{rujukan-lengkap-fungsi-plot2d}

function plot2d (xv, yv, btest, a, b, c, d, xmin, xmax, r, n, ..\\
logplot, grid, frame, framecolor, square, color, thickness, style, ..\\
auto, add, user, delta, points, addpoints, pointstyle, bar, histogram, ..\\
distribution, even, steps, own, adaptive, hue, level, contour, ..\\
nc, filled, fillcolor, outline, title, xl, yl, maps, contourcolor, ..\\
contourwidth, ticks, margin, clipping, cx, cy, insimg, spectral, ..\\
cgrid, vertical, smaller, dl, niveau, levels)

Multipurpose plot function for plots in the plane (2D plots). This function can do plots of functions of one variables, data plots, curves in the plane, bar plots, grids of complex numbers, and implicit plots of functions of two variables.

Parameters

x,y : equations, functions or data vectors a,b,c,d : Plot area (default a=-2,b=2) r : if r is set, then a=cx-r, b=cx+r, c=cy-r, d=cy+r

\begin{verbatim}
        r can be a vector [rx,ry] or a vector [rx1,rx2,ry1,ry2].
\end{verbatim}

xmin,xmax : range of the parameter for curves auto : Determine y-range automatically (default) square : if true, try to keep square x-y-ranges n : number of intervals (default is adaptive) grid :

\begin{verbatim}
        0 = no grid and labels,
        1 = axis only,
        2 = normal grid (see below for the number of grid lines)
        3 = inside axis
        4 = no grid
        5 = full grid including margin
        6 = ticks at the frame
        7 = axis only
        8 = axis only, sub-ticks
\end{verbatim}

frame : 0 = no frame

framecolor: color of the frame and the grid

margin : number between 0 and 0.4 for the margin around the plot

color : Color of curves. If this is a vector of colors,

\begin{verbatim}
        it will be used for each row of a matrix of plots. In the case of


        point plots, it should be a column vector. If a row vector or a


        full matrix of colors is used for point plots, it will be used for


        each data point.
\end{verbatim}

thickness : line thickness for curves

\begin{verbatim}
        This value can be smaller than 1 for very thin lines.
\end{verbatim}

style : Plot style for lines, markers, and fills.

\begin{verbatim}
        For points use


        "[]", "&lt;&gt;", ".", "..", "...",


        "*", "+", "|", "-", "o"


        "[]#", "&lt;&gt;#", "o#" (filled shapes)


        "[]w", "&lt;&gt;w", "ow" (non-transparent)


        For lines use


        "-", "--", "-.", ".", ".-.", "-.-", "-&gt;"


        For filled polygons or bar plots use


        "#", "#O", "O", "/", "\", "\/",


        "+", "|", "-", "t"
\end{verbatim}

points : plot single points instead of line segments

addpoints : if true, plots line segments and points

add : add the plot to the existing plot

user : enable user interaction for functions

delta : step size for user interaction

bar : bar plot (x are the interval bounds, y the interval values)

histogram : plots the frequencies of x in n subintervals

distribution=n : plots the distribution of x with n subintervals

even : use inter values for automatic histograms.

steps : plots the function as a step function (steps=1,2)

adaptive : use adaptive plots (n is the minimal number of steps)

level : plot level lines of an implicit function of two variables

outline : draws boundary of level ranges.

If the level value is a 2xn matrix, ranges of levels will be drawn in the color using the given fill style. If outline is true, it will be drawn in the contour color. Using this feature, regions of f(x,y) between limits can be marked.

hue : add hue color to the level plot to indicate the function

\begin{verbatim}
        value
\end{verbatim}

contour : Use level plot with automatic levels

nc : number of automatic level lines

title : plot title (default ``\,``)

xl, yl : labels for the x- and y-axis

smaller : if \textgreater0, there will be more space to the left for labels.

vertical :

Turns vertical labels on or off. This changes the global variable verticallabels locally for one plot. The value 1 sets only vertical text, the value 2 uses vertical numerical labels on the y axis.

filled : fill the plot of a curve

fillcolor : fill color for bar and filled curves

outline : boundary for filled polygons

logplot : set logarithmic plots

\begin{verbatim}
        1 = logplot in y,


        2 = logplot in xy,


        3 = logplot in x
\end{verbatim}

own :

A string, which points to an own plot routine. With \textgreater user, you get the same user interaction as in plot2d. The range will be set before each call to your function.

maps : map expressions (0 is faster), functions are always mapped.

contourcolor : color of contour lines

contourwidth : width of contour lines

clipping : toggles the clipping (default is true)

title :

This can be used to describe the plot. The title will appear above the plot. Moreover, a label for the x and y axis can be added with xl=``string'' or yl=``string''. Other labels can be added with the functions label() or labelbox(). The title can be a unicode string or an image of a Latex formula.

cgrid :

Determines the number of grid lines for plots of complex grids. Should be a divisor of the the matrix size minus 1 (number of subintervals). cgrid can be a vector {[}cx,cy{]}. Overview The function can plot + expressions, call collections or functions of one variable, + parametric curves, + x data against y data,

\begin{itemize}
\item
  implicit functions,
\item
  bar plots,
\item
  complex grids,
\item
  polygons.
\end{itemize}

If a function or expression for xv is given, plot2d() will compute values in the given range using the function or expression. The expression must be an expression in the variable x. The range must be defined in the parameters a and b unless the default range {[}-2,2{]} should be used. The y-range will be computed automatically, unless c and d are specified, or a radius r, which yields the range {[}-r,r{]} for x and y. For plots of functions, plot2d will use an adaptive evaluation of the function by default. To speed up the plot for complicated functions, switch this off with \textless adaptive, and optionally decrease the number of intervals n.~Moreover, plot2d() will by default use mapping. I.e., it will compute the plot element for element. If your expression or your functions can handle a vector x, you can switch that off with \textless maps for faster evaluation. Note that adaptive plots are always computed element for element. If functions or expressions for both xv and for yv are specified, plot2d() will compute a curve with the xv values as x-coordinates and the yv values as y-coordinates. In this case, a range should be defined for the parameter using xmin, xmax. Expressions contained in strings must always be expressions in the parameter variable x.

\backmatter
\end{document}
