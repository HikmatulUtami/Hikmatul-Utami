% Options for packages loaded elsewhere
\PassOptionsToPackage{unicode}{hyperref}
\PassOptionsToPackage{hyphens}{url}
\documentclass[
]{book}
\usepackage{xcolor}
\usepackage{amsmath,amssymb}
\setcounter{secnumdepth}{-\maxdimen} % remove section numbering
\usepackage{iftex}
\ifPDFTeX
  \usepackage[T1]{fontenc}
  \usepackage[utf8]{inputenc}
  \usepackage{textcomp} % provide euro and other symbols
\else % if luatex or xetex
  \usepackage{unicode-math} % this also loads fontspec
  \defaultfontfeatures{Scale=MatchLowercase}
  \defaultfontfeatures[\rmfamily]{Ligatures=TeX,Scale=1}
\fi
\usepackage{lmodern}
\ifPDFTeX\else
  % xetex/luatex font selection
\fi
% Use upquote if available, for straight quotes in verbatim environments
\IfFileExists{upquote.sty}{\usepackage{upquote}}{}
\IfFileExists{microtype.sty}{% use microtype if available
  \usepackage[]{microtype}
  \UseMicrotypeSet[protrusion]{basicmath} % disable protrusion for tt fonts
}{}
\makeatletter
\@ifundefined{KOMAClassName}{% if non-KOMA class
  \IfFileExists{parskip.sty}{%
    \usepackage{parskip}
  }{% else
    \setlength{\parindent}{0pt}
    \setlength{\parskip}{6pt plus 2pt minus 1pt}}
}{% if KOMA class
  \KOMAoptions{parskip=half}}
\makeatother
\usepackage{graphicx}
\makeatletter
\newsavebox\pandoc@box
\newcommand*\pandocbounded[1]{% scales image to fit in text height/width
  \sbox\pandoc@box{#1}%
  \Gscale@div\@tempa{\textheight}{\dimexpr\ht\pandoc@box+\dp\pandoc@box\relax}%
  \Gscale@div\@tempb{\linewidth}{\wd\pandoc@box}%
  \ifdim\@tempb\p@<\@tempa\p@\let\@tempa\@tempb\fi% select the smaller of both
  \ifdim\@tempa\p@<\p@\scalebox{\@tempa}{\usebox\pandoc@box}%
  \else\usebox{\pandoc@box}%
  \fi%
}
% Set default figure placement to htbp
\def\fps@figure{htbp}
\makeatother
\setlength{\emergencystretch}{3em} % prevent overfull lines
\providecommand{\tightlist}{%
  \setlength{\itemsep}{0pt}\setlength{\parskip}{0pt}}
\usepackage{bookmark}
\IfFileExists{xurl.sty}{\usepackage{xurl}}{} % add URL line breaks if available
\urlstyle{same}
\hypersetup{
  hidelinks,
  pdfcreator={LaTeX via pandoc}}

\author{}
\date{}

\begin{document}
\frontmatter

\mainmatter
\chapter{Menggambar Plot 3D dengan EMT}\label{menggambar-plot-3d-dengan-emt}

Ini adalah pengantar plot 3D di Euler. Kita memerlukan plot 3D untuk memvisualisasikan fungsi dua variabel.

Euler menggambar fungsi tersebut menggunakan algoritma pengurutan untuk menyembunyikan bagian di latar belakang. Secara umum, Euler menggunakan proyeksi pusat. Defaultnya adalah dari kuadran x-y positif ke arah titik asal x=y=z=0, tetapi sudut=0° terlihat dari arah sumbu y. Sudut pandang dan tinggi dapat diubah.

Euler dapat memplot

\begin{itemize}
\item
  permukaan dengan bayangan dan garis datar atau rentang datar,
\item
  awan titik,
\item
  kurva parametrik,
\item
  permukaan implisit.
\end{itemize}

Plot 3D suatu fungsi menggunakan plot3d. Cara termudah adalah memplot ekspresi dalam x dan y. Parameter r mengatur rentang plot di sekitar (0,0).

\textgreater aspect(1.5); plot3d(``x\^{}2+sin(y)'',-5,5,0,6*pi):

\begin{figure}
\centering
\pandocbounded{\includegraphics[keepaspectratio]{images/EMT4Plot3D_23030630082_Hikmatul Utami-001.png}}
\caption{images/EMT4Plot3D\_23030630082\_Hikmatul\%20Utami-001.png}
\end{figure}

\textgreater plot3d(``x\^{}2+x*sin(y)'',-5,5,0,6*pi):

\begin{figure}
\centering
\pandocbounded{\includegraphics[keepaspectratio]{images/EMT4Plot3D_23030630082_Hikmatul Utami-002.png}}
\caption{images/EMT4Plot3D\_23030630082\_Hikmatul\%20Utami-002.png}
\end{figure}

\chapter{Fungsi Dua Variabel}\label{fungsi-dua-variabel}

Untuk grafik fungsi, gunakan ekspresi sederhana dalam x dan y, nama fungsi dua variabel atau matriks data.

Defaultnya adalah kisi kawat yang terisi dengan warna berbeda di kedua sisinya. Perhatikan bahwa jumlah interval kisi default adalah 10, tetapi plot menggunakan jumlah persegi panjang 40x40 default untuk membuat permukaan. Ini dapat diubah. + n=40, n={[}40,40{]}: jumlah garis kisi di setiap arah + kisi=10, kisi={[}10,10{]}: jumlah garis kisi di setiap arah.

Kami menggunakan default n=40 dan kisi=10.

\textgreater plot3d(``x\textsuperscript{2+y}2''):

\begin{figure}
\centering
\pandocbounded{\includegraphics[keepaspectratio]{images/EMT4Plot3D_23030630082_Hikmatul Utami-003.png}}
\caption{images/EMT4Plot3D\_23030630082\_Hikmatul\%20Utami-003.png}
\end{figure}

Interaksi pengguna dimungkinkan dengan parameter \textgreater user. Pengguna dapat menekan tombol berikut. + kiri, kanan, atas, bawah: mengubah sudut pandang + +, -: memperbesar atau memperkecil + a: menghasilkan anaglif (lihat di bawah) + l: mengubah arah sumber cahaya (lihat di bawah) + spasi: mengatur ulang ke default + return: mengakhiri interaksi

\textgreater plot3d(``exp(-x\textsuperscript{2+y}2)'',\textgreater user, \ldots{}\\
\textgreater{} title=``Turn with the vector keys (press return to finish)''):

\begin{figure}
\centering
\pandocbounded{\includegraphics[keepaspectratio]{images/EMT4Plot3D_23030630082_Hikmatul Utami-004.png}}
\caption{images/EMT4Plot3D\_23030630082\_Hikmatul\%20Utami-004.png}
\end{figure}

Rentang plot untuk fungsi dapat ditentukan dengan

\begin{itemize}
\tightlist
\item
  a,b: rentang x
\item
  c,d: rentang y
\item
  r: persegi simetris di sekitar (0,0).
\item
  n: jumlah subinterval untuk plot.
\end{itemize}

Ada beberapa parameter untuk menskalakan fungsi atau mengubah tampilan grafik. fscale: skala ke nilai fungsi (default adalah \textless fscale). cale: angka atau vektor 1x2 untuk diskalakan ke arah x dan y. frame: jenis frame (default 1).

\textgreater plot3d(``exp(-(x\textsuperscript{2+y}2)/5)'',r=10,n=80,fscale=4,scale=1.2,frame=3,\textgreater user):

\begin{figure}
\centering
\pandocbounded{\includegraphics[keepaspectratio]{images/EMT4Plot3D_23030630082_Hikmatul Utami-005.png}}
\caption{images/EMT4Plot3D\_23030630082\_Hikmatul\%20Utami-005.png}
\end{figure}

Tampilan dapat diubah dengan berbagai cara.

\begin{itemize}
\item
  jarak: jarak pandang ke plot.
\item
  perbesaran: nilai perbesaran.
\item
  sudut: sudut ke sumbu y negatif dalam radian.
\item
  tinggi: tinggi tampilan dalam radian.
\end{itemize}

Nilai default dapat diperiksa atau diubah dengan fungsi view(). Fungsi ini mengembalikan parameter dalam urutan di atas.

\textgreater view

\begin{verbatim}
[5,  2.6,  2,  0.4]
\end{verbatim}

Jarak yang lebih dekat membutuhkan zoom yang lebih sedikit. Efeknya lebih seperti lensa sudut lebar.

Dalam contoh berikut, sudut=0 dan tinggi=0 terlihat dari sumbu y negatif. Label sumbu untuk y disembunyikan dalam kasus ini.

\textgreater plot3d(``x\^{}2+y'',distance=3,zoom=1,angle=pi/2,height=0):

\begin{figure}
\centering
\pandocbounded{\includegraphics[keepaspectratio]{images/EMT4Plot3D_23030630082_Hikmatul Utami-006.png}}
\caption{images/EMT4Plot3D\_23030630082\_Hikmatul\%20Utami-006.png}
\end{figure}

Plot selalu mengarah ke tengah kubus plot. Anda dapat memindahkan bagian tengah dengan parameter center.

\textgreater plot3d(``x\textsuperscript{4+y}2'',a=0,b=1,c=-1,d=1,angle=-20°,height=20°, \ldots{}\\
\textgreater{} center={[}0.4,0,0{]},zoom=5):

\begin{figure}
\centering
\pandocbounded{\includegraphics[keepaspectratio]{images/EMT4Plot3D_23030630082_Hikmatul Utami-007.png}}
\caption{images/EMT4Plot3D\_23030630082\_Hikmatul\%20Utami-007.png}
\end{figure}

Plot diskalakan agar sesuai dengan kubus satuan untuk dilihat. Jadi tidak perlu mengubah jarak atau zoom tergantung pada ukuran plot. Namun, label merujuk pada ukuran sebenarnya.

Jika Anda menonaktifkannya dengan scale=false, Anda perlu berhati-hati agar plot tetap sesuai dengan jendela plot, dengan mengubah jarak tampilan atau zoom, dan memindahkan bagian tengah.

\textgreater plot3d(``5*exp(-x\textsuperscript{2-y}2)'',r=2,\textless fscale,\textless scale,distance=13,height=50°, \ldots{}\\
\textgreater{} center={[}0,0,-2{]},frame=3):

\begin{figure}
\centering
\pandocbounded{\includegraphics[keepaspectratio]{images/EMT4Plot3D_23030630082_Hikmatul Utami-008.png}}
\caption{images/EMT4Plot3D\_23030630082\_Hikmatul\%20Utami-008.png}
\end{figure}

Plot polar juga tersedia. Parameter polar=true menggambar plot polar. Fungsi tersebut harus tetap berupa fungsi x dan y. Parameter ``fscale'' menskalakan fungsi dengan skalanya sendiri. Jika tidak, fungsi tersebut diskalakan agar sesuai dengan kubus.

\textgreater plot3d(``1/(x\textsuperscript{2+y}2+1)'',r=5,\textgreater polar, \ldots{}\\
\textgreater{} fscale=2,\textgreater hue,n=100,zoom=4,\textgreater contour,color=blue):

\begin{figure}
\centering
\pandocbounded{\includegraphics[keepaspectratio]{images/EMT4Plot3D_23030630082_Hikmatul Utami-009.png}}
\caption{images/EMT4Plot3D\_23030630082\_Hikmatul\%20Utami-009.png}
\end{figure}

\textgreater function f(r) := exp(-r/2)*cos(r); \ldots{}\\
\textgreater{} plot3d(``f(x\textsuperscript{2+y}2)'',\textgreater polar,scale={[}1,1,0.4{]},r=pi,frame=3,zoom=4):

\begin{figure}
\centering
\pandocbounded{\includegraphics[keepaspectratio]{images/EMT4Plot3D_23030630082_Hikmatul Utami-010.png}}
\caption{images/EMT4Plot3D\_23030630082\_Hikmatul\%20Utami-010.png}
\end{figure}

Parameter rotate memutar fungsi dalam x di sekitar sumbu x.

\begin{itemize}
\item
  rotate=1: Menggunakan sumbu x
\item
  rotate=2: Menggunakan sumbu z
\end{itemize}

\textgreater plot3d(``x\^{}2+1'',a=-1,b=1,rotate=true,grid=5):

\begin{figure}
\centering
\pandocbounded{\includegraphics[keepaspectratio]{images/EMT4Plot3D_23030630082_Hikmatul Utami-011.png}}
\caption{images/EMT4Plot3D\_23030630082\_Hikmatul\%20Utami-011.png}
\end{figure}

\textgreater plot3d(``x\^{}2+1'',a=-1,b=1,rotate=2,grid=5):

\begin{figure}
\centering
\pandocbounded{\includegraphics[keepaspectratio]{images/EMT4Plot3D_23030630082_Hikmatul Utami-012.png}}
\caption{images/EMT4Plot3D\_23030630082\_Hikmatul\%20Utami-012.png}
\end{figure}

\textgreater plot3d(``sqrt(25-x\^{}2)'',a=0,b=5,rotate=1):

\begin{figure}
\centering
\pandocbounded{\includegraphics[keepaspectratio]{images/EMT4Plot3D_23030630082_Hikmatul Utami-013.png}}
\caption{images/EMT4Plot3D\_23030630082\_Hikmatul\%20Utami-013.png}
\end{figure}

\textgreater plot3d(``x*sin(x)'',a=0,b=6pi,rotate=2):

\begin{figure}
\centering
\pandocbounded{\includegraphics[keepaspectratio]{images/EMT4Plot3D_23030630082_Hikmatul Utami-014.png}}
\caption{images/EMT4Plot3D\_23030630082\_Hikmatul\%20Utami-014.png}
\end{figure}

Berikut adalah plot dengan tiga fungsi.

\textgreater plot3d(``x'',``x\textsuperscript{2+y}2'',``y'',r=2,zoom=3.5,frame=3):

\begin{figure}
\centering
\pandocbounded{\includegraphics[keepaspectratio]{images/EMT4Plot3D_23030630082_Hikmatul Utami-015.png}}
\caption{images/EMT4Plot3D\_23030630082\_Hikmatul\%20Utami-015.png}
\end{figure}

\chapter{Plot Kontur}\label{plot-kontur}

Untuk plot, Euler menambahkan garis kisi. Sebagai gantinya, dimungkinkan untuk menggunakan garis level dan rona satu warna atau rona warna spektral. Euler dapat menggambar tinggi fungsi pada plot dengan bayangan. Dalam semua plot 3D, Euler dapat menghasilkan anaglif merah/sian.

\begin{itemize}
\item
  \textgreater hue: Mengaktifkan bayangan terang alih-alih kabel.
\item
  \textgreater contour: Memplot garis kontur otomatis pada plot.
\item
  level=\ldots{} (atau level): Vektor nilai untuk garis kontur.
\end{itemize}

Nilai default adalah level=``auto'', yang menghitung beberapa garis level secara otomatis. Seperti yang Anda lihat di plot, level sebenarnya adalah rentang level.

Gaya default dapat diubah. Untuk plot kontur berikut, kami menggunakan kisi yang lebih halus untuk titik 100x100, menskalakan fungsi dan plot, dan menggunakan sudut pandang yang berbeda.

\textgreater plot3d(``exp(-x\textsuperscript{2-y}2)'',r=2,n=100,level=``thin'', \ldots{}\\
\textgreater{} \textgreater contour,\textgreater spectral,fscale=1,scale=1.1,angle=45°,height=20°):

\begin{figure}
\centering
\pandocbounded{\includegraphics[keepaspectratio]{images/EMT4Plot3D_23030630082_Hikmatul Utami-016.png}}
\caption{images/EMT4Plot3D\_23030630082\_Hikmatul\%20Utami-016.png}
\end{figure}

\textgreater plot3d(``exp(x*y)'',angle=100°,\textgreater contour,color=green):

\begin{figure}
\centering
\pandocbounded{\includegraphics[keepaspectratio]{images/EMT4Plot3D_23030630082_Hikmatul Utami-017.png}}
\caption{images/EMT4Plot3D\_23030630082\_Hikmatul\%20Utami-017.png}
\end{figure}

Shading default menggunakan warna abu-abu. Namun, rentang warna spektral juga tersedia.

\begin{itemize}
\item
  \textgreater spectral: Menggunakan skema spektral default
\item
  color=\ldots: Menggunakan warna khusus atau skema spektral
\end{itemize}

Untuk plot berikut, kami menggunakan skema spektral default dan menambah jumlah titik untuk mendapatkan tampilan yang sangat halus.

\textgreater plot3d(``x\textsuperscript{2+y}2'',\textgreater spectral,\textgreater contour,n=100):

\begin{figure}
\centering
\pandocbounded{\includegraphics[keepaspectratio]{images/EMT4Plot3D_23030630082_Hikmatul Utami-018.png}}
\caption{images/EMT4Plot3D\_23030630082\_Hikmatul\%20Utami-018.png}
\end{figure}

Alih-alih garis level otomatis, kita juga dapat mengatur nilai garis level. Ini akan menghasilkan garis level tipis alih-alih rentang level.

\textgreater plot3d(``x\textsuperscript{2-y}2'',0,5,0,5,level=-1:0.1:1,color=redgreen):

\begin{figure}
\centering
\pandocbounded{\includegraphics[keepaspectratio]{images/EMT4Plot3D_23030630082_Hikmatul Utami-019.png}}
\caption{images/EMT4Plot3D\_23030630082\_Hikmatul\%20Utami-019.png}
\end{figure}

Dalam plot berikut, kami menggunakan dua pita level yang sangat lebar dari -0,1 hingga 1, dan dari 0,9 hingga 1. Ini dimasukkan sebagai matriks dengan batas level sebagai kolom.

Selain itu, kami melapisi kisi dengan 10 interval di setiap arah.

\textgreater plot3d(``x\textsuperscript{2+y}3'',level={[}-0.1,0.9;0,1{]}, \ldots{}\\
\textgreater{} \textgreater spectral,angle=30°,grid=10,contourcolor=gray):

\begin{figure}
\centering
\pandocbounded{\includegraphics[keepaspectratio]{images/EMT4Plot3D_23030630082_Hikmatul Utami-020.png}}
\caption{images/EMT4Plot3D\_23030630082\_Hikmatul\%20Utami-020.png}
\end{figure}

\textgreater plot3d(``x\textsuperscript{y-y}x'',level=0,a=0,b=6,c=0,d=6,contourcolor=red,n=100):

\begin{figure}
\centering
\pandocbounded{\includegraphics[keepaspectratio]{images/EMT4Plot3D_23030630082_Hikmatul Utami-021.png}}
\caption{images/EMT4Plot3D\_23030630082\_Hikmatul\%20Utami-021.png}
\end{figure}

Dimungkinkan untuk menunjukkan bidang kontur di bawah plot. Warna dan jarak ke plot dapat ditentukan.

\textgreater plot3d(``x\textsuperscript{2+y}4'',\textgreater cp,cpcolor=green,cpdelta=0.2):

\begin{figure}
\centering
\pandocbounded{\includegraphics[keepaspectratio]{images/EMT4Plot3D_23030630082_Hikmatul Utami-022.png}}
\caption{images/EMT4Plot3D\_23030630082\_Hikmatul\%20Utami-022.png}
\end{figure}

Berikut ini beberapa gaya lainnya. Kami selalu menonaktifkan bingkai, dan menggunakan berbagai skema warna untuk plot dan kisi.

\textgreater figure(2,2); \ldots{}\\
\textgreater{} expr=``y\textsuperscript{3-x}2''; \ldots{}\\
\textgreater{} figure(1); \ldots{}\\
\textgreater{} plot3d(expr,\textless frame,\textgreater cp,cpcolor=spectral); \ldots{}\\
\textgreater{} figure(2); \ldots{}\\
\textgreater{} plot3d(expr,\textless frame,\textgreater spectral,grid=10,cp=2); \ldots{}\\
\textgreater{} figure(3); \ldots{}\\
\textgreater{} plot3d(expr,\textless frame,\textgreater contour,color=gray,nc=5,cp=3,cpcolor=greenred); \ldots{}\\
\textgreater{} figure(4); \ldots{}\\
\textgreater{} plot3d(expr,\textless frame,\textgreater hue,grid=10,\textgreater transparent,\textgreater cp,cpcolor=gray); \ldots{}\\
\textgreater{} figure(0):

\begin{figure}
\centering
\pandocbounded{\includegraphics[keepaspectratio]{images/EMT4Plot3D_23030630082_Hikmatul Utami-023.png}}
\caption{images/EMT4Plot3D\_23030630082\_Hikmatul\%20Utami-023.png}
\end{figure}

Ada beberapa skema spektral lain, yang diberi nomor dari 1 hingga 9. Namun, Anda juga dapat menggunakan color=value, di mana value

\begin{itemize}
\item
  spektral: untuk rentang dari biru hingga merah
\item
  putih: untuk rentang yang lebih redup
\item
  kuning biru, ungu hijau, biru kuning, hijau merah
\item
  biru kuning, hijau ungu, kuning biru, merah hijau
\end{itemize}

\textgreater figure(3,3); \ldots{}\\
\textgreater{} for i=1:9; \ldots{}\\
\textgreater{} figure(i); plot3d(``x\textsuperscript{2+y}2'',spectral=i,\textgreater contour,\textgreater cp,\textless frame,zoom=4); \ldots{}\\
\textgreater{} end; \ldots{}\\
\textgreater{} figure(0):

\begin{figure}
\centering
\pandocbounded{\includegraphics[keepaspectratio]{images/EMT4Plot3D_23030630082_Hikmatul Utami-024.png}}
\caption{images/EMT4Plot3D\_23030630082\_Hikmatul\%20Utami-024.png}
\end{figure}

Sumber cahaya dapat diubah dengan l dan tombol kursor selama interaksi pengguna. Sumber cahaya juga dapat diatur dengan parameter.

\begin{itemize}
\item
  light: arah cahaya
\item
  amb: cahaya sekitar antara 0 dan 1
\end{itemize}

Perlu dicatat bahwa program tidak membuat perbedaan antara sisi plot. Tidak ada bayangan. Untuk ini, Anda memerlukan Povray.

\textgreater plot3d(``-x\textsuperscript{2-y}2'', \ldots{}\\
\textgreater{} hue=true,light={[}0,1,1{]},amb=0,user=true, \ldots{}\\
\textgreater{} title=``Press l and cursor keys (return to exit)''):

\begin{figure}
\centering
\pandocbounded{\includegraphics[keepaspectratio]{images/EMT4Plot3D_23030630082_Hikmatul Utami-025.png}}
\caption{images/EMT4Plot3D\_23030630082\_Hikmatul\%20Utami-025.png}
\end{figure}

Parameter warna mengubah warna permukaan. Warna garis level juga dapat diubah.

\textgreater plot3d(``-x\textsuperscript{2-y}2'',color=rgb(0.2,0.2,0),hue=true,frame=false, \ldots{}\\
\textgreater{} zoom=3,contourcolor=red,level=-2:0.1:1,dl=0.01):

\begin{figure}
\centering
\pandocbounded{\includegraphics[keepaspectratio]{images/EMT4Plot3D_23030630082_Hikmatul Utami-026.png}}
\caption{images/EMT4Plot3D\_23030630082\_Hikmatul\%20Utami-026.png}
\end{figure}

Warna 0 memberikan efek pelangi khusus.

\textgreater plot3d(``x\textsuperscript{2/(x}2+y\^{}2+1)'',color=0,hue=true,grid=10):

\begin{figure}
\centering
\pandocbounded{\includegraphics[keepaspectratio]{images/EMT4Plot3D_23030630082_Hikmatul Utami-027.png}}
\caption{images/EMT4Plot3D\_23030630082\_Hikmatul\%20Utami-027.png}
\end{figure}

Permukaannya juga bisa transparan.

\textgreater plot3d(``x\textsuperscript{2+y}2'',\textgreater transparent,grid=10,wirecolor=red):

\begin{figure}
\centering
\pandocbounded{\includegraphics[keepaspectratio]{images/EMT4Plot3D_23030630082_Hikmatul Utami-028.png}}
\caption{images/EMT4Plot3D\_23030630082\_Hikmatul\%20Utami-028.png}
\end{figure}

\chapter{Plot Implisit}\label{plot-implisit}

Ada juga plot implisit dalam tiga dimensi. Euler menghasilkan potongan melalui objek. Fitur plot3d mencakup plot implisit. Plot ini menunjukkan himpunan nol dari suatu fungsi dalam tiga variabel.

Solusi dari

\[f(x,y,z) = 0\]dapat divisualisasikan dalam potongan yang sejajar dengan bidang x-y, x-z, dan y-z.

\begin{itemize}
\item
  implisit=1: potongan sejajar dengan bidang y-z
\item
  implisit=2: potongan sejajar dengan bidang x-z
\item
  implisit=4: potongan sejajar dengan bidang x-y
\end{itemize}

Tambahkan nilai-nilai ini, jika Anda suka. Dalam contoh ini, kami memplot

\[M = \{ (x,y,z) : x^2+y^3+zy=1 \}\]\textgreater plot3d(``x\textsuperscript{2+y}3+z*y-1'',r=5,implicit=3):

\begin{figure}
\centering
\pandocbounded{\includegraphics[keepaspectratio]{images/EMT4Plot3D_23030630082_Hikmatul Utami-031.png}}
\caption{images/EMT4Plot3D\_23030630082\_Hikmatul\%20Utami-031.png}
\end{figure}

\textgreater c=1; d=1;

\textgreater plot3d(``((x\textsuperscript{2+y}2-c\textsuperscript{2)}2+(z\textsuperscript{2-1)}2)*((y\textsuperscript{2+z}2-c\textsuperscript{2)}2+(x\textsuperscript{2-1)}2)*((z\textsuperscript{2+x}2-c\textsuperscript{2)}2+(y\textsuperscript{2-1)}2)-d'',r=2,\textless frame,\textgreater implicit,\textgreater user):

\begin{figure}
\centering
\pandocbounded{\includegraphics[keepaspectratio]{images/EMT4Plot3D_23030630082_Hikmatul Utami-032.png}}
\caption{images/EMT4Plot3D\_23030630082\_Hikmatul\%20Utami-032.png}
\end{figure}

\textgreater plot3d(``x\textsuperscript{2+y}2+4*x*z+z\^{}3'',\textgreater implicit,r=2,zoom=2.5):

\begin{figure}
\centering
\pandocbounded{\includegraphics[keepaspectratio]{images/EMT4Plot3D_23030630082_Hikmatul Utami-033.png}}
\caption{images/EMT4Plot3D\_23030630082\_Hikmatul\%20Utami-033.png}
\end{figure}

\chapter{Plotting 3D Data}\label{plotting-3d-data}

Sama seperti plot2d, plot3d menerima data. Untuk objek 3D, Anda perlu menyediakan matriks nilai x, y, dan z, atau tiga fungsi atau ekspresi fx(x,y), fy(x,y), fz(x,y).

\[\gamma(t,s) = (x(t,s),y(t,s),z(t,s))\]Karena x,y,z adalah matriks, kami berasumsi bahwa (t,s) berjalan melalui kisi persegi. Hasilnya, Anda dapat membuat plot gambar persegi panjang di ruang angkasa.

Anda dapat menggunakan bahasa matriks Euler untuk menghasilkan koordinat secara efektif.

Dalam contoh berikut, kami menggunakan vektor nilai t dan vektor kolom nilai s untuk membuat parameter permukaan bola. Dalam gambar, kami dapat menandai wilayah, dalam kasus kami wilayah kutub.

\textgreater t=linspace(0,2pi,180); s=linspace(-pi/2,pi/2,90)'; \ldots{}\\
\textgreater{} x=cos(s)*cos(t); y=cos(s)*sin(t); z=sin(s); \ldots{}\\
\textgreater{} plot3d(x,y,z,\textgreater hue, \ldots{}\\
\textgreater{} color=blue,\textless frame,grid={[}10,20{]}, \ldots{}\\
\textgreater{} values=s,contourcolor=red,level={[}90°-24°;90°-22°{]}, \ldots{}\\
\textgreater{} scale=1.4,height=50°):

\begin{figure}
\centering
\pandocbounded{\includegraphics[keepaspectratio]{images/EMT4Plot3D_23030630082_Hikmatul Utami-035.png}}
\caption{images/EMT4Plot3D\_23030630082\_Hikmatul\%20Utami-035.png}
\end{figure}

Berikut adalah contoh, yang merupakan grafik suatu fungsi.

\textgreater t=-1:0.1:1; s=(-1:0.1:1)'; plot3d(t,s,t*s,grid=10):

\begin{figure}
\centering
\pandocbounded{\includegraphics[keepaspectratio]{images/EMT4Plot3D_23030630082_Hikmatul Utami-036.png}}
\caption{images/EMT4Plot3D\_23030630082\_Hikmatul\%20Utami-036.png}
\end{figure}

Namun, kita dapat membuat berbagai macam permukaan. Berikut ini adalah permukaan yang sama sebagai fungsi

\[x = y \, z\]\textgreater plot3d(t*s,t,s,angle=180°,grid=10):

\begin{figure}
\centering
\pandocbounded{\includegraphics[keepaspectratio]{images/EMT4Plot3D_23030630082_Hikmatul Utami-038.png}}
\caption{images/EMT4Plot3D\_23030630082\_Hikmatul\%20Utami-038.png}
\end{figure}

Dengan usaha lebih, kita dapat menghasilkan banyak permukaan.

Dalam contoh berikut, kita membuat tampilan berbayang dari bola yang terdistorsi. Koordinat yang biasa untuk bola adalah

\[\gamma(t,s) = (\cos(t)\cos(s),\sin(t)\sin(s),\cos(s))\]dengan

\[0 \le t \le 2\pi, \quad \frac{-\pi}{2} \le s \le \frac{\pi}{2}.\]Kita mendistorsi ini dengan faktor

\[d(t,s) = \frac{\cos(4t)+\cos(8s)}{4}.\]\textgreater t=linspace(0,2pi,320); s=linspace(-pi/2,pi/2,160)'; \ldots{}\\
\textgreater{} d=1+0.2*(cos(4*t)+cos(8*s)); \ldots{}\\
\textgreater{} plot3d(cos(t)*cos(s)*d,sin(t)*cos(s)*d,sin(s)*d,hue=1, \ldots{}\\
\textgreater{} light={[}1,0,1{]},frame=0,zoom=5):

\begin{figure}
\centering
\pandocbounded{\includegraphics[keepaspectratio]{images/EMT4Plot3D_23030630082_Hikmatul Utami-042.png}}
\caption{images/EMT4Plot3D\_23030630082\_Hikmatul\%20Utami-042.png}
\end{figure}

Tentu saja, titik awan juga memungkinkan. Untuk memplot data titik di ruang, kita memerlukan tiga vektor untuk koordinat titik.

Gayanya sama seperti di plot2d dengan points=true;

\textgreater n=500; \ldots{}\\
\textgreater{} plot3d(normal(1,n),normal(1,n),normal(1,n),points=true,style=``.''):

\begin{figure}
\centering
\pandocbounded{\includegraphics[keepaspectratio]{images/EMT4Plot3D_23030630082_Hikmatul Utami-043.png}}
\caption{images/EMT4Plot3D\_23030630082\_Hikmatul\%20Utami-043.png}
\end{figure}

Anda juga dapat memplot kurva dalam 3D. Dalam kasus ini, lebih mudah untuk menghitung titik-titik kurva terlebih dahulu. Untuk kurva dalam bidang, kami menggunakan urutan koordinat dan parameter wire=true.

\textgreater t=linspace(0,8pi,500); \ldots{}\\
\textgreater{} plot3d(sin(t),cos(t),t/10,\textgreater wire,zoom=3):

\begin{figure}
\centering
\pandocbounded{\includegraphics[keepaspectratio]{images/EMT4Plot3D_23030630082_Hikmatul Utami-044.png}}
\caption{images/EMT4Plot3D\_23030630082\_Hikmatul\%20Utami-044.png}
\end{figure}

\textgreater t=linspace(0,4pi,1000); plot3d(cos(t),sin(t),t/2pi,\textgreater wire, \ldots{}\\
\textgreater{} linewidth=3,wirecolor=blue):

\begin{figure}
\centering
\pandocbounded{\includegraphics[keepaspectratio]{images/EMT4Plot3D_23030630082_Hikmatul Utami-045.png}}
\caption{images/EMT4Plot3D\_23030630082\_Hikmatul\%20Utami-045.png}
\end{figure}

\textgreater X=cumsum(normal(3,100)); \ldots{}\\
\textgreater{} plot3d(X{[}1{]},X{[}2{]},X{[}3{]},\textgreater anaglyph,\textgreater wire):

\begin{figure}
\centering
\pandocbounded{\includegraphics[keepaspectratio]{images/EMT4Plot3D_23030630082_Hikmatul Utami-046.png}}
\caption{images/EMT4Plot3D\_23030630082\_Hikmatul\%20Utami-046.png}
\end{figure}

EMT juga dapat membuat grafik dalam mode anaglif. Untuk melihat grafik tersebut, Anda memerlukan kacamata merah/sian.

\textgreater{} plot3d(``x\textsuperscript{2+y}3'',\textgreater anaglyph,\textgreater contour,angle=30°):

\begin{figure}
\centering
\pandocbounded{\includegraphics[keepaspectratio]{images/EMT4Plot3D_23030630082_Hikmatul Utami-047.png}}
\caption{images/EMT4Plot3D\_23030630082\_Hikmatul\%20Utami-047.png}
\end{figure}

Seringkali, skema warna spektral digunakan untuk plot. Ini menekankan tinggi fungsi.

\textgreater plot3d(``x\textsuperscript{2*y}3-y'',\textgreater spectral,\textgreater contour,zoom=3.2):

\begin{figure}
\centering
\pandocbounded{\includegraphics[keepaspectratio]{images/EMT4Plot3D_23030630082_Hikmatul Utami-048.png}}
\caption{images/EMT4Plot3D\_23030630082\_Hikmatul\%20Utami-048.png}
\end{figure}

Euler juga dapat memplot permukaan berparameter, ketika parameternya adalah nilai x, y, dan z dari gambar kotak persegi panjang di ruang tersebut.

Untuk demo berikut, kami menyiapkan parameter u dan v, dan menghasilkan koordinat ruang dari parameter tersebut.

\textgreater u=linspace(-1,1,10); v=linspace(0,2*pi,50)'; \ldots{}\\
\textgreater{} X=(3+u*cos(v/2))*cos(v); Y=(3+u*cos(v/2))*sin(v); Z=u*sin(v/2); \ldots{}\\
\textgreater{} plot3d(X,Y,Z,\textgreater anaglyph,\textless frame,\textgreater wire,scale=2.3):

\begin{figure}
\centering
\pandocbounded{\includegraphics[keepaspectratio]{images/EMT4Plot3D_23030630082_Hikmatul Utami-049.png}}
\caption{images/EMT4Plot3D\_23030630082\_Hikmatul\%20Utami-049.png}
\end{figure}

Berikut adalah contoh yang lebih rumit, yang tampak megah dengan kaca merah/cyan.

\textgreater u:=linspace(-pi,pi,160); v:=linspace(-pi,pi,400)'; \ldots{}\\
\textgreater{} x:=(4*(1+.25*sin(3*v))+cos(u))*cos(2*v); \ldots{}\\
\textgreater{} y:=(4*(1+.25*sin(3*v))+cos(u))*sin(2*v); \ldots{}\\
\textgreater{} z=sin(u)+2*cos(3*v); \ldots{}\\
\textgreater{} plot3d(x,y,z,frame=0,scale=1.5,hue=1,light={[}1,0,-1{]},zoom=2.8,\textgreater anaglyph):

\begin{figure}
\centering
\pandocbounded{\includegraphics[keepaspectratio]{images/EMT4Plot3D_23030630082_Hikmatul Utami-050.png}}
\caption{images/EMT4Plot3D\_23030630082\_Hikmatul\%20Utami-050.png}
\end{figure}

\chapter{Plot Statistik}\label{plot-statistik}

Plot batang juga dimungkinkan. Untuk ini, kita harus menyediakan

\begin{itemize}
\item
  x: vektor baris dengan n+1 elemen
\item
  y: vektor kolom dengan n+1 elemen
\item
  z: matriks nilai nxn.
\end{itemize}

z dapat lebih besar, tetapi hanya nilai nxn yang akan digunakan.

Dalam contoh, pertama-tama kita menghitung nilai. Kemudian kita menyesuaikan x dan y, sehingga vektor berpusat pada nilai yang digunakan.

\textgreater x=-1:0.1:1; y=x'; z=x\textsuperscript{2+y}2; \ldots{}\\
\textgreater{} xa=(x\textbar1.1)-0.05; ya=(y\_1.1)-0.05; \ldots{}\\
\textgreater{} plot3d(xa,ya,z,bar=true):

\begin{figure}
\centering
\pandocbounded{\includegraphics[keepaspectratio]{images/EMT4Plot3D_23030630082_Hikmatul Utami-051.png}}
\caption{images/EMT4Plot3D\_23030630082\_Hikmatul\%20Utami-051.png}
\end{figure}

Dimungkinkan untuk membagi bidang permukaan menjadi dua bagian atau lebih.

\textgreater x=-1:0.1:1; y=x'; z=x+y; d=zeros(size(x)); \ldots{}\\
\textgreater{} plot3d(x,y,z,disconnect=2:2:20):

\begin{figure}
\centering
\pandocbounded{\includegraphics[keepaspectratio]{images/EMT4Plot3D_23030630082_Hikmatul Utami-052.png}}
\caption{images/EMT4Plot3D\_23030630082\_Hikmatul\%20Utami-052.png}
\end{figure}

Jika memuat atau membuat matriks data M dari sebuah file dan perlu memplotnya dalam 3D, Anda dapat menskalakan matriks ke {[}-1,1{]} dengan scale(M), atau menskalakan matriks dengan \textgreater zscale. Ini dapat dikombinasikan dengan faktor penskalaan individual yang diterapkan sebagai tambahan.

\textgreater i=1:20; j=i'; \ldots{}\\
\textgreater{} plot3d(i*j\^{}2+100*normal(20,20),\textgreater zscale,scale={[}1,1,1.5{]},angle=-40°,zoom=1.8):

\begin{figure}
\centering
\pandocbounded{\includegraphics[keepaspectratio]{images/EMT4Plot3D_23030630082_Hikmatul Utami-053.png}}
\caption{images/EMT4Plot3D\_23030630082\_Hikmatul\%20Utami-053.png}
\end{figure}

\textgreater Z=intrandom(5,100,6); v=zeros(5,6); \ldots{}\\
\textgreater{} loop 1 to 5; v{[}\#{]}=getmultiplicities(1:6,Z{[}\#{]}); end; \ldots{}\\
\textgreater{} columnsplot3d(v',scols=1:5,ccols={[}1:5{]}):

\begin{figure}
\centering
\pandocbounded{\includegraphics[keepaspectratio]{images/EMT4Plot3D_23030630082_Hikmatul Utami-054.png}}
\caption{images/EMT4Plot3D\_23030630082\_Hikmatul\%20Utami-054.png}
\end{figure}

\chapter{Permukaan Benda Putar}\label{permukaan-benda-putar}

\textgreater plot2d(``(x\textsuperscript{2+y}2-1)\textsuperscript{3-x}2*y\^{}3'',r=1.3, \ldots{}\\
\textgreater{} style=``\#'',color=red,\textless outline, \ldots{}\\
\textgreater{} level={[}-2;0{]},n=100):

\begin{figure}
\centering
\pandocbounded{\includegraphics[keepaspectratio]{images/EMT4Plot3D_23030630082_Hikmatul Utami-055.png}}
\caption{images/EMT4Plot3D\_23030630082\_Hikmatul\%20Utami-055.png}
\end{figure}

\textgreater ekspresi \&= (x\textsuperscript{2+y}2-1)\textsuperscript{3-x}2*y\^{}3; \$ekspresi

\[\left(y^2+x^2-1\right)^3-x^2\,y^3\]Kita ingin memutar kurva jantung di sekitar sumbu y. Berikut ini adalah ekspresi yang mendefinisikan jantung:

\[f(x,y)=(x^2+y^2-1)^3-x^2.y^3.\]selanjutnya kita tetapkan

\[x=r.cos(a),\quad y=r.sin(a).\]\textgreater function fr(r,a) \&= ekspresi with {[}x=r*cos(a),y=r*sin(a){]} \textbar{} trigreduce; \$fr(r,a)

\[\left(r^2-1\right)^3+\frac{\left(\sin \left(5\,a\right)-\sin \left(3\,a\right)-2\,\sin a\right)\,r^5}{16}\]Hal ini memungkinkan untuk mendefinisikan fungsi numerik, yang memecahkan r, jika a diberikan. Dengan fungsi itu kita dapat memplot jantung yang diputar sebagai permukaan parametrik.

\textgreater function map f(a) := bisect(``fr'',0,2;a); \ldots{}\\
\textgreater{} t=linspace(-pi/2,pi/2,100); r=f(t); \ldots{}\\
\textgreater{} s=linspace(pi,2pi,100)'; \ldots{}\\
\textgreater{} plot3d(r*cos(t)*sin(s),r*cos(t)*cos(s),r*sin(t), \ldots{}\\
\textgreater{} \textgreater hue,\textless frame,color=red,zoom=4,amb=0,max=0.7,grid=12,height=50°):

\begin{figure}
\centering
\pandocbounded{\includegraphics[keepaspectratio]{images/EMT4Plot3D_23030630082_Hikmatul Utami-060.png}}
\caption{images/EMT4Plot3D\_23030630082\_Hikmatul\%20Utami-060.png}
\end{figure}

Berikut ini adalah plot 3D dari gambar di atas yang diputar di sekitar sumbu z. Kami mendefinisikan fungsi yang menggambarkan objek tersebut.

\textgreater function f(x,y,z) \ldots{}

\begin{verbatim}
r=x^2+y^2;
return (r+z^2-1)^3-r*z^3;
 endfunction
\end{verbatim}

\textgreater plot3d(``f(x,y,z)'', \ldots{}\\
\textgreater{} xmin=0,xmax=1.2,ymin=-1.2,ymax=1.2,zmin=-1.2,zmax=1.4, \ldots{}\\
\textgreater{} implicit=1,angle=-30°,zoom=2.5,n={[}10,100,60{]},\textgreater anaglyph):

\begin{figure}
\centering
\pandocbounded{\includegraphics[keepaspectratio]{images/EMT4Plot3D_23030630082_Hikmatul Utami-061.png}}
\caption{images/EMT4Plot3D\_23030630082\_Hikmatul\%20Utami-061.png}
\end{figure}

\chapter{Plot 3D Khusus}\label{plot-3d-khusus}

Fungsi plot3d memang bagus, tetapi tidak memenuhi semua kebutuhan. Selain rutinitas yang lebih mendasar, Anda dapat memperoleh plot berbingkai dari objek apa pun yang Anda suka.

Meskipun Euler bukanlah program 3D, ia dapat menggabungkan beberapa objek dasar. Kami mencoba memvisualisasikan parabola dan garis singgungnya.

\textgreater function myplot \ldots{}

\begin{verbatim}
  y=-1:0.01:1; x=(-1:0.01:1)';
  plot3d(x,y,0.2*(x-0.1)/2,<scale,<frame,>hue, ..
    hues=0.5,>contour,color=orange);
  h=holding(1);
  plot3d(x,y,(x^2+y^2)/2,<scale,<frame,>contour,>hue);
  holding(h);
endfunction
\end{verbatim}

Sekarang framedplot() menyediakan bingkai dan mengatur tampilan.

\textgreater framedplot(``myplot'',{[}-1,1,-1,1,0,1{]},height=0,angle=-30°, \ldots{}\\
\textgreater{} center={[}0,0,-0.7{]},zoom=3):

\begin{figure}
\centering
\pandocbounded{\includegraphics[keepaspectratio]{images/EMT4Plot3D_23030630082_Hikmatul Utami-062.png}}
\caption{images/EMT4Plot3D\_23030630082\_Hikmatul\%20Utami-062.png}
\end{figure}

Dengan cara yang sama, Anda dapat memplot bidang kontur secara manual. Perhatikan bahwa plot3d() menetapkan jendela ke fullwindow() secara default, tetapi plotcontourplane() mengasumsikannya.

\textgreater x=-1:0.02:1.1; y=x'; z=x\textsuperscript{2-y}4;

\textgreater function myplot (x,y,z) \ldots{}\\
\textgreater{}\\

\textgreater myplot(x,y,z):

\begin{figure}
\centering
\pandocbounded{\includegraphics[keepaspectratio]{images/EMT4Plot3D_23030630082_Hikmatul Utami-063.png}}
\caption{images/EMT4Plot3D\_23030630082\_Hikmatul\%20Utami-063.png}
\end{figure}

\chapter{Animasi}\label{animasi}

Euler dapat menggunakan bingkai untuk melakukan pra-komputasi animasi.

Salah satu fungsi yang memanfaatkan teknik ini adalah rotate. Fungsi ini dapat mengubah sudut pandang dan menggambar ulang plot 3D. Fungsi ini memanggil addpage() untuk setiap plot baru. Terakhir, fungsi ini menganimasikan plot tersebut.

Silakan pelajari sumber rotate untuk melihat detail selengkapnya.

\textgreater function testplot () := plot3d(``x\textsuperscript{2+y}3''); \ldots{}\\
\textgreater{} rotate(``testplot''); testplot():

\begin{figure}
\centering
\pandocbounded{\includegraphics[keepaspectratio]{images/EMT4Plot3D_23030630082_Hikmatul Utami-064.png}}
\caption{images/EMT4Plot3D\_23030630082\_Hikmatul\%20Utami-064.png}
\end{figure}

\chapter{Menggambar Povray}\label{menggambar-povray}

Dengan bantuan file Euler povray.e, Euler dapat membuat file Povray. Hasilnya sangat bagus untuk dilihat.

Anda perlu menginstal Povray (32bit atau 64bit) dari \textless a href=``http://www.povray.org/, dan meletakkan subdirektori''bin'' dari Povray ke dalam jalur lingkungan, atau mengatur variabel ``defaultpovray'' dengan jalur lengkap yang mengarah ke ``pvengine.exe''.''\textgreater http://www.povray.org/, dan meletakkan subdirektori ``bin'' dari Povray ke dalam jalur lingkungan, atau mengatur variabel ``defaultpovray'' dengan jalur lengkap yang mengarah ke ``pvengine.exe''.

Antarmuka Povray dari Euler membuat file Povray di direktori home pengguna, dan memanggil Povray untuk mengurai file-file ini. Nama file default adalah current.pov, dan direktori default adalah eulerhome(), biasanya c:\Users\Username\Euler. Povray membuat file PNG, yang dapat dimuat oleh Euler ke dalam buku catatan. Untuk membersihkan file-file ini, gunakan povclear().

Fungsi pov3d memiliki semangat yang sama dengan plot3d. Fungsi ini dapat menghasilkan grafik fungsi f(x,y), atau permukaan dengan koordinat X,Y,Z dalam matriks, termasuk garis level opsional. Fungsi ini memulai raytracer secara otomatis, dan memuat adegan ke dalam buku catatan Euler.

Selain pov3d(), ada banyak fungsi, yang menghasilkan objek Povray. Fungsi-fungsi ini mengembalikan string, yang berisi kode Povray untuk objek. Untuk menggunakan fungsi-fungsi ini, mulai file Povray dengan povstart(). Kemudian gunakan writeln(\ldots) untuk menulis objek ke file adegan. Terakhir, akhiri file dengan povend(). Secara default, raytracer akan mulai, dan PNG akan dimasukkan ke dalam buku catatan Euler.

Fungsi objek memiliki parameter yang disebut ``look'', yang memerlukan string dengan kode Povray untuk tekstur dan penyelesaian objek. Fungsi povlook() dapat digunakan untuk menghasilkan string ini. Fungsi ini memiliki parameter untuk warna, transparansi, Phong Shading, dll.

Perhatikan bahwa alam semesta Povray memiliki sistem koordinat lain. Antarmuka ini menerjemahkan semua koordinat ke sistem Povray. Jadi Anda dapat terus berpikir dalam sistem koordinat Euler dengan z menunjuk vertikal ke atas, dan sumbu x, y, z dalam arah kanan.

nda perlu memuat berkas povray.

\textgreater load povray;

Pastikan direktori bin Povray ada di jalur tersebut. Jika tidak, edit variabel berikut sehingga berisi jalur ke povray yang dapat dieksekusi.

\textgreater defaultpovray=``C:\textbackslash Program Files\textbackslash POV-Ray\textbackslash v3.7\textbackslash bin\textbackslash pvengine.exe''

\begin{verbatim}
C:\Program Files\POV-Ray\v3.7\bin\pvengine.exe
\end{verbatim}

Untuk kesan pertama, kami membuat fungsi sederhana. Perintah berikut menghasilkan file povray di direktori pengguna Anda, dan menjalankan Povray untuk melakukan ray tracing pada file ini.

Jika Anda menjalankan perintah berikut, GUI Povray akan terbuka, menjalankan file, dan menutup secara otomatis. Karena alasan keamanan, Anda akan ditanya apakah Anda ingin mengizinkan file exe untuk berjalan. Anda dapat menekan batal untuk menghentikan pertanyaan lebih lanjut. Anda mungkin harus menekan OK di jendela Povray untuk mengakui dialog awal Povray.

\textgreater plot3d(``x\textsuperscript{2+y}2'',zoom=2):

\begin{figure}
\centering
\pandocbounded{\includegraphics[keepaspectratio]{images/EMT4Plot3D_23030630082_Hikmatul Utami-065.png}}
\caption{images/EMT4Plot3D\_23030630082\_Hikmatul\%20Utami-065.png}
\end{figure}

\textgreater pov3d(``x\textsuperscript{2+y}2'',zoom=3);

\begin{figure}
\centering
\pandocbounded{\includegraphics[keepaspectratio]{images/EMT4Plot3D_23030630082_Hikmatul Utami-066.png}}
\caption{images/EMT4Plot3D\_23030630082\_Hikmatul\%20Utami-066.png}
\end{figure}

Kita dapat membuat fungsi tersebut transparan dan menambahkan penyelesaian lainnya. Kita juga dapat menambahkan garis level pada plot fungsi.

\textgreater pov3d(``x\textsuperscript{2+y}3'',axiscolor=red,angle=-45°,\textgreater anaglyph, \ldots{}\\
\textgreater{} look=povlook(cyan,0.2),level=-1:0.5:1,zoom=3.8);

\begin{figure}
\centering
\pandocbounded{\includegraphics[keepaspectratio]{images/EMT4Plot3D_23030630082_Hikmatul Utami-067.png}}
\caption{images/EMT4Plot3D\_23030630082\_Hikmatul\%20Utami-067.png}
\end{figure}

Terkadang perlu untuk mencegah penskalaan fungsi, dan menskalakan fungsi secara manual.

Kami memplot himpunan titik pada bidang kompleks, di mana hasil kali jarak ke 1 dan -1 sama dengan 1.

\textgreater pov3d(``((x-1)\textsuperscript{2+y}2)*((x+1)\textsuperscript{2+y}2)/40'',r=2, \ldots{}\\
\textgreater{} angle=-120°,level=1/40,dlevel=0.005,light={[}-1,1,1{]},height=10°,n=50, \ldots{}\\
\textgreater{} \textless fscale,zoom=3.8);

\begin{figure}
\centering
\pandocbounded{\includegraphics[keepaspectratio]{images/EMT4Plot3D_23030630082_Hikmatul Utami-068.png}}
\caption{images/EMT4Plot3D\_23030630082\_Hikmatul\%20Utami-068.png}
\end{figure}

\chapter{Membuat Plot dengan Koordinat}\label{membuat-plot-dengan-koordinat}

Alih-alih fungsi, kita dapat membuat plot dengan koordinat. Seperti dalam plot3d, kita memerlukan tiga matriks untuk menentukan objek.

Dalam contoh ini, kita memutar fungsi di sekitar sumbu z.

\textgreater function f(x) := x\^{}3-x+1; \ldots{}\\
\textgreater{} x=-1:0.01:1; t=linspace(0,2pi,50)'; \ldots{}\\
\textgreater{} Z=x; X=cos(t)*f(x); Y=sin(t)*f(x); \ldots{}\\
\textgreater{} pov3d(X,Y,Z,angle=40°,look=povlook(red,0.1),height=50°,axis=0,zoom=4,light={[}10,5,15{]});

\begin{figure}
\centering
\pandocbounded{\includegraphics[keepaspectratio]{images/EMT4Plot3D_23030630082_Hikmatul Utami-069.png}}
\caption{images/EMT4Plot3D\_23030630082\_Hikmatul\%20Utami-069.png}
\end{figure}

Dalam contoh berikut, kami memplot gelombang yang diredam. Kami menghasilkan gelombang dengan bahasa matriks Euler.

Kami juga menunjukkan, bagaimana objek tambahan dapat ditambahkan ke adegan pov3d. Untuk pembuatan objek, lihat contoh berikut. Perhatikan bahwa plot3d menskalakan plot, sehingga sesuai dengan kubus satuan.

\textgreater r=linspace(0,1,80); phi=linspace(0,2pi,80)'; \ldots{}\\
\textgreater{} x=r*cos(phi); y=r*sin(phi); z=exp(-5*r)*cos(8*pi*r)/3; \ldots{}\\
\textgreater{} pov3d(x,y,z,zoom=6,axis=0,height=30°,add=povsphere({[}0.5,0,0.25{]},0.15,povlook(red)), \ldots{}\\
\textgreater{} w=500,h=300);

\begin{figure}
\centering
\pandocbounded{\includegraphics[keepaspectratio]{images/EMT4Plot3D_23030630082_Hikmatul Utami-070.png}}
\caption{images/EMT4Plot3D\_23030630082\_Hikmatul\%20Utami-070.png}
\end{figure}

Dengan metode shading Povray yang canggih, hanya sedikit titik yang dapat menghasilkan permukaan yang sangat halus. Hanya pada batas dan bayangan, triknya mungkin menjadi jelas.

Untuk ini, kita perlu menambahkan vektor normal di setiap titik matriks.

\textgreater Z \&= x\textsuperscript{2*y}3

\begin{verbatim}
                                 2  3
                                x  y
\end{verbatim}

Persamaan permukaannya adalah {[}x,y,Z{]}. Kita hitung dua turunan x dan y dari persamaan ini dan ambil perkalian silang sebagai normalnya.

\textgreater dx \&= diff({[}x,y,Z{]},x); dy \&= diff({[}x,y,Z{]},y);

Kami mendefinisikan normal sebagai perkalian silang turunan-turunan ini dan mendefinisikan fungsi koordinat.

\textgreater N \&= crossproduct(dx,dy); NX \&= N{[}1{]}; NY \&= N{[}2{]}; NZ \&= N{[}3{]}; N,

\begin{verbatim}
                               3       2  2
                       [- 2 x y , - 3 x  y , 1]
\end{verbatim}

Kami hanya menggunakan 25 poin.

\textgreater x=-1:0.5:1; y=x';

\textgreater pov3d(x,y,Z(x,y),angle=10°, \ldots{}\\
\textgreater{} xv=NX(x,y),yv=NY(x,y),zv=NZ(x,y),\textless shadow);

\begin{figure}
\centering
\pandocbounded{\includegraphics[keepaspectratio]{images/EMT4Plot3D_23030630082_Hikmatul Utami-071.png}}
\caption{images/EMT4Plot3D\_23030630082\_Hikmatul\%20Utami-071.png}
\end{figure}

Berikut ini adalah simpul Trefoil yang dibuat oleh A. Busser di Povray. Ada versi yang lebih baik dari simpul ini dalam contoh-contohnya.

Lihat: Contoh\Simpul Trefoil \textbar{} Simpul Trefoil

Untuk tampilan yang bagus dengan tidak terlalu banyak titik, kami menambahkan vektor normal di sini. Kami menggunakan Maxima untuk menghitung normal bagi kami. Pertama, tiga fungsi untuk koordinat sebagai ekspresi simbolik.

\textgreater X \&= ((4+sin(3*y))+cos(x))*cos(2*y); \ldots{}\\
\textgreater{} Y \&= ((4+sin(3*y))+cos(x))*sin(2*y); \ldots{}\\
\textgreater{} Z \&= sin(x)+2*cos(3*y);

Kemudian dua vektor turunan ke x dan y.

\textgreater dx \&= diff({[}X,Y,Z{]},x); dy \&= diff({[}X,Y,Z{]},y);

Sekarang normal, yang merupakan perkalian silang dari dua turunan.

\textgreater dn \&= crossproduct(dx,dy);

Sekarang mari kita evaluasi semua ini secara numerik.

\textgreater x:=linspace(-\%pi,\%pi,40); y:=linspace(-\%pi,\%pi,100)';

Vektor normal adalah evaluasi ekspresi simbolik dn{[}i{]} untuk i=1,2,3. Sintaks untuk ini adalah \&``ekspresi''(parameter). Ini adalah alternatif untuk metode pada contoh sebelumnya, di mana kita mendefinisikan ekspresi simbolik NX, NY, NZ terlebih dahulu.

\textgreater pov3d(X(x,y),Y(x,y),Z(x,y),\textgreater anaglyph,axis=0,zoom=5,w=450,h=350, \ldots{}\\
\textgreater{} \textless shadow,look=povlook(blue), \ldots{}\\
\textgreater{} xv=\&``dn{[}1{]}''(x,y), yv=\&``dn{[}2{]}''(x,y), zv=\&``dn{[}3{]}''(x,y));

\begin{figure}
\centering
\pandocbounded{\includegraphics[keepaspectratio]{images/EMT4Plot3D_23030630082_Hikmatul Utami-072.png}}
\caption{images/EMT4Plot3D\_23030630082\_Hikmatul\%20Utami-072.png}
\end{figure}

Kita juga dapat membuat grid dalam 3D.

\textgreater povstart(zoom=4); \ldots{}\\
\textgreater{} x=-1:0.5:1; r=1-(x+1)\^{}2/6; \ldots{}\\
\textgreater{} t=(0°:30°:360°)'; y=r*cos(t); z=r*sin(t); \ldots{}\\
\textgreater{} writeln(povgrid(x,y,z,d=0.02,dballs=0.05)); \ldots{}\\
\textgreater{} povend();

\begin{figure}
\centering
\pandocbounded{\includegraphics[keepaspectratio]{images/EMT4Plot3D_23030630082_Hikmatul Utami-073.png}}
\caption{images/EMT4Plot3D\_23030630082\_Hikmatul\%20Utami-073.png}
\end{figure}

Dengan povgrid(), kurva dimungkinkan.

\textgreater povstart(center={[}0,0,1{]},zoom=3.6); \ldots{}\\
\textgreater{} t=linspace(0,2,1000); r=exp(-t); \ldots{}\\
\textgreater{} x=cos(2*pi*10*t)*r; y=sin(2*pi*10*t)*r; z=t; \ldots{}\\
\textgreater{} writeln(povgrid(x,y,z,povlook(red))); \ldots{}\\
\textgreater{} writeAxis(0,2,axis=3); \ldots{}\\
\textgreater{} povend();

\begin{figure}
\centering
\pandocbounded{\includegraphics[keepaspectratio]{images/EMT4Plot3D_23030630082_Hikmatul Utami-074.png}}
\caption{images/EMT4Plot3D\_23030630082\_Hikmatul\%20Utami-074.png}
\end{figure}

\chapter{Objek Povray}\label{objek-povray}

Di atas, kami menggunakan pov3d untuk memplot permukaan. Antarmuka povray di Euler juga dapat menghasilkan objek Povray. Objek-objek ini disimpan sebagai string di Euler, dan perlu ditulis ke berkas Povray.

Kami memulai output dengan povstart().

\textgreater povstart(zoom=4);

Pertama, kita mendefinisikan tiga silinder, dan menyimpannya dalam string di Euler.

Fungsi povx() dll. hanya mengembalikan vektor {[}1,0,0{]}, yang dapat digunakan sebagai gantinya.

\textgreater c1=povcylinder(-povx,povx,1,povlook(red)); \ldots{}\\
\textgreater{} c2=povcylinder(-povy,povy,1,povlook(yellow)); \ldots{}\\
\textgreater{} c3=povcylinder(-povz,povz,1,povlook(blue)); \ldots{}\\
\textgreater{}\\
Rangkaian tersebut berisi kode Povray, yang tidak perlu kita pahami saat itu.

\textgreater c2

\begin{verbatim}
cylinder { &lt;0,0,-1&gt;, &lt;0,0,1&gt;, 1
 texture { pigment { color rgb &lt;0.941176,0.941176,0.392157&gt; }  } 
 finish { ambient 0.2 } 
 }
\end{verbatim}

Seperti yang Anda lihat, kami menambahkan tekstur ke objek dalam tiga warna berbeda.

Hal itu dilakukan oleh povlook(), yang mengembalikan string dengan kode Povray yang relevan. Kita dapat menggunakan warna Euler default, atau menentukan warna kita sendiri. Kita juga dapat menambahkan transparansi, atau mengubah cahaya sekitar.

\textgreater povlook(rgb(0.1,0.2,0.3),0.1,0.5)

\begin{verbatim}
 texture { pigment { color rgbf &lt;0.101961,0.2,0.301961,0.1&gt; }  } 
 finish { ambient 0.5 } 
\end{verbatim}

Sekarang kita mendefinisikan objek persimpangan dan menulis hasilnya ke berkas.

\textgreater writeln(povintersection({[}c1,c2,c3{]}));

Persimpangan tiga silinder sulit dibayangkan, jika Anda belum pernah melihatnya sebelumnya.

\textgreater povend;

\begin{figure}
\centering
\pandocbounded{\includegraphics[keepaspectratio]{images/EMT4Plot3D_23030630082_Hikmatul Utami-075.png}}
\caption{images/EMT4Plot3D\_23030630082\_Hikmatul\%20Utami-075.png}
\end{figure}

Fungsi-fungsi berikut menghasilkan fraktal secara rekursif.

Fungsi pertama menunjukkan bagaimana Euler menangani objek-objek Povray sederhana. Fungsi povbox() mengembalikan string yang berisi koordinat kotak, tekstur, dan hasil akhir.

\textgreater function onebox(x,y,z,d) := povbox({[}x,y,z{]},{[}x+d,y+d,z+d{]},povlook());

\textgreater function fractal (x,y,z,h,n) \ldots{}\\
\textgreater{}\\

\textgreater povstart(fade=10,\textless shadow);

\textgreater fractal(-1,-1,-1,2,4);

\textgreater povend();

\begin{figure}
\centering
\pandocbounded{\includegraphics[keepaspectratio]{images/EMT4Plot3D_23030630082_Hikmatul Utami-076.png}}
\caption{images/EMT4Plot3D\_23030630082\_Hikmatul\%20Utami-076.png}
\end{figure}

Perbedaan memungkinkan pemisahan satu objek dari objek lainnya. Seperti halnya persimpangan, ada beberapa objek CSG dari Povray.

\textgreater povstart(light={[}5,-5,5{]},fade=10);

Untuk demonstrasi ini, kami mendefinisikan objek dalam Povray, alih-alih menggunakan string dalam Euler. Definisi langsung ditulis ke berkas.

Koordinat kotak -1 berarti {[}-1,-1,-1{]}.

\textgreater povdefine(``mycube'',povbox(-1,1));

Kita dapat menggunakan objek ini dalam povobject(), yang mengembalikan string seperti biasa.

\textgreater c1=povobject(``mycube'',povlook(red));

Kita buat kubus kedua, lalu putar dan ubah skalanya sedikit.

\textgreater c2=povobject(``mycube'',povlook(yellow),translate={[}1,1,1{]}, \ldots{}\\
\textgreater{} rotate=xrotate(10°)+yrotate(10°), scale=1.2);

Lalu kita ambil selisih kedua benda tersebut.

\textgreater writeln(povdifference(c1,c2));

Sekarang tambahkan tiga sumbu.

\textgreater writeAxis(-1.2,1.2,axis=1); \ldots{}\\
\textgreater{} writeAxis(-1.2,1.2,axis=2); \ldots{}\\
\textgreater{} writeAxis(-1.2,1.2,axis=4); \ldots{}\\
\textgreater{} povend();

\begin{figure}
\centering
\pandocbounded{\includegraphics[keepaspectratio]{images/EMT4Plot3D_23030630082_Hikmatul Utami-077.png}}
\caption{images/EMT4Plot3D\_23030630082\_Hikmatul\%20Utami-077.png}
\end{figure}

\chapter{Fungsi Implisit}\label{fungsi-implisit}

Povray dapat memplot himpunan di mana f(x,y,z)=0, sama seperti parameter implisit dalam plot3d. Namun, hasilnya terlihat jauh lebih baik.

Sintaks untuk fungsi-fungsi tersebut sedikit berbeda. Anda tidak dapat menggunakan output dari ekspresi Maxima atau Euler.

\[((x^2+y^2-c^2)^2+(z^2-1)^2)*((y^2+z^2-c^2)^2+(x^2-1)^2)*((z^2+x^2-c^2)^2+(y^2-1)^2)=d\]\textgreater povstart(angle=70°,height=50°,zoom=4);

\textgreater c=0.1; d=0.1; \ldots{}\\
\textgreater{} writeln(povsurface(``(pow(pow(x,2)+pow(y,2)-pow(c,2),2)+pow(pow(z,2)-1,2))*(pow(pow(y,2)+pow(z,2)-pow(c,2),2)+pow(pow(x,2)-1,2))*(pow(pow(z,2)+pow(x,2)-pow(c,2),2)+pow(pow(y,2)-1,2))-d'',povlook(red))); \ldots{}\\
\textgreater{} povend();

\begin{verbatim}
Error : Povray error!

Error generated by error() command

povray:
    error("Povray error!");
Try "trace errors" to inspect local variables after errors.
povend:
    povray(file,w,h,aspect,exit); 
\end{verbatim}

\textgreater povstart(angle=25°,height=10°);

\textgreater writeln(povsurface(``pow(x,2)+pow(y,2)*pow(z,2)-1'',povlook(blue),povbox(-2,2,``\,``)));

\textgreater povend();

\begin{figure}
\centering
\pandocbounded{\includegraphics[keepaspectratio]{images/EMT4Plot3D_23030630082_Hikmatul Utami-079.png}}
\caption{images/EMT4Plot3D\_23030630082\_Hikmatul\%20Utami-079.png}
\end{figure}

\textgreater povstart(angle=70°,height=50°,zoom=4);

Buat permukaan implisit. Perhatikan sintaksis yang berbeda dalam ekspresi.

\textgreater writeln(povsurface(``pow(x,2)*y-pow(y,3)-pow(z,2)'',povlook(green))); \ldots{}\\
\textgreater{} writeAxes(); \ldots{}\\
\textgreater{} povend();

\begin{figure}
\centering
\pandocbounded{\includegraphics[keepaspectratio]{images/EMT4Plot3D_23030630082_Hikmatul Utami-080.png}}
\caption{images/EMT4Plot3D\_23030630082\_Hikmatul\%20Utami-080.png}
\end{figure}

\chapter{Objek Mesh}\label{objek-mesh}

Dalam contoh ini, kami menunjukkan cara membuat objek mesh, dan menggambarnya dengan informasi tambahan.

Kami ingin memaksimalkan xy dalam kondisi x+y=1 dan menunjukkan sentuhan tangensial garis-garis level.

\textgreater povstart(angle=-10°,center={[}0.5,0.5,0.5{]},zoom=7);

Kita tidak dapat menyimpan objek dalam string seperti sebelumnya, karena terlalu besar. Jadi kita mendefinisikan objek dalam file Povray menggunakan \#declare. Fungsi povtriangle() melakukan ini secara otomatis. Fungsi ini dapat menerima vektor normal seperti pov3d().

Berikut ini mendefinisikan objek mesh, dan langsung menuliskannya ke dalam file.

\textgreater x=0:0.02:1; y=x'; z=x*y; vx=-y; vy=-x; vz=1;

\textgreater mesh=povtriangles(x,y,z,``\,``,vx,vy,vz);

Sekarang kita definisikan dua cakram, yang akan berpotongan dengan permukaan.

\textgreater cl=povdisc({[}0.5,0.5,0{]},{[}1,1,0{]},2); \ldots{}\\
\textgreater{} ll=povdisc({[}0,0,1/4{]},{[}0,0,1{]},2);

Tuliskan permukaan dikurangi kedua cakram.

\textgreater writeln(povdifference(mesh,povunion({[}cl,ll{]}),povlook(green)));

Tuliskan dua titik potongnya.

\textgreater writeln(povintersection({[}mesh,cl{]},povlook(red))); \ldots{}\\
\textgreater{} writeln(povintersection({[}mesh,ll{]},povlook(gray)));

Tuliskan titik maksimumnya.

\textgreater writeln(povpoint({[}1/2,1/2,1/4{]},povlook(gray),size=2*defaultpointsize));

Tambahkan sumbu dan selesaikan.

\textgreater writeAxes(0,1,0,1,0,1,d=0.015); \ldots{}\\
\textgreater{} povend();

\begin{figure}
\centering
\pandocbounded{\includegraphics[keepaspectratio]{images/EMT4Plot3D_23030630082_Hikmatul Utami-081.png}}
\caption{images/EMT4Plot3D\_23030630082\_Hikmatul\%20Utami-081.png}
\end{figure}

\chapter{Anaglif dalam Povray}\label{anaglif-dalam-povray}

Untuk menghasilkan anaglif untuk kacamata merah/sian, Povray harus dijalankan dua kali dari posisi kamera yang berbeda. Ia menghasilkan dua file Povray dan dua file PNG, yang dimuat dengan fungsi loadanaglyph().

Tentu saja, Anda memerlukan kacamata merah/sian untuk melihat contoh berikut dengan benar.

Fungsi pov3d() memiliki sakelar sederhana untuk menghasilkan anaglif.

\textgreater pov3d(``-exp(-x\textsuperscript{2-y}2)/2'',r=2,height=45°,\textgreater anaglyph, \ldots{}\\
\textgreater{} center={[}0,0,0.5{]},zoom=3.5);

\begin{figure}
\centering
\pandocbounded{\includegraphics[keepaspectratio]{images/EMT4Plot3D_23030630082_Hikmatul Utami-082.png}}
\caption{images/EMT4Plot3D\_23030630082\_Hikmatul\%20Utami-082.png}
\end{figure}

Jika Anda membuat suatu pemandangan dengan objek, Anda perlu memasukkan pembuatan pemandangan tersebut ke dalam suatu fungsi, dan menjalankannya dua kali dengan nilai yang berbeda untuk parameter anaglyph.

\textgreater function myscene \ldots{}

\begin{verbatim}
  s=povsphere(povc,1);
  cl=povcylinder(-povz,povz,0.5);
  clx=povobject(cl,rotate=xrotate(90°));
  cly=povobject(cl,rotate=yrotate(90°));
  c=povbox([-1,-1,0],1);
  un=povunion([cl,clx,cly,c]);
  obj=povdifference(s,un,povlook(red));
  writeln(obj);
  writeAxes();
endfunction
\end{verbatim}

Fungsi povanaglyph() melakukan semua ini. Parameternya seperti pada povstart() dan povend() yang digabungkan.

\textgreater povanaglyph(``myscene'',zoom=4.5);

\begin{figure}
\centering
\pandocbounded{\includegraphics[keepaspectratio]{images/EMT4Plot3D_23030630082_Hikmatul Utami-083.png}}
\caption{images/EMT4Plot3D\_23030630082\_Hikmatul\%20Utami-083.png}
\end{figure}

\chapter{Menentukan Objek Sendiri}\label{menentukan-objek-sendiri}

Antarmuka povray Euler berisi banyak objek. Namun, Anda tidak terbatas pada objek-objek ini. Anda dapat membuat objek sendiri, yang menggabungkan objek lain, atau objek yang sama sekali baru.

Kami mendemonstrasikan sebuah torus. Perintah Povray untuk ini adalah ``torus''. Jadi, kami mengembalikan string dengan perintah ini dan parameternya. Perhatikan bahwa torus selalu berpusat di titik asal.

\textgreater function povdonat (r1,r2,look=``\,``) \ldots{}

\begin{verbatim}
  return "torus {"+r1+","+r2+look+"}";
endfunction
\end{verbatim}

Inilah torus pertama kita.

\textgreater t1=povdonat(0.8,0.2)

\begin{verbatim}
torus {0.8,0.2}
\end{verbatim}

Mari kita gunakan objek ini untuk membuat torus kedua, diterjemahkan dan diputar.

\textgreater t2=povobject(t1,rotate=xrotate(90°),translate={[}0.8,0,0{]})

\begin{verbatim}
object { torus {0.8,0.2}
 rotate 90 *x 
 translate &lt;0.8,0,0&gt;
 }
\end{verbatim}

Sekarang kita tempatkan objek-objek ini ke dalam sebuah scene. Untuk tampilannya, kita gunakan Phong Shading.

\textgreater povstart(center={[}0.4,0,0{]},angle=0°,zoom=3.8,aspect=1.5); \ldots{}\\
\textgreater{} writeln(povobject(t1,povlook(green,phong=1))); \ldots{}\\
\textgreater{} writeln(povobject(t2,povlook(green,phong=1))); \ldots{}\\
\textgreater{}\\
\textgreater povend();

memanggil program Povray. Namun, jika terjadi kesalahan, program tersebut tidak menampilkan kesalahan tersebut. Oleh karena itu, Anda harus menggunakan

\textgreater povend(\textless exit);

jika ada yang tidak berhasil. Ini akan membiarkan jendela Povray terbuka.

\textgreater povend(h=320,w=480);

\begin{figure}
\centering
\pandocbounded{\includegraphics[keepaspectratio]{images/EMT4Plot3D_23030630082_Hikmatul Utami-084.png}}
\caption{images/EMT4Plot3D\_23030630082\_Hikmatul\%20Utami-084.png}
\end{figure}

Berikut adalah contoh yang lebih rinci. Kami memecahkan

\[Ax \le b, \quad x \ge 0, \quad c.x \to \text{Max.}\]dan menunjukkan titik-titik yang layak dan titik-titik optimum dalam plot 3D.

\textgreater A={[}10,8,4;5,6,8;6,3,2;9,5,6{]};

\textgreater b={[}10,10,10,10{]}';

\textgreater c={[}1,1,1{]};

Pertama, mari kita periksa, apakah contoh ini punya solusi.

\textgreater x=simplex(A,b,c,\textgreater max,\textgreater check)'

\begin{verbatim}
[0,  1,  0.5]
\end{verbatim}

Ya, benar.

Berikutnya kita mendefinisikan dua objek. Yang pertama adalah bidang datar

\[a \cdot x \le b\]\textgreater function oneplane (a,b,look=``\,``) \ldots{}

\begin{verbatim}
  return povplane(a,b,look)
endfunction
\end{verbatim}

Kemudian kita mendefinisikan irisan semua ruang setengah dan sebuah kubus.

\textgreater function adm (A, b, r, look=``\,``) \ldots{}

\begin{verbatim}
  ol=[];
  loop 1 to rows(A); ol=ol|oneplane(A[#],b[#]); end;
  ol=ol|povbox([0,0,0],[r,r,r]);
  return povintersection(ol,look);
endfunction
\end{verbatim}

Sekarang, kita dapat merencanakan adegannya.

\textgreater povstart(angle=120°,center={[}0.5,0.5,0.5{]},zoom=3.5); \ldots{}\\
\textgreater{} writeln(adm(A,b,2,povlook(green,0.4))); \ldots{}\\
\textgreater{} writeAxes(0,1.3,0,1.6,0,1.5); \ldots{}\\
\textgreater{}\\
Berikut ini adalah lingkaran di sekitar titik optimum.

\textgreater writeln(povintersection({[}povsphere(x,0.5),povplane(c,c.x'){]}, \ldots{}\\
\textgreater{} povlook(red,0.9)));

Dan kesalahan dalam arah yang optimum.

\textgreater writeln(povarrow(x,c*0.5,povlook(red)));

Kita menambahkan teks ke layar. Teks hanyalah objek 3D. Kita perlu menempatkan dan memutarnya sesuai dengan pandangan kita.

\textgreater writeln(povtext(``Linear Problem'',{[}0,0.2,1.3{]},size=0.05,rotate=5°)); \ldots{}\\
\textgreater{} povend();

\begin{figure}
\centering
\pandocbounded{\includegraphics[keepaspectratio]{images/EMT4Plot3D_23030630082_Hikmatul Utami-087.png}}
\caption{images/EMT4Plot3D\_23030630082\_Hikmatul\%20Utami-087.png}
\end{figure}

\chapter{Contoh Lainnya}\label{contoh-lainnya}

Anda dapat menemukan beberapa contoh lainnya untuk Povray di Euler dalam berkas berikut.

Lihat: Contoh/Bola Dandelin

Lihat: Contoh/Donat Math

Lihat: Contoh/Simpul Trefoil

Lihat: Contoh/Optimasi dengan Penskalaan Afinitas

\chapter{Latihan}\label{latihan}

CONTOH BOLA DANDELIN

Diberikan dua garis g1 dan g2 yang membentuk sebuah kerucut dengan titik puncak di (0,0). Garis-garis tersebut diberikan oleh persamaan:

g1 : lineThrough({[}0,0{]},{[}1,b{]}) dan g2 = lineThrough({[}0,0{]},{[}-1,b{]})

Sebuah garis ketiga g memotong kerucut tersebut, dengan persamaan garis:

g:lineThrough({[}-1,0{]},{[}1,1{]})

Titik P terletak di sumbu y dengan koordinat {[}0,v{]}.

\begin{enumerate}
\def\labelenumi{\arabic{enumi}.}
\item
  Tentukan jarak antara titik P dan garis g1.
\item
  Tentukan jarak antara titik P dan garis g1.
\item
  Temukan nilai v dimana jarak P ke g1 sama dengan jarak dari P ke g.
\item
  Gambarkan dua lingkaran dengan pusat di sumbu y menggunakan hari langkah ke 3.
\end{enumerate}

Penyelesaian :

\textgreater load geometry;

\textgreater g1 \&= lineThrough({[}0,0{]},{[}1,b{]})

\begin{verbatim}
                             [- b, 1, 0]
\end{verbatim}

\textgreater g2 \&= lineThrough({[}0,0{]},{[}-1,b{]})

\begin{verbatim}
                            [- b, - 1, 0]
\end{verbatim}

\textgreater g \&= lineThrough({[}-1,0{]},{[}1,1{]})

\begin{verbatim}
                             [- 1, 2, 1]
\end{verbatim}

\textgreater setPlotRange(-1,1,0,2);

\textgreater color(black); plotLine(g(),``\,``)

\textgreater b:=2; color(green); plotLine(g1(),``\,``), plotLine(g2(),''\,``):

\begin{figure}
\centering
\pandocbounded{\includegraphics[keepaspectratio]{images/EMT4Plot3D_23030630082_Hikmatul Utami-088.png}}
\caption{images/EMT4Plot3D\_23030630082\_Hikmatul\%20Utami-088.png}
\end{figure}

\textgreater P \&= {[}0,v{]}

\begin{verbatim}
                                [0, v]
\end{verbatim}

\textgreater d1 \&= distance(P,projectToLine(P,g1))

\begin{verbatim}
                           2               2  2
                          b  v      2     b  v
                   sqrt((------ - v)  + ---------)
                          2               2     2
                         b  + 1         (b  + 1)
\end{verbatim}

\textgreater d \&= distance(P,projectToLine(P,g))

\begin{verbatim}
                                                2
                         v + 2     2   (2 v - 1)
                   sqrt((----- - v)  + ----------)
                           5               25
\end{verbatim}

\textgreater sol \&= solve(d1\textsuperscript{2=d}2,v)

\begin{verbatim}
                             2           2
             - sqrt(5) sqrt(b  + 1) + 2 b  + 2
        [v = ---------------------------------, 
                            2
                         4 b  - 1
                                                    2           2
                                      sqrt(5) sqrt(b  + 1) + 2 b  + 2
                                  v = -------------------------------]
                                                    2
                                                 4 b  - 1
\end{verbatim}

\textgreater v := sol()

\begin{verbatim}
[0.333333,  1]
\end{verbatim}

\textgreater dd := d()

\begin{verbatim}
[0.149071,  0.447214]
\end{verbatim}

\textgreater color(red);

\textgreater plotCircle(circleWithCenter({[}0,v{[}1{]}{]},dd{[}1{]}),``\,``);

\textgreater plotCircle(circleWithCenter({[}0,v{[}2{]}{]},dd{[}2{]}),``\,``);

\textgreater insimg;

\begin{figure}
\centering
\pandocbounded{\includegraphics[keepaspectratio]{images/EMT4Plot3D_23030630082_Hikmatul Utami-089.png}}
\caption{images/EMT4Plot3D\_23030630082\_Hikmatul\%20Utami-089.png}
\end{figure}

CONTOH TREFOIL KNOT

\textgreater load povray;

\textgreater u:=linspace(-pi,pi,160); v:=linspace(-pi,pi,400)';

\textgreater x:=(4*(1+.25*sin(3*v))+cos(u))*cos(2*v);

\textgreater y:=(4*(1+.25*sin(3*v))+cos(u))*sin(2*v);

\textgreater z:=sin(u)+2*cos(3*v);

\textgreater plot3d(x,y,z,frame=0,scale=1.5,hue=1,light={[}1,0,-1{]},zoom=3.2):

\begin{figure}
\centering
\pandocbounded{\includegraphics[keepaspectratio]{images/EMT4Plot3D_23030630082_Hikmatul Utami-090.png}}
\caption{images/EMT4Plot3D\_23030630082\_Hikmatul\%20Utami-090.png}
\end{figure}

\textgreater plot3d(x,y,z,frame=0,scale=1.5,hue=1,light={[}1,0,-1{]},anaglyph=1,zoom=3.2):

\begin{figure}
\centering
\pandocbounded{\includegraphics[keepaspectratio]{images/EMT4Plot3D_23030630082_Hikmatul Utami-091.png}}
\caption{images/EMT4Plot3D\_23030630082\_Hikmatul\%20Utami-091.png}
\end{figure}

\textgreater x:=(4*(1+.4*sin(5*v))+cos(u))*cos(2*v);

\textgreater y:=(4*(1+.4*sin(5*v))+cos(u))*sin(2*v); z=sin(u)+2*cos(5*v);

\textgreater plot3d(x,y,z,frame=0,scale=1.5,hue=1,light={[}1,0,-1{]},zoom=3,anaglyph=1):

\begin{figure}
\centering
\pandocbounded{\includegraphics[keepaspectratio]{images/EMT4Plot3D_23030630082_Hikmatul Utami-092.png}}
\caption{images/EMT4Plot3D\_23030630082\_Hikmatul\%20Utami-092.png}
\end{figure}

Diberikan fungsi

buatlah grafik permukaan dari fungsi tersebut untuk x dalam rentang {[}-1,1{]}. Dengan grafik yang akan memiliki 5 garis horizontal dan 5 garis vertikal.

PENYELESAIAN :

\textgreater plot3d(``x\^{}2+1'',a=-1,b=1,rotate=true,grid=5):

\begin{figure}
\centering
\pandocbounded{\includegraphics[keepaspectratio]{images/EMT4Plot3D_23030630082_Hikmatul Utami-093.png}}
\caption{images/EMT4Plot3D\_23030630082\_Hikmatul\%20Utami-093.png}
\end{figure}

\backmatter
\end{document}
